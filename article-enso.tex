% !TeX root = article-enso.tex

\listfiles
\documentclass[review, 12pt]{elsarticle}

\usepackage{color}
\usepackage{subfig}
\usepackage{gensymb}
\usepackage{graphics}
\usepackage{graphicx}

\usepackage[colorlinks=true]{hyperref}
\usepackage{lineno}
\modulolinenumbers[5]

\journal{Journal of \LaTeX\ Templates}

%%%%%%%%%%%%%%%%%%%%%%%
%% Elsevier bibliography styles
%%%%%%%%%%%%%%%%%%%%%%%
%% To change the style, put a % in front of the second line of the current style and
%% remove the % from the second line of the style you would like to use.
%%%%%%%%%%%%%%%%%%%%%%%

% Numbered
% \bibliographystyle{model1-num-names}

%% Numbered without titles
% \bibliographystyle{model1a-num-names}

%% Harvard
% \bibliographystyle{model2-names}\biboptions{authoryear}

%% Vancouver numbered
% \usepackage{numcompress}\bibliographystyle{model3-num-names}

%% Vancouver name/year
% \usepackage{numcompress}\bibliographystyle{model4-names}\biboptions{authoryear}

%% APA style
\bibliographystyle{model5-names}\biboptions{authoryear}

\hypersetup{
  urlcolor     = blue, %Colour for external hyperlinks
  linkcolor    = blue, %Colour of internal links
  citecolor   = red, % Colour of citations
  %hidelinks
}

% Dossier des figures 
%\graphicspath{{figs/}{../figs/}}
\graphicspath{ {figs/} }

% Liste des extensions de figures (pour pdflatex)
\DeclareGraphicsExtensions{.pdf,.png,.jpg,.tiff}

%% AMA style
% \usepackage{numcompress}\bibliographystyle{model6-num-names}

%% `Elsevier LaTeX' style, distributed in TeX Live 2019
%\bibliographystyle{elsarticle-num}
% \usepackage{numcompress}\bibliographystyle{elsarticle-num-names}
% \bibliographystyle{elsarticle-harv}\biboptions{authoryear}
%%%%%%%%%%%%%%%%%%%%%%%


\newcommand{\warn}[1]{\textbf{\color{red}{#1}}}
\newcommand\nino{El Ni\~no}
\newcommand\nina{La Ni\~na}
\newcommand\ap{APECOSM}
\newcommand\hov{Hovm\"oller}
\newcommand\ppar{photosynthetically active radiation}
\newcommand\omz{oxygen minimum zone}
\newcommand\phy{nano-phytoplankton}
\newcommand\Phy{Nano-phytoplankton}
\newcommand{\phyd}{diatoms}
\newcommand{\Phyd}{Diatoms}
\newcommand\zoo{micro-zooplankton}
\newcommand\Zoo{Micro-zooplankton}
\newcommand{\zood}{meso-zooplankton}
\newcommand{\Zood}{Meso-zooplankton}
\newcommand{\goc}{big organic carbon}
\newcommand{\Goc}{Big organic carbon}
\newcommand{\cpn}{Central Pacific \nino}
\newcommand{\epn}{Eastern Pacific \nino}
\newcommand{\sst}{sea-surface temperature}




\begin{document}


\begin{frontmatter}

\title{Mechanisms underlying the epipelagic ecosystem response to ENSO variability in the equatorial Pacific ocean}
%\tnotetext[mytitlenote]{Fully documented templates are available in the elsarticle package on \href{http://www.ctan.org/tex-archive/macros/latex/contrib/elsarticle}{CTAN}.}

%% Group authors per affiliation:
\author[mymainaddress]{Nicolas Barrier\corref{mycorrespondingauthor}}
%\address{Radarweg 29, Amsterdam}
\fntext[myfootnote]{MARBEC, Univ. Montpellier, CNRS, Ifremer, IRD, S\`ete, France}

\author[mymainaddress]{Matthieu Lengaigne}
\author[mymainaddress]{Jonathan Rault}
\author[renaud]{Renaud Person}
\author[chris]{Christian Eth\'{e}}
\author[mymainaddress]{Olivier Maury}

\cortext[mycorrespondingauthor]{Corresponding author}	

%% or include affiliations in footnotes:
%\author[mymainaddress,mysecondaryaddress]{Elsevier Inc}
%\ead[url]{www.elsevier.com}
%\author[mysecondaryaddress]{Global Customer Service\corref{mycorrespondingauthor}}
%\ead{support@elsevier.com}
\address[mymainaddress]{MARBEC, Univ. Montpellier, CNRS, Ifremer, IRD, Sète, France}
\address[renaud]{LOCEAN, IRD}
\address[chris]{IPSL, CNRS}

\begin{abstract}

\nino{}/Southern Oscillation, which is the Earth’s most energetic year-to-year climate event, has a significant impact on the physical and biogeochemical properties
of the tropical Pacific Ocean. In turn, it strongly affects the high trophic level organisms, especially the epipelagic fishes. 
Understanding the impact of \nino{} events on marine ecosystems and the associated mechanisms is an issue at stake, especially so for the countries that border the tropical Pacific and exploit its marine resources.

This question is addressed by using the mechanistic, DEB-based ecosystem model APECOSM, which is forced by a coupled physical/biogeochemical hindcast simulation over the 1958-2018 period. First, the simulated response to ENSO variability is compared with observations. Then, the contributions of the key processes to changes in epipelagic fish biomass in response to extreme \nino{} events are investigated. Our results suggest that advection and diffusion play a major role, especially so for larger organisms, whose biological response is very small. Smaller organisms show a strong biological response through predation and growth, with these two components compensating each other. 

\end{abstract}

\begin{keyword}
fish \sep biomass \sep ENSO \sep \nino \sep \nina \sep modelling
\end{keyword}

\end{frontmatter}

\linenumbers

%\section{Introduction}
%
%Assessing the impacts of \nino\ events on the fish biomass is a valuable way to infer the future state of ocean in warming ocean.\\
%
%\nino\ events are not comparable to each other, they have their own characteristics.
%\newpage

\section{Introduction}

\textbf{Societal relevance.} Improving our understanding of how climate variability and change impacts marine ecosystems is a critical issue for the countries bordering the tropical Pacific and harvesting its marine resources. Marine ecosystems in the tropical Pacific Ocean underpin a variety of small-scale artisanal fisheries that are critical for food security and livelihood in most tropical Pacific islands. They also support major oceanic fisheries that are responsible for 60\% of the world tuna catches, which substantially contribute to the revenue of most Pacific Island Countries and Territories. 

\textbf{Tropical Pacific climatology.} The physics, biogeochemistry and biology are tightly coupled in the tropical Pacific, which makes it an ideal region to study biophysical interactions. This oceanic region is characterized by prevalent trade wind conditions, which accumulate a pool of warm water in the western equatorial Pacific and induce an upwelling of cold nutrient-rich subsurface waters that cools the entire central and eastern equatorial Pacific. The Eastern Tropical Pacific upwelling of, leading to an oligotrophic west Pacific warm pool and a mesotrophic, upwelling cold tongue (e.g., \cite{leborgneCarbonFluxesEquatorial2002}). Tropical skipjack and yellowfin tunas mainly inhabit and reproduce in warm surface waters above 26°C in the western and northeast Pacific and feed opportunistically on a wide range of small epipelagic planktonic and nektonic prey. Smaller quantities of the tropical skipjack, yellowfin and bigeye tunas (Thunnus obesus) and the temperate albacore tuna (Thunnus alalunga) are also taken by industrial longliners off equator.

\textbf{ENSO physical and biogeochemical response.} The tropical Pacific is also home to the most energetic earth’s year-to-year climate variations, the El Niño/ Southern Oscillation (ENSO) (\cite{caiChangingNinoSouthern2021}), with strong impacts on the physical and biogeochemical properties of the tropical Pacific Ocean.  An El Niño event (i.e. the warm phase of ENSO) is characterized by a thermocline and nutricline deepening in the central/eastern Pacific, which reduces of upward vertical nutrient and cold water flux, leading to anomalously warm sea surface temperature anomaly (SSTA) and reduced productivity there  (e.g. \cite{chavezBiologicalChemicalResponse1999a, murtuguddeOceanColorVariability1999}). In contrast, the opposite stands for the western Pacific Ocean, which experiences a thermocline/nutricline shoaling resulting in a weak cooling and a productivity increase. The zonal eastward advection of nutrient‐poor warm pool waters by anomalous eastward currents further contributes to the decrease in biological productivity in the central Pacific (e.g. \cite{chavezBiologicalChemicalResponse1999a, picautOceanicZoneConvergence2001}). La Niña (i.e. the cold ENSO phase) can broadly be viewed as a mirror image of El Niño despite some asymmetrical features: El Niño events can indeed occasionally reach much larger amplitudes than La Niña events, like in 1982, 1997 and 2015.  

\textbf{Synchronous ecological response to ENSO.} These physical and biogeochemical impacts of ENSO variability ultimately cascade on the fish distribution, either through changes in habitat conditions (oxygen, temperature) and/or food availability (\cite{aNinoSouthernOscillation2020}). For instance, the spatial distribution of purse seine catches in the western Pacific is strongly influenced by ENSO events, with catches typically expanding to the east during El Niño events and shifting back to the west during La Niña years (\cite{lehodeyNinoSouthernOscillation1997}), in conjunction with the migration of the convergence zone at the eastern edge of the warm pool, where important aggregating mechanisms occur for phytoplantkon (\cite{picautOceanicZoneConvergence2001}). The strong temperature gradient of the thermocline exerts a strong control on the vertical tuna habitat (e.g. \cite{schaeferMovementsBehaviorHabitat2002}) and its shoaling in the western Pacific during El Niño may squeeze their thermal and feeding vertical habitat, favouring their catchability (\cite{bertrandHydrologicalTrophicCharacteristics2002}). 

\textbf{Delayed ecological response to ENSO.} ENSO may not only have a synchronous effect on ecosystems through horizontal and vertical displacement of fish population but also a delayed impact on fish abundance by affecting the survival of larvae, which variability propagates through the population structure and can be detected with some delay in the adult population. Despite the absence of direct larvae abundance surveys, higher skipjack recruitments has been reported two to four years following a bloom in the central Pacific, suggesting that La Niña-related high primary productivity is favorable to nursery and feeding conditions (\cite{yenSpatialTemporalVariations2016}). El Niño has also been reported to negatively impact gonad maturation and mean length of skipjack tuna with time lags of 12 and 7 months, respectively (\cite{kimEffectsClimateinducedVariation2015}). For long‐living species, such as bigeye and albacore, the decreasing growth rate with age leads to a smoothed and damped recruitment variability associated with ENSO, while it combines with the internal dynamic processes of the species propagating toward the older cohorts (e.g. \cite{seninaImpactsRecentHigh2017}). Catches of albacore have however been reported to decrease between 4 to 8 years after an El Niño onset south of 10°S, a time interval that would be expected before the recruitment for the fish spawned during ENSO episodes (\cite{luRelationshipNinoSouthern1998}). A similar relationship has also been found in the northern Pacific (\cite{zhangStudyRelationshipsLargescale2014}), a stock productivity decrease lagging by four years El Niño events in this region.  Given that all tuna species spawn in tropical warm waters, it can be assumed that their spawning habitats and the subsequent fish recruitment are all impacted by ENSO variability (\cite{zhangStudyRelationshipsLargescale2014}), resulting in a delayed fluctuation in abundance of poleward moving adult tuna. This may also be the case for the Japanese eels spawning in the subtropical northwestern Pacific, where surface currents changes associated with ENSO could impact larval transport and the recruitment of juveniles in the northwestern Pacific (\cite{hsiungEffectENSOEvents2018}).

\textbf{Lower-frequency variations.} Marine populations dynamics do not only lag but may also integrate ENSO interannual variations, resulting in a marine population response at lower decadal timescales, characterized by potentially strong and prolonged apparent state transitions (\cite{lorenzoDoubleintegrationHypothesisExplain2013a}). Decadal to multi-decadal variations tuna population have indeed been partly attributed to ENSO-related low-frequency climate variability (e.g. \cite{lehodeyModellingClimaterelatedVariability2003,  singhImpactClimaticFactors2015, singhENVIRONMENTALCONDITIONSARE2017, wuDeterminingEffectMultiscale2020a}), such as the Pacific Decadal Oscillation (PDO). For instance, the PDO is leading by four to five years the recruitment and spawning stock biomass of albacore tuna in both the south (Lehodey et al. 2003; Singh et al. 2015) and the north Pacific (\cite{singhENVIRONMENTALCONDITIONSARE2017}) and by one to five years the yellowfin tuna recruitment in the western Pacific Ocean (\cite{wuDeterminingEffectMultiscale2020a}). The PDO is also leading by three years the recruitment of big-eye tuna mature cohorts in the western central Pacific Ocean (\cite{lanEffectsClimateChange2021}).  This has been related to synchronous shifts in the pelagic ecosystems at low trophic levels (eggs, Pacific saury, neon flying squid) and immature cohort of big-eye tuna with the PDO.

\textbf{Ecosystem modelling.} Most of the observational studies discussing the influence of tropical climate variability on marine ecosystems in the equatorial Pacific mainly rely on tuna catch data, which variability is not only controlled by climate variations but also by fishing pressure and internal ecosystem dynamics. These observations are thus heterogeneous, limited to the surface, focused on a few adult species and their embedded climate signals are aliased by other external factors. Several ecosystem models gathered in the Fisheries and Marine Ecosystem Model Intercomparison Project (Fish-MIP) have been developed to better characterize and understand ecosystems response to climate variations. These models have mainly been used to project biomass changes in response to climate change, generally pointing to a global decline of marine biomass more pronounced for higher trophic levels in response to temperature increase and primary production decrease (\cite{lotzeGlobalEnsembleProjections2019a}). While these models are now routinely used to project future biomass changes, very few studies did attempt to understand how these ecosystem models respond to past climate variability. This is however a must because (1) an accurate representation of past fish biomass variations would allow a more confident assessment of future projections and (2) a better understanding of the processes response for past variability would provide some guidance to understand future changes. To date, a single ecosystem model  (SEAPODYM, \cite{lehodeySpatialEcosystemPopulations2008}) has been used to evaluate the ecosystem response to ENSO in the tropical Pacific, focusing on tuna spatial population. This model reproduces the large-scale east-west displacement of skipjack tuna population in the equatorial Pacific in response to ENSO and attributes these migrations to ENSO-related changes in temperature, prey and oxygen concentrations. It also suggests that El Niño not only drives an eastward tuna displacement but also favour a strong larvae recruitment (\cite{seninaParameterEstimationBasinscale2008}). 

The modeling studies described in the above, however, have some drawbacks. First, all the models used simulate the marine ecosystem in 2D, although the impacts of El Nino events on the physical and biogeochemical ocean are 3D (shoaling/weakening of the thermocline and the relation with oxygen for instance, (\cite{leungENSODrivesNearsurface2019}). Add more limitations? The Apecosm model addresses these different drawbacks. Complete advantages of Apecosm (mechanistic, 3D, advection/diffusion)

The aim of the present simulation is to analyse the impacts of ENSO-related climate variations on the fish biomass variability simulated by Apecosm. The paper is organized as follows. First, the El Nino indexes, the statistical methods and the model simulations are described in section 2. The synchronous response of epipelagic fish biomass to ENSO variability is investigated in section 3, while the delayed and low frequency response is addressed in section 4. Discussions and perspectives are finally provided in section 5.

%Improving our understanding of how climate variability impacts marine ecosystems and  how this will change in the future is a critical issue for the countries bordering the tropical Pacific and harvesting its marine resources. Marine ecosystems in the Western and Central Pacific Ocean (WCPO) are indeed supporting major oceanic fisheries that are responsible for 60% of the world tuna catches, nearly 3 million metric tons worth almost $7 billion each year and they substantially contribute to the revenue of most Pacific Island Countries and Territories. They also underpin a variety of small-scale artisanal fisheries that are critical for food security and livelihood in most tropical Pacific islands. Small-scale fleets operate in the coastal waters of the tropical Pacific States while industrial purse-seine, pole-and-line and long-line fleets operate both in their exclusive economic zones (EEZs) and in international waters. Highly migratory top-predators such as the tropical skipjack and yellowfin tunas (Katsuwonus pelamis and Thunnus albacares) constitute the bulk of the catch in the WCPO region. They are mostly harvested by industrial purse seine fisheries in the equatorial zone (10\degree{}N-10\degree{}S), where they inhabit and reproduce in warm surface waters above 26\degree{}C and feed opportunistically on a wide range of small epipelagic planktonic and nektonic prey. Smaller quantities of the tropical skipjack, yellowfin and bigeye tunas (Thunnus obesus) and the temperate albacore tuna (Thunnus alalunga) are also taken by industrial longliners in the WCPO up to 20\degree{}N for bigeye and 20\degree{}S for albacore. \\ 
%
%The spatial distribution of purse seine catches in the WCPO is strongly influenced by ENSO events, with catches typically expanding to the east during El Niño years and shifting back to the west during La Niña years (Williams and Terawasi 2014). 
%The ocean physics, biogeochemistry and biology are tightly coupled in the tropical Pacific, which makes it an ideal region to study biophysical interactions. This oceanic region is characterized by prevalent trade wind conditions, which accumulate a pool of warm water in the western equatorial Pacific and induce an upwelling of cold nutrient-rich subsurface waters that cools the entire central and eastern equatorial Pacific. The Eastern Tropical Pacific (ETP) upwelling of, leading to an oligotrophic west Pacific warm pool and a mesotrophic, upwelling cold tongue (e.g., Le Borgne et al. 2002). \warn{INCLUDE A BRIEF DESCRIPTION OF THE MAIN EPIPELAGIC FISH DISTRIBUTION IN THE TROPICAL PACIFIC.}
%
%The tropical Pacific is also home to the most energetic earth’s year-to-year climate variations, the El Niño/ Southern Oscillation (ENSO) (Cai et al., 2015). This phenomenon affects climate worldwide through atmospheric teleconnections (Taschetto et al., 2020), causing droughts and floods (Goddard and Gershunov 2020), modulating globally-averaged annual surface air temperature, tropical cyclone activity (Lin et al., 2020) and impacting agriculture and ecosystems  worldwide (Bertrand et al., 2020; Holbrook et al., 2020).  An El Niño event, the warm phase of ENSO, develops as the result of a positive ocean–atmosphere feedback that was first suggested by Bjerknes (1969). In this positive feedback loop, an anomalously warm sea surface temperature anomaly (SSTA) in the central/eastern Pacific promotes enhanced deep atmospheric convection and westerly wind anomalies in the central Pacific (Gill, 1980). This wind anomaly drives anomalous eastward surface currents that pushes the warm pool eastward and a deepening of the thermocline in the equatorial central/eastern Pacific, which both reinforce the initial warming. An El Niño event generally starts in late spring, peak at the end of the calendar year and recede during the following winter. La Niña (i.e. the cold ENSO phase) can broadly be viewed as a mirror image of El Niño despite some asymmetrical features: El Niño events can indeed occasionally reach much larger amplitudes than La Niña events, like in 1982, 1997 and 2015.  \\
%
%ENSO not only strongly impacts the physical properties of the tropical Pacific Ocean but also its biogeochemical properties. The dominant mode of global primary production anomalies as estimated from satellites is indeed directly related to ENSO (e.g. Behrenfeld et al., 2006; Chavez et al., 2010; Méssié and Chavez, 2012), with the strongest imprint in the tropical regions. These biological changes during ENSO are largely driven by changes in nutrient supply through vertical nutricline movements and horizontal advective processes (e.g. Barber and Chavez, 1983; Chavez et al. 1999; Wilson and Adamec, 2001; Christian et al., 2002; Ryan et al. 2002), other processes such as changes in stratification or light availability being of secondary importance. El Niño events are generally characterized by a productivity decrease in the eastern and central Pacific and an increase in the western Pacific (Chavez et al. 1999; Murtugudde et al. 1999; Gierach et al., 2012; Radenac et al., 2012). This productivity increase in the eastern equatorial Pacific have been related to a reduction of upward vertical nutrient flux in response to the equatorial thermocline/nutricline depth deepening observed during El Niño events, while the opposite stands in the western equatorial Pacific (e.g., Chavez et al., 1999; Wilson and Adamec, 2001; Radenac et al., 2012). In addition, the zonal eastward advection of nutrient‐poor warm pool waters by anomalous eastward currents further contributes to the decrease in biological productivity in the central Pacific (e.g. Chavez et al., 1999; Picaut et al., 2001). During La Nina events, enhanced equatorial upwelling in the central and eastern Pacific lifts deep nutrients into the photic zone, resulting in sometimes spectacular large-scale phytoplankton blooms throughout the cold tongue (e.g. Chavez et al., 1999; Ryan et al., 2002; Gorgues et al., 2010). ENSO not only affects productivity in the equatorial Pacific region but also along the south American west coast. Equatorial downwelling Kelvin waves during El Niño events propagate along this coast as coastally-trapped downwelling Kelvin waves,  inducing a deepening of the thermocline, nutricline and oxycline along the south American coast (Leung et al., 2019)  (Espinoza-Morriberón et al., 2019). \\
%
%These impacts of ENSO variability on the physical and biogeochemical properties of the tropical Pacific ultimately impacts the fish distribution, either through changes in habitat conditions (oxygen, temperature) and/or food availability (Bertrand et al. 2020). For instance, ENSO is known to be a major driver of location and related catches of four prominent tuna species in the Pacific (e.g. Bertrand et al. 2020; Lehodey et al., 2006, Nichols et al., 2014): skipjack (Lehodey et al., 1997; Kim et al., 2020), yellowfin, bigeye (Lu et al. 2001), and albacore (Lu et al. 1998; Kimura et al. 1997; Briand et al., 2011). ENSO directly influences tuna horizontal movements (Lehodey et al., 2020). For example, skipjack tuna catch shift eastward from the western to the central equatorial Pacific during  El  Niño events in response to the eastward migration of the convergence zone at the eastern edge of the warm pool (Lehodey et al., 1997, 2011), where important aggregating mechanisms occur for phytoplantkon (Picaut et al., 2001). El Niño events however generally exert an overall negative influence on relative abundance of skipjack tuna when averaged over the western and central Pacific (Yen et al. 2017). In contrast, the skipjack habitat retracts westward during La Niña events in response to warm pool retraction in the far western Pacific. The extent of the warm pool might also be a good indicator for monitoring the effect of environmental variability on yellowfin tuna recruitment (Kirby et al. 2007). Bigeye tuna has also been reported to extend eastward during El Niño years (Yukinawa, et al., 1988). Prolonged and recurring El Niño increases the effort of finding suitable fishing grounds and leads to decreased tuna harvest being landed in the Philippines (Vera and Hipolito, 2006). This also happened to Taiwan mackerel purse seine fishery, which experienced a sharp decrease in harvest by ~50% during the 1997/1998 El Niño (Sun et al., 2006). The strong temperature gradient of the thermocline exerts a strong control on the vertical tuna habitat (Brill et al., 1999; Schaefer et al., 2002) and its shoaling in the western Pacific during El Niño may squeeze their thermal and feeding vertical habitat, favouring their catchability (Lu et al. 1998; Lehodey et al., 2004). The opposite occurs during La Niña. 
%ENSO may not only affect tuna migration and vertical distribution but also the survival of tuna larvae and, therefore, the abundance of tuna populations. Unfortunately, there are no direct abundance surveys, such as the eggs and larvae sampling commonly used for coastal small pelagic stocks, to monitor such large‐scale variability of tropical tuna species larval densities. However, this variability propagates through the population structure and can be detected with some delay in the exploited stock, either through the analysis of catch rates and size frequencies of catch or as inferred from model and stock assessment analyses. High recruitments of skipjack were for instance reported two to four years following a bloom in the central Pacific, suggesting that La Niña-related high primary productivity is favorable to nursery and feeding conditions (Yen and Lu, 2016). El Niño has also been reported to negatively impact gonad maturation and mean length of skipjack tuna with time lags of 12 and 7 months, respectively (Kim et al. 2015). For long‐living species, such as bigeye and albacore, the decreasing growth rate with age and its natural variability over time and space leads to a cohort (age) signal more and more difficult to detect in larger fish. Therefore, the recruitment variability associated with ENSO, i.e. low or high peaks of abundance in the first cohort, is smoothed and damped while it combines with the internal dynamic processes of the species propagating toward the older cohorts (Lehodey et al., 2010; Sibert et al., 2012; Senina et al., 2017). Despite these issues, catches of albacore have been reported to decrease between 4 to 8 years after an El Niño onset south of 10\degree{}S, a time interval that would be expected before the recruitment for the fish spawned during ENSO episodes (Lu et al., 1998). This negative relationship with El Niño has been hypothesized to be related to the extension of warm waters in the central Pacific during El Niño events which reduces the extent of spawning grounds of south Pacific albacore (Lehodey et al. 2003). A similar relationship has also been found in the northern Pacific (Zhang et al. 2014), a stock productivity decrease lagging by four years El Niño events in this region.  Strong El Niño events shift the distribution of small and medium pelagic fishes (anchovy, sardine, mackerel, and jack mackerel) closer to the coast and, in some cases, move into deeper waters (Alheit and Niquen, 2004). They also generally lead to a reduction of anchovy biomass but actual impacts on post-event recovery differ considerably between events (Alheit and Niquen, 2004). \\ 
%
%\emph{
%Discussion on decadal signals: 
%Lehodey et al. (2003): -> In terms of total fluctuations in stock size and recruitment rates, skipjack and yellowfin tuna are positively influenced by the El Niño events and by the warm phases of the Pacific Decadal Oscillation. 
%Suarez-Sanchez et al. (2004): -> Low-frequency yellow-fin tuna variability in the northeastern Pacific over the 1967-1993 period in apparent relation with IPO (but not discussed in the text which only mention the influence of strong EN)
%Pedraza and Diaz-Ochoa, 2006:
%Lima and Naya (2011): -> In addition to ENSO, one‐year lagged negative PDO effects on annual tuna fluctuations in the western tropical Pacific. The negative effects of PDO on skipjack tuna could be associated with the observation that during the warm phases of PDO (i.e. positive PDO values) there is a reduction in the equatorial upwelling and a rise of the sea surface temperatures at the equatorial Pacific Ocean (McPhaden and Zhang 2002). Therefore, this reduction in the ocean productivity could be influencing negatively the recruitment of skipjack tuna that are perceived the next year when the individuals recruit to the fishery. It seems that changes in sea surface temperature and ecosystem processes within the equatorial Pacific Ocean are influenced by the decadal oscillation within the North Pacific Ocean (Linsley et al. 2000, McPhaden and Zhang 2002). In sum, we determined one‐year lagged effects of SOI and PDO that can be related with ecological effects on recruitment, whereas the direct PDO effects could be caused by the effect of oceanographic conditions on tuna catchability. 
%Di Lorenzo and Ohman (2013): Theoretical modeling suggests that the cumulative integrations of white‐noise (high‐frequency) atmospheric forcing can generate red‐noise (low‐frequency) responses in oceanographic variables (as for the PDO) and thus generate marine population responses that are characterized by different regimes and strong transitions.
%Philips et al. (2014): Results indicate that SST had a positive and spatially variable effect on albacore CPUE, with increasingly positive effects to the North, while PDO had an overall negative effect. Although albacore CPUE increased with SST both before and after a threshold year of 1986, such effect geographically shifted north after 1986. This is the first study to demonstrate the non-stationary spatial dynamics of albacore tuna, linked with a major shift of the North Pacific. 
%Nicol et al. (2014): El Niño or La Niña events during multi-year periods, possibly in correlation with the PDO, could lead to regimes of high and low productivity in the tuna population (Kirby et al. 2004; Lehodey et al. 2003; Lehodey et al. 2006). However, particularly strong shifts in the environment were not always detected in tuna recruitment time series (Briand and Kirby 2006), which implies that the relationship between tuna recruitment and climatic oscillations is not linear and might depend on several interrelated factors including the adaptation of spawners to environmental variability.
%Singh et al. (2015): Significant links were established between PDO and albacore time series trajectory from 1957 to 2008 in the South Pacific Ocean.
%Singh et al. (2017): Lehodey et al. (2003) attempted to deduce the mechanisms by which alterations in environmental variables affect the stock of important tuna species in the Pacific. Results indicated that albacore tuna recruitment was significantly affected by the negative and positive phases of PDO. PDO had highly significant correlation with R in the same year and with SSB having a lag period of 5 years. This indicates that variability in the PDO pattern influences the early life stages of the North Pacific albacore tuna. With reference to Figure 3, the relationship of PDO with R and SSB is negative. This means that the negative PDO phase is favorable and positive PDO phase is unfavorable for the larvae and juvenile stages of albacore. Similar explanations have been derived for the South Pacific albacore stock by Lehodey et al. (2003) and Singh et al. (2015).
%Wu et al. 2020: The standardized CPUE in the western Pacific Ocean was significantly correlated to the Pacific Decadal Oscillation (PDO) with a 1–5 year lag. 
%Lehodey et al. (2020): given that all tuna species spawn in tropical warm waters under the influence of ENSO, it can be assumed that their spawning habitats and the subsequent fish recruitment are all impacted by ENSO variability, as has been demonstrated for skipjack tuna (cf section above). The result can be a delayed fluctuation in abundance of juvenile and adult tuna moving to the NWP. This also seems to be the case for the Japanese eels (Anguilla japonica) that spawn in the subtropical NWP, where the intensity of the flow and position of the bifurcation  of the North Equatorial Current change with ENSO and therefore impact larval transport and the recruitment of juveniles in the NWP (Hsiung et al., 2018).\\
%}
%
%Several ecosystem models have been developed to better characterize and understand ecosystems response to climate variability and change. Models gathered in the Fisheries and Marine Ecosystem Model Intercomparison Project (Fish-MIP) have been used to analyse changes in fish biomass induced by climate change projected by climate simulations gathered in the Climate Model Intercomparison Project (CMIP). This project pointed to a global decline of marine biomass, mainly driven by an increase of the mean temperature and by a decrease in the primary production  (Lotze et al., 2019). This biomass change is however spatially inhomogeneous, with a decrease at middle to low latitudes, and an increase at higher latitudes. The biomass response is also more pronounced for higher trophic levels.\\
%
%While these models are now routinely used to project future biomass changes, few studies did attempt to understand how these ecosystem models respond to past climate variability. This is however a must because (1) an accurate representation of past fish biomass variations would allow a more confident assessment of future projections and (2) a better understanding of the processes response for past variability would provide some guidance to understand future changes. A tuna spatial population model (SEAPODYM, Lehodey 2004) has for instance been shown to successfully reproduce the responses of the population dynamics to changes in their physical and biological habitat (Lehodey et al. 2003). Input dataset describing the oceanic environment includes temperature, currents and oxygen concentration averaged over three integrated layers: 0–200 m (epipelagic), 200–500 m (mesopelagic) and 500–1000 m (bathypelagic); and also primary production integrated over all the vertical layers.  This model indicates that the eastward extension of the species and fisheries distributions during El Nino phases are driven by changes in temperature, prey, and dissolved oxygen concentration (Lehodey et al. 2020), and conversely during La Nina conditions (westward contraction) . This model also suggests that ENSO not only drives large-scale displacement of tuna population but can also impact their recruitment and abundance statistics (Lehodey, 2001, Lehodey et al., 2003). For instance, SEAPODYM results also suggest an increasing skipjack and yellowfin tuna recruitment in the central and the western Pacific during El Niño events (Lehodey et al., 2003, 2006; Senina et al., 2008, Langley et al. 2009) that might be a result of four mechanisms: (1) the eastward extension of the warmpool, favouring spawning of these two species, (2) enhanced food for tuna larvae due to higher primary production in the west, (3) lower predation of tuna larvae and (4) larvae retention in these favourable areas as a result of ocean currents.  Similar positive effects of El Ninos on early life stages were detected with SEAPODYM in bigeye and yellowfin tuna species, mainly in the eastern Pacific Ocean. Favorable conditions for larvae survival increase during El Nino events in the eastern Pacific Ocean and decrease in the central region. These species with longer life spans are also more susceptible to present decadal regimes of high and low productivity due to the accumulation of successive low or high peaks of recruitment driven by the decadal modulation of ENSO. A dominance of either El Nino or La Nina events is observed over multiyear periods, possibly in correlation with the Pacific‐scale Interdecadal Pacific Oscillation (IPO).\\
%		
%\warn{Continue modelling studies}
%
%The modeling studies described in the above, however, have some drawbacks. First, all the models used simulate the marine ecosystem in 2D, although the impacts of El Nino events on the physical and biogeochemical ocean are 3D (shoaling/weakening of the thermocline and the relation with oxygen for instance, (Leung et al., 2019)). Add more limitations?
%
%The Apecosm model addresses these different drawbacks. Complete advantages of Apecosm (mechanistic, 3D, advection/diffusion)
%
%The aim of the present simulation is to analyse the impacts of ENSO on the fish biomass variability simulated by the Apecosm model when it is forced by a hindcast physical biogeochemical simulation that covers the 1958-2018 period. The paper is organized as follows. The physical-biogeochemical and ecosystem models used in this study are presented in section 2. The sea-surface temperature and chlorophyll simulated by the coupled physical biogeochemical model will be confronted to observation-based datasets, in order to ensure that it properly simulates the ocean response to ENSO variability (section 3). Then, the response of epipelagic fish biomass to ENSO variability will be investigated (section 4). Discussions and perspectives are provided in section 5.

\section{Numerical models}
\label{sec:model-des}

Numerical models such as the APECOSM model used here are valuable tools to understand the mechanisms that drive the response of oceanic ecosystems to ENSO variability. They usually require physical and biogeochemical forcings (temperature, oxygen, low-trophic levels) as an input. In the present study, we use outputs from a coupled physical and biogeochemical simulation, which is described in \ref{sec:nemo} section. The ecosystem simulation is discussed in \ref{sec:apecosm}.

\subsection{Physical and biogeochemical model}
\label{sec:nemo}

The three-dimensional physical and biogeochemical fields required to run APECOSM are extracted from an oceanic simulation performed with the physical ocean model NEMO (Nucleus for European Modelling of the Ocean, \citealp{madecNEMOOceanEngine2019}) coupled to the ocean biogeochemical model PISCES (Pelagic Interaction Scheme for Carbon and Ecosystem Studies, \citealp{aumontPISCESv2OceanBiogeochemical2015}). 

NEMO simulates the dynamics and thermodynamics of the physical ocean. Prognostic variables are
the three-dimensional velocity field, a non-linear sea surface height, the
conservative temperature and the absolute salinity, distributed on a three-dimensional Arakawa C-type grid. Density is computed from potential temperature, salinity and pressure using the \cite{iocInternationalThermodynamicEquation2010} equation of state. Vertical mixing is parameterized from a turbulence closure scheme based on a prognostic vertical turbulent kinetic equation, which has been shown to perform well in the tropics before \citep{blankeVariabilityTropicalAtlantic1993}. Lateral mixing acts along isopycnal surfaces, with a Laplacian operator and $200 m^2 s^{-1}$ constant isopycnal diffusivity coefficient \citep{lengaigneImpactIsopycnalMixing2003}. Shortwave fluxes penetrate into the ocean based on a single exponential profile \citep{paulsonIrradianceMeasurementsUpper1977} corresponding to oligotrophic water (attenuation depth of 23 m). 

PISCES is a biogeochemical model of intermediate complexity with 24 prognostic variables designed for global ocean applications \citep{aumontPISCESv2OceanBiogeochemical2015}. It simulates the biogeochemical cycles of oxygen, carbon and the main nutrients controlling phytoplankton growth (nitrate, ammonium, phosphate, silicic acid, and iron) and the lower trophic levels of marine ecosystems, distinguishing four plankton functional types based on size: two phytoplankton groups ("small phytoplankton" -e.g. nanophytoplankton- and "large phytoplankton" -e.g. diatoms-) and two zooplankton groups ("small zooplankton" -e.g. microzooplankton- and "large zooplankton" -e.g. mesozooplankton-). It also includes small and large particulate organic matter.

The NEMO-PISCES simulation used in this study is deployed on the tripolar ORCA1 grid \citep{madecGlobalOceanMesh1996}, with a 1\degree{} nominal horizontal resolution and a refined 1/3\degree{} meridional resolution in the equatorial band. Its vertical resolution ranges from 1m at the surface to 100m at 1 kilometer depth and varies over time, following \cite{levierFreeSurfaceVariable2007}. This simulation  is forced over the period 1958-2018 with atmospheric inputs from the JRA atmospheric reanalysis \citep{kobayashiJRA55ReanalysisGeneral2015}, which is representative of  surface atmospheric variability observed over the historical period. 

%Temperature, ocean transports, oxygen, plankton concentration (diatoms, mesozooplankton and microzooplankton, big particulate organic matter), photosynthetically active radiation (PAR) and the layer thickness from this simulation are then used to force the Apecosm ecosystem model.

\subsection{Marine ecosystem model}
\label{sec:apecosm}

We use the Apex Predators Ecosystem Model (Apecosm, \citealp{mauryModelingEnvironmentalEffects2007, mauryOverviewAPECOSMSpatialized2010}) to simulate the energy transfer through marine ecosystems. 
APECOSM is an eulerian ecosystem model that represents mechanistically the 3D dynamics of size-structured pelagic populations and communities. It integrates individual, population and community levels and includes the effects of life-history diversity with a trait-based approach \citep{mauryIndividualsPopulationsCommunities2013}. In APECOSM, the uptake and use of energy for individual growth, development, reproduction, somatic and maturity maintenance are modelled according to the DEB theory \citep{koojmanDynamicEnergyBudget2010}. The model considers important ecological processes such as opportunistic size-structured trophic interactions and competition for food, predatory, disease, ageing and starvation mortality, key physiological aspects such as vision and respiration, as well as essential behaviours such as 3D passive transport by marine currents and active habitat-based movements (\cite{faugerasAdvectiondiffusionreactionSizestructuredFish2005}), schooling and swarming (see \citealp{mauryModelingEnvironmentalEffects2007, mauryIndividualsPopulationsCommunities2013, mauryCanSchoolingRegulate2017} for a detailed description of the model). APECOSM is driven by 3D outputs from the physics-biogeochemistry coupled model NEMO-PISCES (3D fields of temperature and horizontal currents, vertical mixing, small and large phytoplankton, small and large zooplankton, detritus, light and oxygen) that constrain the biological and ecological dynamics at various levels.
The bio-energetic bases of the model are based on the Dynamic Energy Budget theory (DEB, \citealp{koojmanDynamicEnergyBudget2010}). All the metabolic rates are temperature-dependent and corrected by an Arrhenius factor (Maury, 2007; Maury and Poggiale, 2013). In the model configuration used here, we do not prescribe a limited temperature range any of the simulated communities.

The dynamics of communities is determined by integrating the core state equation below:

\begin{equation}
\partial_t \varepsilon = \underbrace{- \partial_w(\gamma \varepsilon) + \frac{\gamma}{w}\varepsilon}_{Growth} 
\underbrace{- M \varepsilon \vphantom{\frac{\gamma}{w}\varepsilon}}_{Mortalities}
\underbrace{-\overrightarrow{\nabla}.(\overrightarrow{V} \varepsilon) \vphantom{\frac{\gamma}{w}\varepsilon}}_{3D Adv} 
\underbrace{+ \overrightarrow{\nabla} . (D \overrightarrow{\nabla} \varepsilon) \vphantom{\frac{\gamma}{w}\varepsilon}}_{3D Diff.}
\label{eq:apecosm_trend}
\end{equation}

where $\varepsilon$  is the fish biomass density, $w$ the weight, $\gamma$ is the growth rate, $M$ represent the different mortality rates (computed using equation 12 of \citealt{mauryIndividualsPopulationsCommunities2013}), $V$ and $D$ the sum of passive and active velocities and diffusivity coefficients (computed following \citealt{faugerasAdvectiondiffusionreactionSizestructuredFish2005}). Reproduction is considered through a Dirichlet boundary condition in w0 that accounts for by the reproductive output from all mature organisms.

In APECOSM, the energy ingested by organisms fuels individual metabolism according to the DEB theory. Ingestion is proportional to a functional Holing type II response function that depends on the visibility of prey, their aggregation in schools and temperature. This functional response can be written in a simplified way as follows:


The APECOSM simulation used in this study is forced by temperature, horizontal current velocity, dissolved oxygen concentration, low-trophic level concentration (diatoms, mesozooplankton and microzooplankton, big particulate organic matter), photosynthetically active radiation (PAR) and dynamical layer thickness outputed from the NEMO-PISCES simulation (section \ref{sec:nemo}). It uses a daily time-step for biological processes, which is decomposed into a day/night cycle, whose duration depends on the latitude and on the day of the year \citep{forsytheModelComparisonDaylength1995}. A sub time-stepping ($dt =0.8h$) is used for horizontal advection and diffusion in order to insure numerical stability.

%In order to assess the mechanisms of fish biomass response to ENSO variability, biomass changes induced by growth, predation, advection and diffusion are stored for the entire simulation. Therefore, the biomass for a given size class, at a given location and for a specific time can be reconstructed by using the following equation:
%
%\begin{equation}
%\varepsilon(T) = \varepsilon(T=0) + \int_{t=0}^{T} \left[ 
%T_{pred}(t) +
%T_{growth}(t) + 
%T_{adv}(t) + 
%T_{diff}(t) 
%\right] dt 
%\label{eq:rec_oope}
%\end{equation}
%
%with $\varepsilon(x,y,s,T=0)$ the fish biomass at the beginning of the simulation and $T_{pred}$, $T_{growth}$, $T_{adv}$ and $T_{diff}$ the respective biomass increments due to predation, growth, advection and diffusion, given by equations \ref{eq:pred}, \ref{eq:growth} and \ref{eq:move}, respectively.

Three interactive communities are simulated in the present study:
\begin{itemize}
\item{The epipelagic community, which includes organisms feeding during the day near the surface such as yellowfin or skipjack tunas for instance. Its vertical distribution is influenced by light, visible food, temperature and oxygen while its functional response is influenced by light.}
\item{The migratory mesopelagic community, which feeds in the surface layer at night and migrate to deeper waters during daytime. Its vertical distribution is influenced by light and visible food during the night.}
\item{The resident mesopelagic community, which remains at depth during both night and day. Its vertical distribution is influenced by light and visible food during the day.}
\end{itemize}

For each community, equation (1) is integrated over 100 size classes, ranging from $0.123cm$ to $196cm$. In order to ensure that the size-spectrum is fully unfolded and a pseudo stationary regime is reached, the model has been integrated three times. First, it has been initialized with an arbitrary small biomass value in every size-class and integrated from 1958 to 2018 (61 years). Then, the end of this first spin-up phase has been used to run another cycle, which was in turn used to initialize the simulation presented in the following.

%Epilagic feed on other epilagic fish only during night daytime. Migrants feed on other migrants and epipelagics, only during night-time. While mesopelagics feed on migrant and mesopelagic during daytime, and only on other mesopelagic during night-time.

In the following, the focus is put on the epipelagic community only , since its near-surface location makes it more sensitive to ENSO variability \citep{lemezoNaturalVariabilityMarine2016} and since it corresponds to organisms such as skipjack and yellowfin that are targeted by industrial purse seine fleet that represent the bulk of tuna catches in the region and that have been reported to respond to ENSO (Lehodey, 1997).


%\section{Data and method}
%
%In the present section, the different observation-based datasets used in this study to validate the numerical models are presented. 
%
%%and statistical tools used in this study are presented.
%
%
%
%\subsection{Sea-level anomalies}
%\label{sec:ssh}
%
%Observation-based sea-level anomalies are obtained from the reprocessed Global Ocean Gridded L4 Sea-Surface Heights\footnote{\url{https://doi.org/10.48670/moi-00148}} data, provided by the Copernicus Climate Service as monthly values on a regular $0.25 \times 0.25$ grid from 1993-01 to 2020-12. This product was built by the DUACS multimission altimeter data processing system, which processes data from all altimeter missions: Jason-3, Sentinel-3A, HY-2A, Saral/AltiKa, Cryosat-2, Jason-2, Jason-1, T/P, ENVISAT, GFO, ERS1/2. Using along-track altimeter data, gridded sea-level anomalies are extracted by using an Optimal Interpolation off all the flying satellites. 
%
%%For our study, a monthly sea-level climatology was computed from 1993 to 2018 and used to extract monthly sea-level anomalies, which have been detrended.
%
%\subsection{Chlorophyll}
%\label{sec:chl}
%
%Chlorophyll observation-based estimates are extracted from the monthly OceanColour-CCI V5 CHL-a dataset\footnote{\url{http://dx.doi.org/10.5285/1dbe7a109c0244aaad713e078fd3059a}} \citep{sathyendranathOceanColourTimeSeries2019} over the 1997-09/2018-12 period. For our purpose, the high resolution dataset (4 km) has been regridded on a regular $1\times 1$ grid by computing weighted chlorophyll averages over $24\times24$ boxes, with the weights being provided by the cosine of latitude. When more than $1/3$ of the data used in the averaging is missing, the regridded cell is masked. 
%
%%The monthly climatology has then been computed over the 1998-2008 period and used to extract monthly anomalies, which have then been detrended.
%
%\subsection{Fish biomass estimates}
%\label{sec:fish}
%
%Fish biomass estimates are extracted from the IRD Level 2 global monthly catch of tuna, tuna-like and shark species dataset\footnote{\url{https://doi.org/10.5281/zenodo.1164128}} \citep{taconetGlobalMonthlyCatch2018}. First, the purse-seine captures of skipjack and yellowfin tunas have been extracted from the raw input file. Then, non-monthly observations (i.e. when the difference between the end and start dates exceed 31 days) have been discarded, as well as those which are unlocalized. Finally, the remaining observations have been regridded into on a regular $1 \times 1$ grid using the overlapping area between the observation polygon and the destination cell. The final product is a $3D$ array of dimensions (time and space) that extends from 1959 to 2016. However, due to the early poor data coverage, this dataset was used from 1985 onward.
%
%%\subsection{Covariance analysis}
%%\label{sec:cov}
%%
%%In order to assess the spatial signature of ENSO variability on physical, biogeochemical and biological variables, covariance analysis is used. For each grid cell, the monthly climatology of the analysed fields is removed, and the resulting monthly anomalies are detrended. Finally, the covariance between the detrended anomalies and the ONI index is computed. The resulting map can be interpreted as the monthly anomalies associated with an ONI value of 1.
%%%Correlations are computed in the same way. 
%%%Their significance has been inferred using a Student t-test, in which the effective number of degrees of freedom has been corrected based on the 1-lag autocorrelation of the 2 two time-series \citep{brethertonEffectiveNumberSpatial1999}.
\section{Model evaluation}

Before analysing the mechanisms driving marine ecosystem response to ENSO variability, the physical, biogeochemical and biological models are evaluated against observation-based estimates.

\subsection{Physical and biogeochemical model}

Although successfully used in a variety of ENSO-related physical and biogeochemical studies in the tropical Pacific (e.g., \citealt{vialardModelStudyOceanic2001, lengaigneOceanResponseMarch2002, lengaigneInfluenceOceanicBiology2007, schneiderClimateinducedInterannualVariability2008, masottiLargescaleShiftsPhytoplankton2011, currieIndianOceanDipole2013}), the ability of our simulation to capture ENSO surface temperature, sea-level anomalies and chlorophyll signature is briefly evaluated. 

This ability is assessed by comparing observation-based and simulated covariance maps of detrended monthly anomalies with the Oceanic Nino Index\footnote{\url{https://www.cpc.ncep.noaa.gov/data/indices/oni.ascii.txt}} (hereafter ONI), which is the NOAA official ENSO indicator. It is based on a 3-month running mean of sea-surface tegeosmperature anomalies averaged over the Niño 3.4 region (5N-5S, 170W-120W).

%Events are classified as full-fledged El Niño or La Niña events if the anomalies exceed +0.5C or -0.5C for at least five consecutive months. 

\subsubsection{Sea-surface temperature}
\label{sec:sst}

As a first evaluation of the simulated sea-surfate temperature, the simulated ONI index is compared with the one provided by the NOAA (Fig\ref{fig:nemo-had-sst}a). The two time-series show a strong correlation of 0.92 over their common 1958-2018 period. In particular, the model is able to accurately capture the timing and amplitude of major El Niño events, like in 1972/73, 1982/83, 1997/98 and 2015/16 and of major La Niña events, like in 1988/89 and 1999/2000. 

As a second evaluation, the spatial structure of simulated sea-surface temperature response to ENSO is now compared with the one obtained with the Hadley Sea-Surface Temperature dataset\footnote{\url{https://www.metoffice.gov.uk/hadobs/hadisst/data/download.html}} (HadISST1, \citealt{raynerGlobalAnalysesSea2003}). HadISST1 uses reduced space optimal interpolation applied to SSTs from the Marine Data Bank (mainly ship tracks) and ICOADS through 1981 and a blend of in-situ and adjusted satellite-derived SSTs for 1982-onwards. The resulting dataset is provided as monthly sea-surface temperature maps on a $1\times 1$ regular grid. The covariances obtained with the HadISST1 anomalies and the simulated ones are shown in Fig.\ref{fig:nemo-had-sst}b and Fig.\ref{fig:nemo-had-sst}c.

\begin{figure}
	\centering
	\includegraphics[scale=0.6]{figs/fig1.png}
	\caption{Observed and simulated ONI indexes (a). Covariance between the ONI index and the Hadley (b) and simulated SST anomalies (c). Observed and simulated sea-level anomalies averaged over the Nino34 region (d). Covariances between the ONI index and the observed (e) and simulated (f) sea-level anomalies.}
	\label{fig:nemo-had-sst}
\end{figure}

Simulated ENSO SST  pattern closely resembles the observed one. This pattern is characterized by warm SST anomalies (1°C) centred in the central and eastern equatorial Pacific  flanked by the traditional horseshoe cooling pattern in the western Pacific extending towards the subtropical north and south Pacific. This SST seesaw in the equatorial region is further accompanied by a shoaling of the thermocline in the west and a deepening in the east, which can be inferred from sea-level anomalies.

\subsubsection{Sea-level anomalies}

Simulated sea-level anomalies are compared with the reprocessed Global Ocean Gridded L4 Sea-Surface Heights\footnote{\url{https://doi.org/10.48670/moi-00148}} data, provided by the Copernicus Climate Service as monthly values on a regular $0.25 \times 0.25$ grid from 1993-01 to 2020-12. This product was built by the DUACS multimission altimeter data processing system, which processes data from all altimeter missions: Jason-3, Sentinel-3A, HY-2A, Saral/AltiKa, Cryosat-2, Jason-2, Jason-1, T/P, ENVISAT, GFO, ERS1/2. Using along-track altimeter data, gridded sea-level anomalies are extracted by using an Optimal Interpolation off all the flying satellites. 

The simulated sea-level anomalies averaged over the Nino34 region compare well with the observed ones (Fig \ref{fig:nemo-had-sst}d), with a correlation coefficient of 0.94. In addition, the spatial patterns associated with ENSO variability are very similar for both the observations and the model (Fig \ref{fig:nemo-had-sst}e, Fig \ref{fig:nemo-had-sst}f). Sea-level anomalies are negative in the western Pacific and positive in the central and eastern Pacific. These anomalies are consistent with thermocline vertical displacements, which shoal in the west and deepens in the east (\warn{REF}).

\subsubsection{Chlorophyll}

Simulated chlorophyll has been compared with observation-based estimates extracted from the monthly OceanColour-CCI V5 CHL-a dataset\footnote{\url{http://dx.doi.org/10.5285/1dbe7a109c0244aaad713e078fd3059a}} \citep{sathyendranathOceanColourTimeSeries2019} over the 1997-09/2018-12 period. The high resolution (4 km)  dataset has been regridded on a regular $1\times 1$ grid by computing weighted chlorophyll averages over $24\times24$ boxes, with the weights being provided by the cosine of latitude. When more than $1/3$ of the data used in the averaging is missing, the regridded cell is masked.

Covariance maps between the surface chlorophyll and the monthly ONI index are shown in Fig \ref{fig:nemo-sat-chl}. The observations (upper panel) and the model (lower panel) show very similar covariance patterns: El Niño induces a decrease in chlorophyll concentration along the equator east of 150°E, consistent with a weaker equatorial upwelling induced by the equatorial trade winds reduction. However, the model overestimates the chlorophyll response to ENSO variability compared with observational based estimates. Note that the covariance for the simulated chlorophyll has also been computed over the entire simulated period (1958-2018), with no significant changes in the resulting pattern (not shown).

\begin{figure}
	\centering
	\includegraphics[scale=0.4]{figs/fig2.png}
	\caption{Simulated (black) and observed (yellow) equatorial chlorophyll anomalies (a). Filtered anomalies are shown for the model in panel (b). Covariance between the observed and simulated chlorophyll anomalies with the ONI (c, d) and filtered TPI (e) index.}
	\label{fig:nemo-sat-chl}
\end{figure}

\subsection{Marine ecosystem model}

Before analysing the mechanisms of fish biomass response to El Nino variability, simulated biomass integrated from 30cm to 70cm is compared with the observation-based estimates extracted from the IRD Level 2 global monthly catch of tuna, tuna-like and shark species dataset\footnote{\url{https://doi.org/10.5281/zenodo.1164128}} \citep{taconetGlobalMonthlyCatch2018}. First, the purse-seine captures of skipjack and yellowfin tunas have been extracted from the raw input file. Then, non-monthly observations (i.e. when the difference between the end and start dates exceed 31 days) have been discarded, as well as those which are unlocalized. Finally, the remaining observations have been regridded into on a regular $1 \times 1$ grid using the overlapping area between the observation polygon and the destination cell. The final product is a $3D$ array of dimensions (time and space) that extends from 1959 to 2016. However, due to the early poor data coverage, this dataset was used from 1985 onward.

Figure \ref{fig:apecosm_validation}a shows the Hovmoller diagram of the observed catches and fish biomass (log-scale) integrated between 10N and 10S for the 2008-2018 period. Although observed and model results cannot be directly compared, some features are well captured by the model. For instance, the easternmost $4.00$ simulated contour seems to closely follow the eastward extensions of catches that occur in 2010, 2012 and 2016. Figure \ref{fig:apecosm_validation}b shows, for a longer period (1985-2018), the longitudinal position of the catch and biomass barycenters. The observed barycenter shows an eastward trend, liley due to the increased power of fishing fleets, which allow them to navigate over greater distances from their fishing ports, mostly located in the western Pacific (\warn{ref}). This emphasizes the fact that the data used in this validation are catch data, which are not only dependent on fish biomass but also depend on fishing effort, and as such must be used with caution for the comparison with simulated fish biomass. Figures \ref{fig:apecosm_validation}c and \ref{fig:apecosm_validation}d show the difference between catch and simulated biomass composites for El Nino (OND2009-JFM2010, OND2014-JFM2015, OND-2015-JFM2016) and La Nina conditions (OND2007-JFM2008, OND2008-JFM2009, OND2010-JFM2011, OND2011-JFM2012). Observed catches increase in the central and eastern Pacific and decrease in the Western Pacific when switching from La Nina to El Nino conditions, which is consistent with an eastward displacement of the fish biomass. Simulated composite also shows an increase in the central Pacific and a decrease in the west, although the pattern seems westward shifted compared with the catch composite. 

\begin{figure}
	\centering
	\includegraphics[scale=0.5]{figs/plot_validation_apecosm.png}
	\caption{Hovmoller diagram of observed catches (colors) and simulated fish biomass (contours) integrated between 10N and 10S (a, log-scale) and associated barycenters (b). Catch (c) and simulated fish-biomass (d) difference between El Nino and La Nina composites.}
	\label{fig:apecosm_validation}
\end{figure}

In the following, the results for three size classes, 3cm, representing small fishes, 20cm, representing intermediate sizes, and 90 cm, representing large individuals, will be detailed. The latter are representative of the sizes of tuna target species within the region (\warn{REF}). \\


% !TeX root = ../article-enso.tex

\section{Response of the epipelagic community to extreme \nino{} events}
\label{sec:nino-epi}

This section focuses on describing the simulated size-dependent response of the epipelagic community to ENSO and understanding the mechanisms responsible for this response. Here, we will specifically study the epipelagic community response to the three strongest \nino{} events over the historical period, namely those of 1982/83, 1997/98 and 2015/16 \citep{santosoDefiningCharacteristicsENSO2017}. During these events, the central and eastern Pacific has warmed by more than 2°C (Figure \ref{fig:nemo-had-sst}a), moving the warm waters and associated atmospheric convection from the west to the central and eastern Pacific. The atmospheric signature of these \nino{} has had dramatic climatic consequences, including droughts and forest fires in countries bordering the western Pacific, but also torrential rains and floods along the south American coast \citep{caiClimateImpactsNino2020}. Their oceanic signature also had major impacts on marine ecosystems and biodiversity, leading to significant disruptions in marine life and seabird populations \citep{valleImpact198219831987}, promoting large-scale marine heatwaves \citep{holbrookKeepingPaceMarine2020} and coral bleaching \citep{claarGlobalPatternsImpacts2018}.  The ocean response during each of these three extreme events has been extensively described and analyzed in terms of physics (e.g. \citealp{philanderChapter33Simulation1985, lengaigneOceanResponseMarch2002, puyModulationEquatorialPacific2019}), biogeochemistry (e.g. \citealp{barberBiologicalConsequencesNino1983, chavezBiologicalChemicalResponse1999, strammaObservedNinoConditions2016}) and marine ecosystems \citep{glynnNINOSOUTHERNOSCILLATION198219831988, glynnCoralBleachingMortality2001, eakin20142017Globalscale2019}. 

To isolate the generic response of epipelagic organisms to extreme \nino{} events, independent of the intrinsic characteristics of each event, we perform a composite analysis of these three extreme events, averaging monthly anomalies of temperature, ocean velocity, low trophic level  (LTL, i.e. phyto and zooplankton, particulate organanic matter) concentrations and biomass of the epipelagic community over the 1982-1983, 1997-1998 and 2015-2016 periods. These extreme \nino{} events are also followed by \nina{} conditions the following year (more intense in the case of the 1997/98 event), which also allows for a discussion  of the epipelagic community response mechanisms to \nina{} events. Although the temporal evolution and amplitude of the processes discussed below vary slightly between events, the relative importance of the processes discussed in our composite analysis remains qualitatively similar when these three extreme events are analyzed individually (not shown). 

\subsection{Model response from physics to ecosystems}

As major environmental drivers of the epipelagic biomass variability, Figure \ref{fig:hov_nemo_ape}a-c first depicts the temporal evolution of monthly equatorial anomalies in upper ocean temperatures, LTL concentration and zonal currents during and after extreme \nino{} events, in the form of equatorial time-longitude diagrams, with the January month preceding the onset of the \nino{} event as origin of time. The warming signal associated with \nino{} initiates in the central equatorial Pacific in early spring, then spreads rapidly to the eastern Pacific, intensifies during the summer and fall, peaks at the end of the calendar year, and finally  declines rapidly and transitions to \nina{} conditions the following spring (from April-y1, Figure \ref{fig:hov_nemo_ape}a). The development phase of \nino{} is also characterized by strong eastward surface currents anomalies in the western and central Pacific (Figure \ref{fig:hov_nemo_ape}b) induced by anomalous westerly winds, promoting warming of the central Pacific and eastward movement of the warm-pool toward the eastern equatorial Pacific. These current anomalies reverse at the peak of \nino{} and during \nina{}. The simulated plankton concentration anomalies largely mirror those of temperature, with a sharp decrease during \nino{} and an increase during \nina{} (Figure \ref{fig:hov_nemo_ape}.c). 

\begin{figure}[h!tp]
	\centering
	\includegraphics[scale=0.4]{plot_all_hovmoller_phys_oope.png}	
	\caption{Time-longitude diagrams in the equatorial Pacific of surface temperatures (in °C) (a), zonal velocity (in m/s) (b), low-trophic level concentrations (in mmol/m3) (c) and fish biomass anomalies (in J/m2) associated with extreme \nino{} events composite (3cm, 20cm and 90 cm in d, e, f, respectively). The eastern location of the warm pool (28\degree{} isotherm) is shown in red in (a).}	
	\label{fig:hov_nemo_ape}
\end{figure}

A similar analysis is then performed for epipelagic biomass for the three selected size classes (Figure \ref{fig:hov_nemo_ape}def). Their responses to \nino{} share common characteristics: positive biomass anomalies appear near the dateline early in the calendar year and propagate eastward toward the central Pacific until late spring (May/June-y0). These positive biomass anomalies in the central Pacific re-intensify in fall and then rapidly disappear in winter. They are also accompanied by a decrease in biomass in the western Pacific from the beginning of the \nino{} year. These negative anomalies persist after the \nino{} peak and during the subsequent \nina{} event but remain largely confined to the western Pacific. Despite similar behaviour, however, the response of the three size classes show some significant differences, including a westward shift in response as size class increases.

Figure \ref{fig:profiles} shows how these surface ENSO-related signals propagate in depth by providing climatological equatorial profiles of temperatures, zonal velocities, low-trophic level concentrations and epipelagic daytime biomass for the three size classes, as well as their boreal winter anomalies for extreme \nino{} composites. The climatological temperature profile indicates that the thermocline is deep in the western Pacific and shallow in the east, and flattens during \nino{}, resulting in warming in the east and cooling in the west (Figure  \ref{fig:profiles}a). The Equatorial Undercurrent also weakens strongly during \nino{}, while strong positive (i.e. eastward) current anomalies occur near the surface (Figure  \ref{fig:profiles}b). Low trophic level biomass, which is maximal in the upper 50m of the eastern Pacific, decrease during \nino{}, due to the flattening of the thermocline, which reduces the nutrient supply in the surface layers (Figure \ref{fig:profiles}c). Regarding the vertical extent of the epipelagic community, the climatological biomass for the three classes in the western Pacific extends from the surface to $100m$ in the western Pacific and decreases during \nino{} events. However, this decrease is not homogeneous along the vertical, with strong positive anomalies appearing around 40m in the west for intermediate and large sizes (Fig. \ref{fig:profiles}def). These are induced by a narrowing of the vertical habitat, due to the shallowing of the thermocline. 
%The climatological biomass decrease rapidly towards the central Pacific to become negligible in the eastern part of the basin. This structure is significantly altered during extreme \nino{} events, where fish biomass increases from the surface to $100m$ depth in the central and eastern Pacific, with a maximum located near $40m$.

\begin{figure}[h!tp]
	\centering
	\includegraphics[scale=0.4]{figs/forage_mean_ond97.png}	
	\caption{Pacific equatorial profiles of temperature (a), zonal velocity (b), low-trophic level concentration (c) and fish biomass (d for small, e for intermediate and f for large sizes). Mean values are represented as black contour lines and \nino{} anomalies are represented in colors.}	
	\label{fig:profiles}
\end{figure}

\subsection{Processes driving the epipelagic upper-ocean response}

The contribution of the different processes responsible for the epipelagic response to \nino{} (Figure \ref{fig:hov_nemo_ape}def) is now assessed by performing the same equatorial time-longitude diagrams for the main tendency terms (right members of equation \ref{eq:apecosm_trend}) and their temporal integral, which represents their contribution to the total biomass change. We also analyze key parameters of the biological response to changing environmental conditions, namely growth rate ($\gamma$ in equation \ref{eq:apecosm_trend}), functional response (equation \ref{eq:repfonct}) and predation mortality rates ($M$ in equation \ref{eq:apecosm_trend}). Because the relative importance of these processes varies among size classes, these analyses are discussed separately for each size class.

Figure \ref{fig:fig7} provides a synthesis of the respective contributions of biological (i.e. the combined action of growth and predation) and physical (i.e. the combined action of advection and diffusion) processes on the epipelagic biomass response to ENSO for each of the three size classes. A first result is that the relative importance of biological processes decreases as fish size increases for two reasons. First, predation is size-based in the APECOSM model, resulting in high predation pressure on small organisms, which decreases with size since larger organisms have fewer predators in the model. Second, growth includes a flux term and a source term (see equation \ref{eq:apecosm_trend}) that are both dependent on temperature. The source term controls biomass production and varies as ${\gamma}/{w}$, which scales linearly with $w^{-\frac{1}{3}}$ and thus decreases strongly with size.

The decrease of the small and intermediate size biomass in the western Pacific is thus primarily the result of biological processes. In the central and eastern Pacific, the combined action of dynamical and biological processes accounts for the increase in biomass during \nino{} (Figure \ref{fig:fig7}a-f) while these processes largely offset each other when the equatorial Pacific reverses to \nina{} conditions, resulting in small  changes in biomass in this region. For the largest size class, physical processes (Figure \ref{fig:fig7}i) explain most of the biomass changes (Figure \ref{fig:fig7}g), with biological processes being negligible (Figure \ref{fig:fig7}h).

The action of dynamical processes on biomass evolution is simple. It results from the transport of biomass from the western to the central Pacific in response to the strong eastward currents anomalies that occur during extreme \nino{} conditions. It is experienced in a similar way by all size organisms, since the ocean current anomalies during \nino{} (up to $0.6 m.s^{-1}$, see Figure \ref{fig:hov_nemo_ape}c) dominate the active velocity anomalies by a factor of $\approx\ 1000$ for small sizes, $20$ for intermediate sizes and $3$ for large sizes (not shown). 

\begin{figure}[h!tp]
	\centering
	\includegraphics[scale=0.4]{figs/fig7.png}	
	\caption{Time-longitude diagrams in the equatorial Pacific of total (left), biologically (middle, predation plus growth terms) and dynamically (right, advection plus diffusion) induced interannual variations in fish biomass (in $J.m^{-2}$) associated with extreme \nino{} events composites for small (top), intermediate (middle) and large (bottom) sizes.}
	\label{fig:fig7}
\end{figure}

The significant and sometimes dominant contribution of biological processes for small and medium size classes, however, is more difficult to understand intuitively because it results from the combined action of predation mortality and growth. Therefore, we further detail the respective contribution of predation and growth and their driving factors for small and medium size classes on Figure \ref{fig:fig8} and Figure \ref{fig:fig9} respectively.

For small size classes (3cm), the effects of predation mortality balance the effects of growth (Figure \ref{fig:fig8}bc), resulting in a net effect of biological processes that is much smaller than the effect of each biological process considered in isolation (Figure \ref{fig:fig8}a). Growth leads to an increase in biomass at the onset of \nino{} in the central Pacific (between dateline and 150\degree{}W), that spreads eastward to its peak. These positive biomass anomalies then decrease slightly during the following \nina{} conditions (Figure \ref{fig:fig8}b).

Figure \ref{fig:fig8}e shows the functional response, which controls the predation swimming speed (that is proportional to the functional response's gradient), the vertical distribution and swarming level of epipelagic fish, which in turn control their availability to predators.  Despite its importance in controlling growth and reproduction, our analysis indicates that the decrease in functional response is not the primary driver of biomass changes, since negative functional response anomalies are associated with positive biomass anomalies.

Instead, the increase in growth rate east of the dateline during \nino{} closely follows the evolution of the anomalous warming (Figure \ref{fig:fig8}e), suggesting that changes in growth rate are largely driven by the influence of temperature on fish physiology. In contrast to the central and eastern Pacific, the growth rate decreases in the western Pacific as it cools from July of the \nino{} year, contributing to a decrease in biomass. Although these growth rate negative anomalies in the western Pacific are smaller than the positive ones in the eastern part, their impact on the biomass, which is proportional to the biomass itself (cf. equation \ref{eq:apecosm_trend}) is greater since biomass levels are ten times larger in the west than in the east (see black contours in Figure \ref{fig:mean_ond97_ape}).

As mentioned previously, predation-induced changes in biomass are largely opposite to growth-induced changes  (Figure \ref{fig:fig8}bc). Predation-induced changes decrease biomass in the central and eastern Pacific and increase biomass in the far western Pacific (Figure \ref{fig:fig8}c), closely following the changes in biomass of intermediate size predators (Figure \ref{fig:fig8}f). Despite their opposite effect on biomass, growth effects generally slightly dominate those of predation, explaining most of the decrease in small size classes in the western Pacific during \nino{} and the subsequent \nina{}, and reinforcing the biomass increase in the central Pacific induced by dynamical processes during \nino{}. An exception is the very early decrease in biomass simulated near the dateline from February of the \nino{} year, which is not driven by growth rate changes (Figure \ref{fig:fig8}b) but rather by increased predation by intermediate size organisms at the \nino{} onset (Figure 8c,f).

\begin{figure}[h!tp]
	\centering
	\includegraphics[scale=0.4]{figs/fig8.png}	
	\caption{Time-longitude diagrams in the equatorial Pacific of interannual anomalies of small sizes biomass trends ($J.m^{-2}.s^{-1}$; in colors) and time-integrated trends ($J.m^{-2}$; in contours) associated with extreme \nino{} events composite for predation plus growth (a), growth (b) and predation (c). Same as (a-c) but for interannual anomalies of the functional response (no unit; in colors) and planktonic prey biomass density ($mmol.m^{-3}$; in contours) (d), growth rate ($kg.day^{-1}$; in colors) and temperature (°C; in contours) (e) and predation mortality rate ($day^{-1}$; in colors) and intermediate size biomass ($J.m^{-2}$; in contour) (f).}
	\label{fig:fig8}
\end{figure}

% ---------------- to complete
The growth and predation induced biomass changes for the intermediate size classes are similar to those simulated for small size classes (Figure \ref{fig:fig9}): they are opposite and of the same order of magnitude, with growth effects generally dominating predation effects. Growth increases fish biomass in the central Pacific from the onset to the peak of \nino{}. However, the influence of temperature on fish physiology is no longer the dominant factor of biologically induced biomass changes for intermediate size organisms as it was for small organisms. In contrast to small size classes, changes in growth rate largely reflect changes in functional response, which is increasing in the central Pacific due to both warmer waters (increased swimming speed controlling the attack rate parameter in the functional response) and increased food availability (due to the increased biomass of small organisms), both of which contribute to increase the biomass of intermediate size organisms in the central Pacific. In the western Pacific, the growth rate decreases only very modestly (Figure \ref{fig:fig9}b) but, as seen for small size classes, this translates into a large reduction of growth-induced biomass from October onwards since its effect is proportional to biomass, which is ten times larger in the western than in the eastern Pacific (Figure \ref{fig:mean_ond97_ape}).
%Thus, the source component of growth is not altered but the advective flux along the weight axis (see equation \ref{eq:apecosm_trend}) decreases during \nina{} (Fig\ref{fig:fig7}a) because of the biomass decrease induced both by passive eastward advection by zonal currents anomalies during \nino{} and increased mortality rates (Fig\ref{fig:fig7}d). 
On the other hand, predation generally mitigates the effects of growth, reducing biomass in the central Pacific through increased predation by large size classes there and increasing biomass in the western Pacific during the subsequent \nina{} through reduced predation. This evolution largely resembles those obtained for small sizes, albeit with a modest westward shift. The changes induced by the combination of these two processes are generally dominated by growth, except in the western Pacific during \nino{} development where the  decrease in biomass from February onwards is due to increased predation by large organisms. As for small sizes, the decrease in biomass in the western Pacific during \nino{} and the subsequent \nina{} are initially due to increased predation followed by a reduction in growth, while dynamical processes dominate the biomass increase east of the dateline with a smaller contribution from growth.

%The strong increase in predation on small organisms in the west (180E) leads to a decrease in biomass. This decline intensifies and spreads eastwards in response to the decrease in growth rate associated with the cooling of the western Pacific waters. This decrease in biomass in turn explains the anomalies in the growth term, which is dominated by the biomass anomalies (source term in equation \ref{eq:apecosm_trend}).

\begin{figure}[h!tp]
	\centering
	\includegraphics[scale=0.4]{figs/fig9.png}	
	\caption{Time-longitude diagrams in the equatorial Pacific of interannual anomalies of intermediate sizes biomass trends ($J.m^{-2}.s^{-1}$; in colors) and time-integrated trends ($J.m^{-2}$; in contours) associated with extreme \nino{} events composite for predation plus growth (a), growth (b) and predation (c). Same as (a-c) but for interannual anomalies of the functional response (no unit; in colors) and small prey biomass density ($J.m^{-2}$; in contours) (d), growth rate ($kg.day^{-1}$; in colors) and temperature (°C; in contours) (e) and predation mortality rate ($day^{-1}$; in colors) and large size biomass ($J.m^{-2}$; in contour) (f).}
	\label{fig:fig9}
\end{figure}

Figure \ref{fig:proc_summary} provides a brief summary of the different processes involved in the epipelagic response to interannual ENSO variability that has been discussed in this section.

\begin{figure}[h!tp]
	\centering
        \includegraphics[scale=0.6]{figs/conclusion/conlusion_fig.pdf}	
        \caption{Summary of the processes involved in the response of epipelagic biomass to \nino{} conditions.  $Pred.$ is predation, $T$ is temperature, $A+D$ is advection/diffusion, $f$ is the functional response and $F$ is food concentration.}
	\label{fig:proc_summary}
\end{figure}



\subsection{Generalization}

All the analyses presented above focused on the equatorial Pacific, where ENSO physical and biogeochemical signatures are the strongest. To ascertain the response of off equatorial regions to ENSO,  Figure \ref{fig:mean_ond97_ape} further provides maps of climatological epipelagic biomass for the three size classes as well as their boreal winter anomalies for extreme \nino{} composites.  In average, epipelagic fish biomass is largest both sides of the equator and in the equatorial western Pacific (Figure \ref{fig:mean_ond97_ape}abc), while smaller biomasses  are found in the eastern Pacific. In agreement with the equatorial analyses provided on Figure \ref{fig:hov_nemo_ape}, Figure \ref{fig:mean_ond97_ape}a indicates that during \nino{}, small epipelagic fish biomass increases in the equatorial eastern Pacific and decreases in the western Pacific. In addition, Figure \ref{fig:mean_ond97_ape}a reveals that this biomass also decreases both sides of the equator, further highlighting that the biomass does not only shift eastward during \nino{} but also equatorward.
As size increases, positive anomalies associated with \nino{} conditions expand westward and poleward while negative anomalies weaken and expand equatorward. 

\begin{figure}[h!tp]
	\centering
	\includegraphics[scale=0.4]{figs/map_mean_anom_OND_97.png}	
	\caption{Maps of boreal winter (DJF) biomass anomalies for extreme \nino{} events composites (left column) and covariance of fish biomass anomalies with the ONI index (right column) for small (upper line), intermediate (middle line) and large sizes (lower line). Black contours correspond to the climatological biomass density distribution (log-scale).}	
	\label{fig:mean_ond97_ape}
\end{figure}


To insure that the biomass response described for extreme \nino{} events is representative of ENSO variability in general, Figure \ref{fig:mean_ond97_ape}def shows the covariance maps computed between the monthly ONI index and the detrended fish biomass anomalies for the three size classes over the 1958-2018 period. This analysis reveals that the patterns in extreme \nino{} composites are very similar to covariance analyses, although amplitudes are about four times larger for extreme events. This  difference  in amplitude is related to the fact that the covariance analysis also includes weaker \nino{} events (such as the 1986, 1991, 1994, 2002, 2004 and 2009 events) as well as \nina{} events, which are known to have weaker physical and biogeochemical signatures. Nevertheless, the very good match between the covariance maps and the extreme \nino{} composites indicates that the biomass response and related mechanisms discussed above for the three major \nino{} events are also representative of other ENSO events.

\section{Response to decadal ENSO variability}

In the previous section, we have analysed the mechanisms of fish biomass response to interannual ENSO variability, 

\section*{Acknowledgement}

The authors acknowledge the Pôle de Calcul et de Données Marines (PCDM, \url{http://www.ifremer.fr/pcdm}) for providing DATARMOR storage, data access, computational resources, visualization and support services.\\

Data analysis and representation has been performed using Python 3.9.4. 
Covariance analysis and detrending has been performed using the \emph{numpy} and \emph{scipy} packages. Plots and maps have been made by using the \emph{matplotlib} and \emph{cartopy} packages.
The EOF analysis has been performed using version 1.4.0 of the \emph{eofs} package (\url{https://ajdawson.github.io/eofs/latest/}). 
Lanczos filtering has been performed using version 1.0.1 of the envtoolkit package (\url{https://doi.org/10.5281/zenodo.4095464}).

\listoffigures
\listoftables

\clearpage

\bibliography{ApecosmBib2}

\end{document}
