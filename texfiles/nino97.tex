\section{Response to interannual ENSO variability}

This section describes the synchronous response of the epipelagic community simulated by the model to ENSO variability in the tropical Pacific. We first discuss the response to the iconic 1997/1998 El Niño and then generalise these results to the overall ENSO variability. 

\subsection{The iconic 1997/1998 El Niño event}

We first focus on the 1997/98 El Niño event because (1) it is the strongest El Niño event on records (Santoso et al. 2017) and (2) the ocean response to this specific event has been extensively described and analyzed, in terms of physics  (e.g. \cite{mcphadenGenesisEvolution1997981999, vialardModelStudyOceanic2001, lengaigneOceanResponseMarch2002}, biogeochemistry \citep{chavezBiologicalChemicalResponse1999a, gierachBiologicalResponse19972012} and ecosystems (e.g. \cite{leaObservationsFishesAssociated2000, glynnCoralBleachingMortality2001,arcosJackMackerelFishery2001}).

Figure \ref{fig:mean_ond97_ape} displays the mean biomass density over the 1958-2018 period (left column) and the biomass density anomalies averaged over the 97 October, November and December (OND) months, when the El Nino signal is at its maximum.

\begin{figure}
	\centering
	\includegraphics[scale=0.5]{figs/fig3.png}	
	\caption{Mean fish biomass (left column, log-scale) and OND-97 anomalies for small (upper line), intermediate (middle line) and large sizes (lower line).}	
	\label{fig:mean_ond97_ape}
\end{figure}

For small sizes (Figure \ref{fig:mean_ond97_ape}a), the mean fish biomass is concentrated at around 10° S and 10°N in the central Pacific and close to the equator in the western Pacific. High biomass concentration is also found east of 90° W, off the coasts of Chile. As size increases, the equatorial "blue spot" extends meridionally and to the west. This pattern is mostly driven by the active and passive advection of fishes in the Apecosm model. Without advection, the biomass will be concentrated at the equator, where the plankton concentration is the maximum.

During the 97 El Nino, small epipelagics show negative anomalies in the Western Equatorial Pacific and positive anomalies in the Central Equatorial Pacific. This pattern can be interpreted as an eastern displacement of the mean biomass in the Western Pacific. Similar dipolar patterns are also obtained for intermediate and large sizes, but the anomalies shift westward as size increases. This can be interpreted, as for small sizes, by a westward shift of fish biomass during positive El Niño phases.

\warn{Discussion of the profiles???}

\begin{figure}
	\centering
	\includegraphics[scale=0.4]{figs/fig5.png}	
	\caption{Hovmoller diagrams of equatorial temperature (a), low-trophic level concentrations (b), zonal velocity (c) and fish biomass anomalies (3cm, 20cm and 90 cm in d, e, f, respectively).}	
	\label{fig:hov_nemo_ape}
\end{figure}

Figures \ref{fig:hov_nemo_ape}a, \ref{fig:hov_nemo_ape}b, \ref{fig:hov_nemo_ape}c display equatorial hovmüller diagrams of the temperature, chlorophyll and zonal currents anomalies averaged over the top 50m simulated by the ocean model during the 1997/1999 period. In agreement with observations (not shown; see \cite{lengaigneOceanResponseMarch2002}), the warming signal associated with the 1997 El Niño initiates in early spring over the central and eastern equatorial Pacific, peaks by the end of the calendar year in the eastern Pacific to quickly recede and switches La Niña conditions the following spring (Figure \ref{fig:hov_nemo_ape}a). The El Niño warming period is accompanied by strong surface eastward anomalous currents (Figure \ref{fig:hov_nemo_ape}c) promoted by anomalous westerly winds and contributing to central Pacific warming and eastward shift of the warm-pool towards the eastern equatorial Pacific. These current anomalies reverse during La Niña. Simulated plankton concentration anomalies tend to mirror temperature anomalies, with a strong decline during El Niño and an enhanced bloom during La Niña (Figure \ref{fig:hov_nemo_ape}.b). 

The same analysis is then performed for epipelagic biomass for the three targeted size classes (Fig\ref{fig:hov_nemo_ape}d, Fig\ref{fig:hov_nemo_ape}e and Fig\ref{fig:hov_nemo_ape}f), which share some common features: positive biomass anomalies for the three communities appear near the dateline in early spring and propagate eastward towards the central Pacific until the end of summer. Another positive patch then develops in the central Pacific in fall and quickly recedes in winter. These positive anomalies in the central and eastern Pacific  are associated with negative biomass anomalies in the western Pacific during the 1997/98 El Niño. These negative anomalies persist after the El Niño demise and during the following La Niña event but remain restricted to the western Pacific. While sharing similarities, the response of larger epipelagic fishes is generally shifted westward compared to smaller ones.

In order to determine the mechanisms behind the pattern of equatorial fish biomass anomalies during the 97 El Nino, equatorial fish biomass anomalies has been reconstructed for the different processes (growth, predation, advection and diffusion) separately based on equation \ref{eq:rec_oope}. The resulting Hovmoller diagrams are presented in Fig\ref{fig:hov_ape_trends}.

\begin{figure}
	\centering
	\includegraphics[scale=0.4]{figs/fig6.png}	
	\caption{Hovmoller diagrams of fish biomass anomalies induced by predation mortality (left column), growth (middle column) and movements (advection and diffusion, right column). Summing the figures on each line returns the fish biomass anomalies presented in Fig\ref{fig:hov_nemo_ape}.}	
	\label{fig:hov_ape_trends}
\end{figure}

Predation and growth show opposite contributions to changes in small fish biomass, which are around 3 times larger than the biomass anomalies (Fig. \ref{fig:hov_nemo_ape}d), implying that at first order these anomalies balance each other. In early 1997, growth induces an increase in the central Pacific (between dateline and 180W), while predation induces a decrease. Starting in 1997-10, these relative contributions to fish biomass anomalies extend to the east, and opposite variations appear in the western Pacific. Interestingly, the anomalies induced by growth are westward shifted and start later than the temperature increase induced by the onset of El Nino conditions. The pattern obtained by summing these two contributions is a negative anomaly centered around the dateline and extending from 1997-01 to 1998-12, with the strongest anomalies in early 1998 (not shown). Advection and diffusion account for an increase in fish biomass from 1997-01 to 1998-12 along a longitudinal band centered on the dateline, which compensates the negative anomalies induced by the combination of growth and predation, in addition to an increase between 150W and 120W that starts around 1997-10, which is also visible in the fish biomass anomalies (Fig. \ref{fig:hov_nemo_ape}d).

For large sizes, predation and growth processes are very similar. They are both associated with positive anomalies west of the dateline, and negative anomalies between the dateline and 150W. However, these anomalies do not seem to change much between 1997-01 and 1998-12 and therefore cannot explain the large fish biomass anomalies depicted in Fig.\ref{fig:hov_nemo_ape}f. On the other hand, the fish biomass anomalies induced by advection and diffusion perfectly match the simulated fish biomass anomalies. \warn(Specify if diffusion/advection, specify if active or passive). Therefore, on interannual time-scales, fish biomass for large sizes is not driven by biological processes but by physical ones.


%The patterns of the functional response and growth rates are very different for small sizes (figures YYY.a and YYY.d). The former shows negative anomalies on the central Pacific during the onset of El Nino conditions, presumably due to the concomitant reduction of plankton concentration (figure YYY.b). Interestingly, no anomalies are found in the eastern basin, due to compensating effects of warming temperatures and reduced plankton concentrations. The growth rate shows a pattern that is very similar to the temperature anomalies (Fig. XXX.a), indicating the dominance of temperature on the growth rate. The mortality rate (Fig. YYY.g) shows a pattern that is consistent with the increased biomass of intermediate epipelagics (XXX.b), which predate on small ones.  Besides, growth rate anomalies superimpose well on the small biomass anomalies (black contours in Fig. YYY.d), hence suggesting that the increased biomass during the El Nino conditions is triggered by enhanced growth associated with warmer temperature, despite a dampening effect of increased mortality rates.
%For intermediate sizes, the functional response and the growth rate show very similar anomalies, hence suggesting that they both are driven by the same factors. Both show positive anomalies during the onset of El Nino between 200E and 250E, presumably induced by the temperature warming that favours both the growth and search rates. Then negative anomalies occur, which are westward shifted relative to the positive ones. These negative anomalies are likely induced by the cooling associated with the La Nina conditions reinforced by the reduced small fish biomass that appears near 200E, as shown in Fig. XXX.d. Mortality rates anomalies are consistent with the increase of large fish biomass (Fig. XXX.f). Comparing the different fields, we can suggest that the increased fish biomass that appears on the central Pacific in early 1997 is first dominated by advective processes, which returns a similar pattern (fig YYY.k). Then, during the El Nino peak, the increased biomass of intermediate fish is driven by both an increased growth rate and an increased functional response, which are induced by warmer temperatures and more food available (more small epipelagic fishes, Fig. XXX.d).
