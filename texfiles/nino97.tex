\section{Response of the epipelagic community to extreme El Niño events}
\label{sec:nino-epi}

This section focuses on describing the simulated size-dependent response of the epipelagic community to ENSO and understanding the mechanisms responsible for this response. Here, we will more specifically investigate the marine ecosystem response to the three strongest El Niño events in the historical record, namely those of 1982/83, 1997/98 and 2015/16 \citep{santosoDefiningCharacteristicsENSO2017}. During these events, the central and eastern Pacific has warmed by more than 2°C (Fig\ref{fig:nemo-had-sst}a), moving the warm waters and associated atmospheric convection from the western to the central Pacific. The atmospheric signature of these El Niño has had dramatic climatic consequences, including droughts and forest fires in countries bordering the western Pacific, but also torrential rains and floods along the south American coast \citep{caiClimateImpactsNino2020}. Their oceanic signature also had a significant impact on marine ecosystems and biodiversity, leading to major disruptions in marine life and seabird populations \citep{valleImpact198219831987}, promoting large-scale marine heatwaves \citep{holbrookKeepingPaceMarine2020} and coral bleaching \citep{claarGlobalPatternsImpacts2018}.  The response of the ocean during each of these three extreme events has been extensively described and analyzed in terms of physics (e.g. \citealp{philanderChapter33Simulation1985, lengaigneOceanResponseMarch2002, puyModulationEquatorialPacific2019}), biogeochemistry (e.g. \citealp{barberBiologicalConsequencesNino1983, chavezBiologicalChemicalResponse1999, strammaObservedNinoConditions2016}) and marine ecosystems \citep{glynnNINOSOUTHERNOSCILLATION198219831988, glynnCoralBleachingMortality2001, eakin20142017Globalscale2019}. 

In order to isolate the generic response of epipelagic communities to extreme El Nino events, independent of the intrinsic characteristics of each event, we perform a composite analysis of these three extreme events, averaging monthly anomalies of temperature, ocean velocity, low-trophic level concentrations (LTL, i.e. phyto and zooplankton, particulate organanic matter) and epipelagic community biomass over the periods 1982-1983, 1997-1998 and 2015-2016. These extreme El Niño events are also followed by La Niña conditions in the following year (more intense in the case of the 1997/98 event), which also allows for a discussion  of the epipelagic community response mechanisms to La Niña events. Although the temporal evolution and amplitude of the processes discussed below vary slightly between events, the relative importance of the processes discussed in our composite analysis remains similar when these three extreme events are analyzed individually (not shown). 

\subsection{Model response: from physics to ecosystems}

Because these are major environmental drivers of epipelagic community biomass variability, we first describe in Figure \ref{fig:hov_nemo_ape}a-c the temporal evolution of monthly equatorial anomalies in upper ocean temperatures, chlorophyll concentrations and zonal currents during and after extreme El Niño events, in the form of equatorial time-longitude diagrams, with time of origin as January month preceding the onset of the El Nino event. The warming signal associated with El Niño initiates in the central equatorial Pacific in early spring, spreads rapidly to the eastern Pacific, intensifies during the summer and fall, peaks at the end of the calendar year, and finally rapidly declines and transitions to La Niña conditions the following spring (from April-y1, Fig.\ref{fig:hov_nemo_ape}a). The development phase of El Niño is also characterized by strong  eastward surface currents anomalies in the western and central Pacific (Fig\ref{fig:hov_nemo_ape}c) induced by anomalous westerly winds, promoting warming of the central Pacific and eastward movement of the warm-pool to the eastern equatorial Pacific. These currents anomalies reverse at the peak of El Niño and during La Niña. The simulated plankton concentration anomalies largely mirror those of temperature, with a sharp decrease during El Niño and an increase during La Niña (Fig\ref{fig:hov_nemo_ape}.b). 

\begin{figure}[h!tp]
	\centering
	\includegraphics[scale=0.4]{plot_all_hovmoller_phys_oope.png}	
	\caption{Time-longitude diagrams in the equatorial Pacific of surface temperatures (in °C) (a), low-trophic level concentrations (in mmol/m3) (b), zonal velocity (in m/s) (c) and fish biomass anomalies (in J/m2) associated with extreme El Niño events composite (3cm, 20cm and 90 cm in d, e, f, respectively). The eastern location of the warm pool (28\degree{} isotherm) is shown in red in (a).}	
	\label{fig:hov_nemo_ape}
\end{figure}

A similar analysis is then performed for epipelagic biomass for the three size classes we selected (Fig\ref{fig:hov_nemo_ape}def). Their responses to El Niño share common features: positive biomass anomalies appear near the dateline early in the calendar year and propagate eastward toward the central Pacific until late spring (May/June-y0). This positive biomass anomalies in the central Pacific re-intensify in fall and quickly disappear in winter. They are also accompanied by a decrease in biomass in the western Pacific from the summer of the El Niño year. These negative anomalies persist after the El Niño peak and during the subsequent La Niña event but remain largely confined to the western Pacific. However, there are significant differences in the behaviour of the three size classes, including a westward shift of the response as size classes increase.

Figure \ref{fig:profiles} then describes how these surface ENSO-related signals propagate to depth by providing climatological equatorial profiles of temperatures, zonal velocities, low-trophic level concentrations and  epipelagic biomass for the three size classes in boreal winter, as well as their anomalies for extreme El Niño composites. The climatological temperature profile indicates that the thermocline is deep in the western Pacific and shallow in the east, and flattens during El Nino, resulting in warming in the east and cooling in the west (Fig. \ref{fig:profiles}a). The Equatorial Undercurrent also weakens strongly during El Nino, while strong positive (i.e. eastward) currents anomalies occur near the surface (Fig. \ref{fig:profiles}b). Low trophic level biomass, which is maximal in the upper 50m of the eastern Pacific, decrease during El Nino, due to the flattening of the thermocline, which reduces the nutrient supply in the surface layers (Fig. \ref{fig:profiles}c). Regarding the vertical extent of the epipelagic signal, climatological biomass for the three classes in the western Pacific extends from the surface to $100m$ in the western Pacific and decreases during El Nino events. However, this decrease is not homogeneous along the vertical, with strong positive anomalies appearing around 40m in the west for intermediate and large sizes (Fig. \ref{fig:profiles}def). These are induced by a narrowing of the vertical habitat, due to the shallowing of the thermocline. The climatological biomass decrease rapidly towards the central Pacific to become negligible in the eastern part of the basin. This structure is significantly altered during extreme El Niño events, where fish biomass increases from the surface to $100m$ depth in the central and eastern Pacific, with a maximum located near $40m$.

\begin{figure}[h!tp]
	\centering
	\includegraphics[scale=0.4]{figs/forage_mean_ond97.png}	
	\caption{Pacific equatorial profiles of temperature (a), zonal velocity (b), low-trophic concentration (c) and fish biomass (d for small, e for intermediate and f for large sizes). Mean are represented as black contour lines and El Nino anomalies are represented in colors.}	
	\label{fig:profiles}
\end{figure}

\subsection{Processes driving epipelagic upper-ocean response}

Analysis of the trend terms (equation \ref{eq:apecosm_trend}) allows us to identify the role of the different processes responsible for the response of the epipelagic community to El Niño highlighted in Figure \ref{fig:hov_nemo_ape}. The same equatorial time-longitude diagrams are made for the main trend terms (right members of \ref{eq:apecosm_trend}) and for their integral representing their contribution to the total biomass change. We also analyze various key parameters of the biological response to changing environmental conditions, namely growth rate ($\gamma$ in equation \ref{eq:apecosm_trend}), functional response and predation mortality rates ($M$ in equation \ref{eq:apecosm_trend}). Because the relative importance of these processes varies among size classes, these analyses are discussed separately for each size class.

Figure \ref{fig:fig7} presents a synthesis of the respective contribution of biological (i.e. the combined action of growth and predation) and physical (i.e. the combined action of advection and diffusion) processes on the epipelagic biomass response to ENSO for each of three size classes. For the largest size class, physical processes (Fig \ref{fig:fig7}i) explain most of the biomass changes (Fig \ref{fig:fig7}g), with biological processes being an order of magnitude smaller (Fig \ref{fig:fig7}h). The increase in biomass in the central Pacific and decrease in the west is consistent with passive transport of large organisms by ocean currents from the western to the central Pacific. This advection is due to the strong eastward current anomalies simulated in the western and central Pacific during the onset and development phase of an extreme El Niño (up to 0.6m/s; see Fig\ref{fig:hov_nemo_ape}c). The influence of passive horizontal transport by currents dominates active movements, with ocean current anomalies greater than active velocity anomalies by a factor of $\approx\ 20$, and is broadly similar for all size classes, the relative importance of biological processes increases as fish size decreases. The decrease in biomass in the western Pacific is indeed primarily the result of biological processes for intermediate and small size classes. In the central and eastern Pacific, the combined action of dynamical and biological processes explain biomass increase during El Niño (Fig \ref{fig:fig7}a-f) while these processes largely offset each other when the equatorial Pacific reverses to La Niña conditions, resulting in small biomass changes in this region. There are two main reasons why the relative importance of biological processes decreases with increasing size. First, predation is size-based in APECOSM, resulting in high predation pressure on small organisms, which decreases with size since larger organisms have no predators in the model. The second important biological process that affects biomass is growth, which includes a flux term and a source term (see equation \ref{eq:apecosm_trend}) that are both dependent on temperature. The source term controls biomass production. It varies as ${\gamma}/{w}$, which scales linearly with $w^{-\frac{1}{3}}$ and thus decreases sharply with size.


\begin{figure}[h!tp]
	\centering
	\includegraphics[scale=0.4]{figs/fig7.png}	
	\caption{Time-longitude diagrams in the equatorial Pacific of total (left), biologically (predation plus growth terms, middle) and dynamically (right, advection plus diffusion) terms induced interannual variations in fish biomass (in J/m2) associated with extreme El Niño events composite for small (top), intermediate (middle) and large (bottom) sizes.}
	\label{fig:fig7}
\end{figure}

The action of dynamical processes on the evolution of biomass can simply be explained. It results from the transport of biomass from the western to the central Pacific in response to the strong eastward currents anomalies that occur during extreme El Niño conditions. It is experienced in a similar way by all size organisms. The significant and sometimes dominant contribution of biological processes for small and medium size classes, however, is more difficult to understand intuitively because it results from the combined action of predation mortality and growth. Therefore, we further detail the respective contribution of predation and growth and their driving factors for small and medium size classes on Figure \ref{fig:fig8} and Fig\ref{fig:fig9} respectively. For small size classes (3cm), the effects of predation mortality balance the effects of growth (Fig\ref{fig:fig8}bc),  resulting in a net effect of biological processes that is much smaller than the effect of each biological process considered in isolation (Fig\ref{fig:fig8}a). Growth leads to an increase  of biomass at the onset of El Niño in the central Pacific (between dateline and 150\degree{}W), that spreads eastward to its peak. These positive biomass anomalies then decrease slightly during the following La Niña conditions. This can be related to the evolution of the modeled growth rate, which increases in the central and eastern Pacific during El Niño and decreases slightly during the following La Niña, closely matching the evolution of temperature. Because changes in growth rate are largely driven by temperature, the anomalous warming observed east of the dateline during El Niño increases the growth rate and thus the biomass of small size classes due to the source term associated to growth in equation \ref{eq:apecosm_trend}. Conversely, the cooling observed during the subsequent La Niña conditions decreases the growth rate, dampening the El Niño-induced increase in biomass. In contrast to the central and eastern Pacific, the growth rate decreases in the western Pacific as it cools from June of the El Niño year and  contributing to a decrease in biomass, which reverses during the following La Niña. Again, temperature changes have a major influence on growth rate and biomass of small size class organisms. Despite its importance in controlling growth and reproduction, Fig\ref{fig:fig8}e shows that the functional response is indeed not the primary driver here, as it exhibits negative anomalies that are consistent with the reduction in low-trophic level prey concentrations induced by temperature, which controls the swimming speed of predators (attack rate of the Holling type 2) and the vertical distribution and swarming level of low trophic level prey, which in turn control their availability to predators. As mentioned earlier, predation-induced changes in biomass largely mirror those induced by growth. They act to decrease biomass in the central and eastern Pacific and increase it in the western Pacific, closely following the biomass changes of intermediate size predators. Despite their opposite effect on biomass, the effects of growth dominate those of predation, explaining most of the decrease in small size classes in the western Pacific during El Niño and the subsequent La Niña, and reinforcing the biomass increase in the central Pacific induced by dynamical processes during El Niño.  

\begin{figure}[h!tp]
	\centering
	\includegraphics[scale=0.4]{figs/fig8.png}	
	\caption{Time-longitude diagrams in the equatorial Pacific of interannual anomalies of small sizes trends (J/m2/s; in colors) and time-integrated trends (J/m2; in contours) associated with extreme El Niño events composite for predation plus growth (a), growth (b) and predation (c). Same as (a-c) but for interannual anomalies of functional response (no unit; in colors) and plankton (mmol/m3; in contours) (d), growth rate (UNITS???; in colors) and temperature (°C; in contours) (c) and predation mortality rate (UNITS???; in colors) and intermediate biomass (J/m2; in contour) (f).}
	\label{fig:fig8}
\end{figure}

Growth- and predation-induced biomass changes for intermediate size classes are similar to those simulated for small size classes: they are opposite and of the same order of magnitude, but overall the effects of growth dominate those from predation. Growth increases fish biomass in the central Pacific from the onset to the peak of El Niño, while it decreases the biomass in the western Pacific during the subsequent La Niña. However, for intermediate size classes, the influence of temperature on fish physiology is no longer the dominant factor of biomass changes induced by biological processes as it was for small organisms. In contrast to small size classes, changes in growth rate reflect changes in functional response, which increases in the central Pacific both because of warmer waters (increasing swimming speed that controls the attack rate parameter in the functional response) and increased food availability due to the increase in biomass of small organisms. These processes contribute to increase the biomass of intermediate size organisms. In the western Pacific, the growth rate does not change much (Fig\ref{fig:fig7}b). This is because the intermediate-sized organisms change their vertical distribution (from around 60m to around 30m, cf. Fig\ref{fig:fig8}e) as a result of the shallowing of the thermocline and concentrate at the depth of their prey so that they do not experience neither substantial changes in prey  (Fig\ref{fig:fig8}d) or water cooling (Fig\ref{fig:fig8}a). Thus, the source component of growth is not altered but the advective flux along the weight axis (see equation \ref{eq:apecosm_trend}) decreases during La Niña (Fig\ref{fig:fig7}a) because of the biomass decrease induced both by passive eastward advection by zonal currents anomalies during El Niño and increased mortality rates (Fig\ref{fig:fig7}d). However, predation acts to dampen the effects of growth, reducing biomass in the central Pacific due to increased predation by large size classes there and increasing biomass in the western Pacific during the subsequent La Niña due to reduced predation. The changes induced by the combination of these two processes are dominated by growth however and resembles those obtained for small sizes, albeit with a modest westward shift. As with small sizes, the decrease in biomass in the western Pacific during El Niño and the subsequent La Niña are largely due to a reduction in growth, while dynamical processes dominate the biomass increase east of the dateline with a smaller contribution from growth.

The strong increase in predation on small organisms in the west (180E) leads to a decrease in biomass. This decline intensifies and spreads eastwards in response to the decrease in growth rate associated with the cooling of the western Pacific waters. This decrease in biomass in turn explains the anomalies in the growth term, which is dominated by the biomass anomalies (source term in equation \ref{eq:apecosm_trend}).


\begin{figure}[h!tp]
	\centering
	\includegraphics[scale=0.4]{figs/fig9.png}	
	\caption{Same as Figure \ref{fig:fig8} but for intermediate sizes.}	
	\label{fig:fig9}
\end{figure}

% Changes in large fish biomass are dominated by advection/diffusion transport processes, while predation and growth have virtually no impact because their importance decreases structurally with size (see above). 

%ADD HERE A SMALL PARAGRAPH SUMMARIZING THE DOMINANT PROCESSES DRIVING BIOMASS SHIFT FOR ALL SIZE CLASSES.


\subsection{Generalization}

%The patterns of the functional response and growth rates are very different for small sizes (figures YYY.a and YYY.d). The former shows negative anomalies on the central Pacific during the onset of El Nino conditions, presumably due to the concomitant reduction of plankton concentration (figure YYY.b). Interestingly, no anomalies are found in the eastern basin, due to compensating effects of warming temperatures and reduced plankton concentrations. The growth rate shows a pattern that is very similar to the temperature anomalies (Fig. XXX.a), indicating the dominance of temperature on the growth rate. The mortality rate (Fig. YYY.g) shows a pattern that is consistent with the increased biomass of intermediate epipelagics (XXX.b), which predate on small ones.  Besides, growth rate anomalies superimpose well on the small biomass anomalies (black contours in Fig. YYY.d), hence suggesting that the increased biomass during the El Nino conditions is triggered by enhanced growth associated with warmer temperature, despite a dampening effect of increased mortality rates.
%For intermediate sizes, the functional response and the growth rate show very similar anomalies, hence suggesting that they both are driven by the same factors. Both show positive anomalies during the onset of El Nino between 200E and 250E, presumably induced by the temperature warming that favours both the growth and search rates. Then negative anomalies occur, which are westward shifted relative to the positive ones. These negative anomalies are likely induced by the cooling associated with the La Nina conditions reinforced by the reduced small fish biomass that appears near 200E, as shown in Fig. XXX.d. Mortality rates anomalies are consistent with the increase of large fish biomass (Fig. XXX.f). Comparing the different fields, we can suggest that the increased fish biomass that appears on the central Pacific in early 1997 is first dominated by advective processes, which returns a similar pattern (fig YYY.k). Then, during the El Nino peak, the increased biomass of intermediate fish is driven by both an increased growth rate and an increased functional response, which are induced by warmer temperatures and more food available (more small epipelagic fishes, Fig. XXX.d).

%\subsection{Generalisation}
%
%In order to see if ecosystem response to the 97 El Nino event is representative, the covariance between fish biomass anomalies and the ONI index have been computed, following the methodology described in section \ref{sec:cov}. The maps are presented in Fig\ref{fig:ape_cov}.
%
%\begin{figure}
%	\centering
%	\includegraphics[scale=0.6]{figs/fig7.png}	
%	\caption{Covariance maps between the ONI index and the vertically integrated fish biomass anomalies for small (a), intermediate (b) and large (c) sizes.}	
%	\label{fig:ape_cov}
%\end{figure}
%
%The covariance patterns are very similar to the biomass anomalies depicted in Fig\ref{fig:mean_ond97_ape}, therefore suggesting that the mechanisms described in the above  apply to the general case. However, we note that the covariance patterns are westward shifted compared to the fish biomass anomalies. 
%
%This westward shift might be explained by the fact that when performing covariance anomalies, different El Nino events (Easter Pacific El Nino, Central Pacific El Nino) and La Nina events are considered, which ultimately impacts the global view of fish biomass response to ENSO variability.


%In this section, the interannual response of epipelagics to ENSO variability is investigated using covariance analysis. The left panels of Figure 3. show the yearly mean vertically integrated fish biomass over the entire simulation for the three different size classes. For small sizes (Figure 3a), the fish biomass is concentrated at around 10° S and 10 ° N in the central Pacific and close to the equator in the western Pacific. High biomass concentration is also found east of 90° W, off the coasts of Chile. As size increases, the equatorial "blue spot" extends meridionnally and to the west. This pattern is mostly driven by the active and passive advection of fishes in the Apecosm model (REF). Without advection, the biomass will be concentrated at the equator, where the plankton concentration is the maximum.
%
%Covariance maps between vertically integrated fish biomass and the winter ONI index are shown in the right panels of Figure 3. Small epipelagics show negative anomalies in the Western Equatorial Pacific and positive anomalies in the Central Equatorial Pacific. This pattern can be interpreted as an eastern displacement of the mean biomass in the Western Pacific. Similar dipolar patterns are also obtained for intermediate and large sizes, but the anomalies westward shifted as size increases. This can be interpreted, as for small sizes, by a westward shift of fish biomass during positive El Niño phases. The same results have been obtained when the covariance analysis is performed on monthly anomalies (not shown).

All analyses presented above focused on the equatorial Pacific, where ENSO physical and biogeochemical signature is the strongest. To ascertain the response of off equatorial regions to ENSO,  Fig\ref{fig:mean_ond97_ape} further provides maps of climatological epipelagic biomass for the three size classes in boreal winter, as well as their anomalies for extreme El Niño composites.  On average, epipelagic fish biomass is largest on each side of the equator and in the equatorial western Pacific (Fig\ref{fig:mean_ond97_ape}abc), while smaller biomasses  found the eastern Pacific. In agreement with the equatorial analyses provided on Figure\ref{fig:hov_nemo_ape}, Fig\ref{fig:mean_ond97_ape}a indicates that, during El Niño, small epipelagic fish biomass biomass increases in the equatorial eastern Pacific and decreases in the western Pacific. In addition, Fig\ref{fig:mean_ond97_ape}a reveals that this biomass also decreases on each side of the equator, further highlighting that the biomass does not only shift eastward during El Niño but also and equatorward.
As size increases, positive anomalies associated with El Nino conditions expands westward and poleward while negative anomalies weakens and expands equatorward. 

\begin{figure}[h!tp]
	\centering
	\includegraphics[scale=0.4]{figs/map_mean_anom_OND_97.png}	
	\caption{Maps of interannual biomass anomalies for extreme El Niño events composite in boreal winter (DJF) (left column) and covariance of fish biomass anomalies with the ONI index (right column) for small (upper line), intermediate (middle line) and large sizes (lower line). Black contours show the contours of climatological biomass density (log-scale).}	
	\label{fig:mean_ond97_ape}
\end{figure}


To insure that the biomass response described for extreme El Nino events is representative of ENSO events in general, Fig\ref{fig:mean_ond97_ape}def shows the covariance maps computed between the monthly ONI index and the detrended fish biomass anomalies for the three size classes over the 1958-2018 period. This analysis reveals that patterns of extreme El Niño composites and of covariance analysis are very similar, although amplitudes are about four times larger for extreme events. This  difference  in amplitude is related to the fact that the covariance analysis also includes weaker El Niño events (such as the 1986, 1991, 1994, 2002, 2004 and 2009 events) as well as La Nina events, which are known to have weaker physical and biogeochemical signatures. Nevertheless, the very good match between the covariance maps and the extreme El Nino composites indicates that the biomass response and related mechanisms discussed above for the three major El Niño events are representative of those of other ENSO events.