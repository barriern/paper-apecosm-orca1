\section{Data and method}

\subsection{El Nino index}
\label{sec:oni}

In the present study, the internannual ENSO variability is inferred from the Oceanic Nino Index (\url{https://www.cpc.ncep.noaa.gov/data/indices/oni.ascii.txt}, hereafter ONI), which is a 3-month running mean of the sea-surface temperature anomalies averaged over the Niño 3.4 region (5N-5S, 170W-120W).\\

\subsection{Statistical tools}

\subsubsection{Covariance and correlation analysis}
\label{sec:cov}

In order to assess the impacts of climate variability (ONI, TPI) on temperature, chlorophyll or fish biomass, covariance analysis has been extensively used. First, the monthly climatology of the analysed fields is computed and subtracted from the monthly data, hence leading to monthly anomalies. Then, these monthly anomalies are detrended. Finally, the covariance between the resulting time series and the climate indexes are computed for each grid point considered. 

Correlations are computed in the same way. Their significance has been inferred using a Student t-test, in which the effective number of degrees of freedom has been corrected based on the 1-lag autocorrelation of the 2 two time-series \citep{brethertonEffectiveNumberSpatial1999}.

\subsection{Observation-based sea-surface temperature data}

Observation-based estimates of sea-surface temperatures are extracted from the the Hadley Sea-Surface Temperature (HadISST1, \citealt{raynerGlobalAnalysesSea2003}). It uses reduced space optimal interpolation applied to SSTs from the Marine Data Bank (mainly ship tracks) and ICOADS through 1981 and a blend of in-situ and adjusted satellite-derived SSTs for 1982-onwards. For this study, monthly sea-surface temperature anomalies have been extracted and detrended for the 1958-2018 period.

\subsection{Observation-based sea-level anomalies}

Observation-based sea-level anomalies have been extracted from the reprocessed Global Ocean Gridded L4 Sea-Surface Heights data, provided by the Copernicus Climate Service (\url{https://doi.org/10.48670/moi-00148}), provided as monthly values on a regular $0.25 \times 0.25$ grid from 1993-01 to 2020-12-31. This product was processed by the DUACS multimission altimeter data processing system, which processes data from all altimeter missions: Jason-3, Sentinel-3A, HY-2A, Saral/AltiKa, Cryosat-2, Jason-2, Jason-1, T/P, ENVISAT, GFO, ERS1/2. Using along-track altimeter data, gridded sea-level anomalies are extracted by using an Optimal Interpolation off all the flying satellites.

\subsection{Observation-based chlorophyll data}

Chlorophyll observation-based estimates are extracted from the monthly OceanColour-CCI V5 CHL-a dataset\footnote{http://dx.doi.org/10.5285/1dbe7a109c0244aaad713e078fd3059a}  \citep{sathyendranathOceanColourTimeSeries2019} over the 1997-09/2018-12 period. For our purpose, the high resolution dataset (4 km) has been regridded on a regular 1-degree resolution grid by computing weighted Chl averages over $24\times24$ boxes, with the weights being provided by the the cosine of latitude. When more than $1/3$ of the data used in the averaging is missing, the regridded cell is masked. The monthly climatology of Chl has then been computed over the 1998-2008 period and used to extract the Chl anomalies over the full period.

\subsection{Observation-based fish biomass estimates}

Fish biomass estimates are extracted from the IRD Level 2 global monthly catch of tuna, tuna-like and shark species dataset\footnote{https://doi.org/10.5281/zenodo.1164128} \citep{taconetGlobalMonthlyCatch2018}. First, the purse-seine captures of skipjack and yellowfin tunas have been extracted from the raw input file. Then, non-monthly observations (i.e. when the difference between the end and start dates exceed 31 days) have been discarded, as well as those which are unlocalized. Finally, the remaining observations have been regridded into a regular 1-degree reolution grid using the overlapping area between the observation polygon and the cell area. The final product is a $3D$ array of dimensions \verb+(time, y, x)+ that extends from 1959 to 2016.

\subsection{Physical and biogeochemical forcing}
\subsection{Physical and biogeochemical model}
\label{sec:nemo}

The three-dimensional physical and biogeochemical fields required to run APECOSM are extracted from an oceanic simulation performed with the physical ocean model NEMO (Nucleus for European Modelling of the Ocean, \citealp{madecNEMOOceanEngine2019}) coupled to the ocean biogeochemical model PISCES (Pelagic Interaction Scheme for Carbon and Ecosystem Studies, \citealp{aumontPISCESv2OceanBiogeochemical2015}). 

NEMO simulates the dynamics and thermodynamics of the physical ocean. Prognostic variables are
the three-dimensional velocity field, a non-linear sea surface height, the
conservative temperature and the absolute salinity, distributed on a three-dimensional Arakawa C-type grid. Density is computed from potential temperature, salinity and pressure using the \cite{iocInternationalThermodynamicEquation2010} equation of state. Vertical mixing is parameterized from a turbulence closure scheme based on a prognostic vertical turbulent kinetic equation, which has been shown to perform well in the tropics before \citep{blankeVariabilityTropicalAtlantic1993}. Lateral mixing acts along isopycnal surfaces, with a Laplacian operator and $200 m^2 s^{-1}$ constant isopycnal diffusivity coefficient \citep{lengaigneImpactIsopycnalMixing2003}. Shortwave fluxes penetrate into the ocean based on a single exponential profile \citep{paulsonIrradianceMeasurementsUpper1977} corresponding to oligotrophic water (attenuation depth of 23 m). 

PISCES is a biogeochemical model of intermediate complexity with 24 prognostic variables designed for global ocean applications \citep{aumontPISCESv2OceanBiogeochemical2015}. It simulates the biogeochemical cycles of oxygen, carbon and the main nutrients controlling phytoplankton growth (nitrate, ammonium, phosphate, silicic acid, and iron) and the lower trophic levels of marine ecosystems, distinguishing four plankton functional types based on size: two phytoplankton groups ("small phytoplankton" -e.g. nanophytoplankton- and "large phytoplankton" -e.g. diatoms-) and two zooplankton groups ("small zooplankton" -e.g. microzooplankton- and "large zooplankton" -e.g. mesozooplankton-). It also includes small and large particulate organic matter.

The NEMO-PISCES simulation used in this study is deployed on the tripolar ORCA1 grid \citep{madecGlobalOceanMesh1996}, with a 1\degree{} nominal horizontal resolution and a refined 1/3\degree{} meridional resolution in the equatorial band. Its vertical resolution ranges from 1m at the surface to 100m at 1 kilometer depth and varies over time, following \cite{levierFreeSurfaceVariable2007}. This simulation  is forced over the period 1958-2018 with atmospheric inputs from the JRA atmospheric reanalysis \citep{kobayashiJRA55ReanalysisGeneral2015}, which is representative of  surface atmospheric variability observed over the historical period. 

%Temperature, ocean transports, oxygen, plankton concentration (diatoms, mesozooplankton and microzooplankton, big particulate organic matter), photosynthetically active radiation (PAR) and the layer thickness from this simulation are then used to force the Apecosm ecosystem model.

\subsection{Marine Ecosystem Model}
\subsection{Marine ecosystem model}
\label{sec:apecosm}

We use the Apex Predators Ecosystem Model (Apecosm, \citealp{mauryModelingEnvironmentalEffects2007, mauryOverviewAPECOSMSpatialized2010}) to simulate the energy transfer through marine ecosystems. 
APECOSM is an eulerian ecosystem model that represents mechanistically the 3D dynamics of size-structured pelagic populations and communities. It integrates individual, population and community levels and includes the effects of life-history diversity with a trait-based approach \citep{mauryIndividualsPopulationsCommunities2013}. In APECOSM, the uptake and use of energy for individual growth, development, reproduction, somatic and maturity maintenance are modelled according to the DEB theory \citep{koojmanDynamicEnergyBudget2010}. The model considers important ecological processes such as opportunistic size-structured trophic interactions and competition for food, predatory, disease, ageing and starvation mortality, key physiological aspects such as vision and respiration, as well as essential behaviours such as 3D passive transport by marine currents and active habitat-based movements (\cite{faugerasAdvectiondiffusionreactionSizestructuredFish2005}), schooling and swarming (see \citealp{mauryModelingEnvironmentalEffects2007, mauryIndividualsPopulationsCommunities2013, mauryCanSchoolingRegulate2017} for a detailed description of the model). APECOSM is driven by 3D outputs from the physics-biogeochemistry coupled model NEMO-PISCES (3D fields of temperature and horizontal currents, vertical mixing, small and large phytoplankton, small and large zooplankton, detritus, light and oxygen) that constrain the biological and ecological dynamics at various levels.
The bio-energetic bases of the model are based on the Dynamic Energy Budget theory (DEB, \citealp{koojmanDynamicEnergyBudget2010}). All the metabolic rates are temperature-dependent and corrected by an Arrhenius factor (Maury, 2007; Maury and Poggiale, 2013). In the model configuration used here, we do not prescribe a limited temperature range any of the simulated communities.

The dynamics of communities is determined by integrating the core state equation below:

\begin{equation}
\partial_t \varepsilon = \underbrace{- \partial_w(\gamma \varepsilon) + \frac{\gamma}{w}\varepsilon}_{Growth} 
\underbrace{- M \varepsilon \vphantom{\frac{\gamma}{w}\varepsilon}}_{Mortalities}
\underbrace{-\overrightarrow{\nabla}.(\overrightarrow{V} \varepsilon) \vphantom{\frac{\gamma}{w}\varepsilon}}_{3D Adv} 
\underbrace{+ \overrightarrow{\nabla} . (D \overrightarrow{\nabla} \varepsilon) \vphantom{\frac{\gamma}{w}\varepsilon}}_{3D Diff.}
\label{eq:apecosm_trend}
\end{equation}

where $\varepsilon$  is the fish biomass density, $w$ the weight, $\gamma$ is the growth rate, $M$ represent the different mortality rates (computed using equation 12 of \citealt{mauryIndividualsPopulationsCommunities2013}), $V$ and $D$ the sum of passive and active velocities and diffusivity coefficients (computed following \citealt{faugerasAdvectiondiffusionreactionSizestructuredFish2005}). Reproduction is considered through a Dirichlet boundary condition in w0 that accounts for by the reproductive output from all mature organisms.

In APECOSM, the energy ingested by organisms fuels individual metabolism according to the DEB theory. Ingestion is proportional to a functional Holing type II response function that depends on the visibility of prey, their aggregation in schools and temperature. This functional response can be written in a simplified way as follows:


The APECOSM simulation used in this study is forced by temperature, horizontal current velocity, dissolved oxygen concentration, low-trophic level concentration (diatoms, mesozooplankton and microzooplankton, big particulate organic matter), photosynthetically active radiation (PAR) and dynamical layer thickness outputed from the NEMO-PISCES simulation (section \ref{sec:nemo}). It uses a daily time-step for biological processes, which is decomposed into a day/night cycle, whose duration depends on the latitude and on the day of the year \citep{forsytheModelComparisonDaylength1995}. A sub time-stepping ($dt =0.8h$) is used for horizontal advection and diffusion in order to insure numerical stability.

%In order to assess the mechanisms of fish biomass response to ENSO variability, biomass changes induced by growth, predation, advection and diffusion are stored for the entire simulation. Therefore, the biomass for a given size class, at a given location and for a specific time can be reconstructed by using the following equation:
%
%\begin{equation}
%\varepsilon(T) = \varepsilon(T=0) + \int_{t=0}^{T} \left[ 
%T_{pred}(t) +
%T_{growth}(t) + 
%T_{adv}(t) + 
%T_{diff}(t) 
%\right] dt 
%\label{eq:rec_oope}
%\end{equation}
%
%with $\varepsilon(x,y,s,T=0)$ the fish biomass at the beginning of the simulation and $T_{pred}$, $T_{growth}$, $T_{adv}$ and $T_{diff}$ the respective biomass increments due to predation, growth, advection and diffusion, given by equations \ref{eq:pred}, \ref{eq:growth} and \ref{eq:move}, respectively.

Three interactive communities are simulated in the present study:
\begin{itemize}
\item{The epipelagic community, which includes organisms feeding during the day near the surface such as yellowfin or skipjack tunas for instance. Its vertical distribution is influenced by light, visible food, temperature and oxygen while its functional response is influenced by light.}
\item{The migratory mesopelagic community, which feeds in the surface layer at night and migrate to deeper waters during daytime. Its vertical distribution is influenced by light and visible food during the night.}
\item{The resident mesopelagic community, which remains at depth during both night and day. Its vertical distribution is influenced by light and visible food during the day.}
\end{itemize}

For each community, equation (1) is integrated over 100 size classes, ranging from $0.123cm$ to $196cm$. In order to ensure that the size-spectrum is fully unfolded and a pseudo stationary regime is reached, the model has been integrated three times. First, it has been initialized with an arbitrary small biomass value in every size-class and integrated from 1958 to 2018 (61 years). Then, the end of this first spin-up phase has been used to run another cycle, which was in turn used to initialize the simulation presented in the following.

%Epilagic feed on other epilagic fish only during night daytime. Migrants feed on other migrants and epipelagics, only during night-time. While mesopelagics feed on migrant and mesopelagic during daytime, and only on other mesopelagic during night-time.

In the following, the focus is put on the epipelagic community only , since its near-surface location makes it more sensitive to ENSO variability \citep{lemezoNaturalVariabilityMarine2016} and since it corresponds to organisms such as skipjack and yellowfin that are targeted by industrial purse seine fleet that represent the bulk of tuna catches in the region and that have been reported to respond to ENSO (Lehodey, 1997).
