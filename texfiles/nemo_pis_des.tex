\subsection{Physical and biogeochemical model}
\label{sec:nemo}

3D physical and biogeochemical fields are extracted from an oceanic simulation performed with the NEMO (Nucleus for European Modelling of the Ocean, \citealp{madecNEMOOceanEngine2019}) dynamical ocean model coupled to the PISCES biogeochemical model (Pelagic Interaction Scheme for Carbon and Ecosystem Studies, \citealp{aumontPISCESv2OceanBiogeochemical2015}). 

NEMO simulates the dynamics and thermodynamics of the physical ocean. Prognostic variables are
the three-dimensional velocity field, a non-linear sea surface height, the
conservative temperature and the absolute salinity, distributed on a three-dimensional Arakawa C-type grid. Density is computed from potential temperature, salinity and pressure using the \cite{iocInternationalThermodynamicEquation2010} equation of state. Vertical mixing is parameterized from a turbulence closure scheme based on a prognostic vertical turbulent kinetic equation, which has been shown to perform well in the tropics before \citep{blankeVariabilityTropicalAtlantic1993}. Lateral mixing acts along isopycnal surfaces, with a Laplacian operator and $200 m^2 s^{-1}$ constant isopycnal diffusivity coefficient \citep{lengaigneImpactIsopycnalMixing2003}. Shortwave fluxes penetrate into the ocean based on a single exponential profile \citep{paulsonIrradianceMeasurementsUpper1977} corresponding to oligotrophic water (attenuation depth of 23 m). 

PISCES is a biogeochemical model of intermediate complexity with 24 prognostic variables designed for global ocean applications \citep{aumontPISCESv2OceanBiogeochemical2015}. It simulates the biogeochemical cycles of oxygen, carbon and the main nutrients controlling phytoplankton growth (nitrate, ammonium, phosphate, silicic acid, and iron) and the lower trophic levels of marine ecosystems, distinguishing four plankton functional types based on size: two phytoplankton groups (small for nanophytoplankton and large for diatoms) and two zooplankton groups (small for microzooplankton and large for mesozooplankton). It also includes small and large particulate organic matter.

The NEMO-PISCES simulation used in this study is deployed on the tripolar ORCA1 grid \citep{madecGlobalOceanMesh1996}, with a 1\degree{} nominal horizontal resolution and a refined 1/3\degree{} meridional resolution in the equatorial band. Its vertical resolution ranges from 1m at the surface to 100m at 1 kilometer depth and varies over time, following \cite{levierFreeSurfaceVariable2007}. It extends from 1958 to 2018 and is forced with atmospheric inputs from the JRA atmospheric reanalysis \citep{kobayashiJRA55ReanalysisGeneral2015}, which is representative of observed variability over the historical period. 

%Temperature, ocean transports, oxygen, plankton concentration (diatoms, mesozooplankton and microzooplankton, big particulate organic matter), photosynthetically active radiation (PAR) and the layer thickness from this simulation are then used to force the Apecosm ecosystem model.