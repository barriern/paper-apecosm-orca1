\section{Introduction}

The interannual variability of the Tropical Pacific Ocean is largely dominated by the El Nino/ Southern Oscillation (ENSO, \citealt{trenberthNinoSouthernOscillation2019}). During normal conditions, trade winds pile up warm surface water in the west, while colder subsurface waters upwell in the east along the west coast of America. The resulting east-west temperature gradient (around 10°C) reinforces the atmospheric Walker circulation and the associated trade winds. The latter are also responsible for an ocean divergence that induces the onset of an equatorial upwelling, which is associated with strong primary production (known as the “Cold Tongue”). During the warm ENSO phase, known as El Nino, the trade winds weaken and the warm pool moves to the east and reaches the central equator, while the thermocline shoals in the east and deepens in the west. In addition, the strength of the equatorial upwelling and the primary production are largely reduced.

In addition to its worldwide impacts on climate (\citealt{toniazzoInfluenceENSOWinter2006} for the impacts of ENSO on winter North Atlantic climate, \citealt{juAsianSummerMonsoon1995} for the impacts on the Asian summer monsoon) and health \cite{kovatsNinoHealth2003}, ENSO variability has been shown to strongly impact the biogeochemistry and biology of the tropical Pacific Ocean, a region that strongly depends on fisheries (REF). \cite{chavezBiologicalChemicalResponse1999} have investigated, using biological, chemical and physical sensors combined with satellite chlorophyll observations, the impacts of the 1997-1998 El Nino event. They show that the weakening of the equatorial upwelling induced a depletion of nitrate and chlorophyll, which were recovered in April 1998, after the onset of La Nina conditions. \cite{gierachBiologicalResponse19972012}, using satellite observations and ocean reanalysis, have compared the processes involved in the 1997-1998 and 2009-2010 El Nino events. In the central Pacific, chlorophyll-a was more reduced during the 2009-2010 event because of stronger eastward advection of nutrient depleted waters from the western Pacific. While in the eastern Pacific, the impacts of the 1998 event were larger because of a strongest reduction of vertical advection and mixing.

These impacts of ENSO variability on the physical and biogeochemical ultimately impacts the fish distribution, either through habitat conditions (oxygen, temperature) and/or food availability. \cite{lehodeyNinoSouthernOscillation1997}  have shown, using United States purse seine catch per unit effort (CPUE), that spatial shifts in the skipjack population are linked to large zonal displacements of the warm pool that occur during ENSO  events. In spite of its low primary production, they suggest that the warm pool provides an habitat capable of supporting this highly productive tuna population, although the source of the forage remains uncertain.

%Opening to modelling (highlights of Apecosm, which includes advection and vertical)
%Using an ensemble approach, \cite{lotzeGlobalEnsembleProjections2019} have investigated the impacts of global warming on the marine biomass of the global oceans. They project a global decline of marine biomass, mainly driven by an increase of the mean temperature and by a decrease in the primary production. They also suggest that this decrease is more pronounced for higher trophic levels. They also point a considerable spatial variability, with a decrease in fish biomass at middle to low latitudes, and an increase at higher latitudes.

The paper is organized as follows. The physical-biogeochemical and ecosystem models used in this study are presented in section 2. The sea-surface temperature and chlorophyll simulated by the coupled physical biogeochemical model will be confronted to observation based datasets, in order to insure that it properly simulates the ocean response to ENSO variability (section 3). Then, the response of epipelagic fish biomass to ENSO variability will be investigated (section 4). Discussions and perspectives are provided in section 5.
