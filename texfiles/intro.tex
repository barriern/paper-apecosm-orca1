\section{Introduction}


The interannual variability of the Tropical Pacific Ocean is largely dominated by the \enso . During normal conditions, trade winds pile up warm surface water in the west, while colder subsurface water upwells in the east along the west coast of America. The resulting east-west temperature gradient (around 10\degree C) reinforces the atmospheric Walker circulation and the associated trade winds. But during the warm phase ENSO phase, known as \nino , the trade winds weaken and the warm pool moves to the east, reaching the central equator, while the thermocline shoals in the east and deepens in the west. 

In a warming climate, \nino\ events are expected to become more frequent and more intense \citep{caiENSOGreenhouseWarming2015}.

Using an ensemble approach, \cite{lotzeGlobalEnsembleProjections2019} have investigated the impacts of global warming on the marine biomass of the global oceans. They project a global decline of marine biomass, mainly driven by an increase of the mean temperature and by a decrease in the primary production. They also suggest that this decrease is more pronounced for higher trophic levels. They also point a considerable spatial variability, with a decrease in fish biomass at middle to low latitudes, and an increase at higher latitudes. 

The paper is organized as follows. The sea-surface temperature and chlorophyll simulated by the coupled physical biogeochemical model will be confronted to observation based datasets, in order to insure that it properly simulates the ocean response to ENSO variability (section \ref{sec:pisces}). 

