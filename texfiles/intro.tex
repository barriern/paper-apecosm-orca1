\section{Introduction}

\textbf{Societal relevance.} Improving our understanding of how climate variability and change impacts marine ecosystems is a critical issue for the countries bordering the tropical Pacific and harvesting its marine resources. Marine ecosystems in the tropical Pacific Ocean underpin a variety of small-scale artisanal fisheries that are critical for food security and livelihood in most tropical Pacific islands. They also support major oceanic fisheries that are responsible for 60\% of the world tuna catches, which substantially contribute to the revenue of most Pacific Island Countries and Territories. 

\textbf{Tropical Pacific climatology.} The physics, biogeochemistry and biology are tightly coupled in the tropical Pacific, which makes it an ideal region to study biophysical interactions. This oceanic region is characterized by prevalent trade wind conditions, which accumulate a pool of warm water in the western equatorial Pacific and induce an upwelling of cold nutrient-rich subsurface waters that cools the entire central and eastern equatorial Pacific. The Eastern Tropical Pacific upwelling of, leading to an oligotrophic west Pacific warm pool and a mesotrophic, upwelling cold tongue (e.g., \cite{leborgneCarbonFluxesEquatorial2002}). Tropical skipjack and yellowfin tunas mainly inhabit and reproduce in warm surface waters above 26°C in the western and northeast Pacific and feed opportunistically on a wide range of small epipelagic planktonic and nektonic prey. Smaller quantities of the tropical skipjack, yellowfin and bigeye tunas (Thunnus obesus) and the temperate albacore tuna (Thunnus alalunga) are also taken by industrial longliners off equator.

\textbf{ENSO physical and biogeochemical response.} The tropical Pacific is also home to the most energetic earth’s year-to-year climate variations, the El Niño/ Southern Oscillation (ENSO) (\cite{caiChangingNinoSouthern2021}), with strong impacts on the physical and biogeochemical properties of the tropical Pacific Ocean.  An El Niño event (i.e. the warm phase of ENSO) is characterized by a thermocline and nutricline deepening in the central/eastern Pacific, which reduces of upward vertical nutrient and cold water flux, leading to anomalously warm sea surface temperature anomaly (SSTA) and reduced productivity there  (e.g. \cite{chavezBiologicalChemicalResponse1999a, murtuguddeOceanColorVariability1999}). In contrast, the opposite stands for the western Pacific Ocean, which experiences a thermocline/nutricline shoaling resulting in a weak cooling and a productivity increase. The zonal eastward advection of nutrient‐poor warm pool waters by anomalous eastward currents further contributes to the decrease in biological productivity in the central Pacific (e.g. \cite{chavezBiologicalChemicalResponse1999a, picautOceanicZoneConvergence2001}). La Niña (i.e. the cold ENSO phase) can broadly be viewed as a mirror image of El Niño despite some asymmetrical features: El Niño events can indeed occasionally reach much larger amplitudes than La Niña events, like in 1982, 1997 and 2015.  

\textbf{Synchronous ecological response to ENSO.} These physical and biogeochemical impacts of ENSO variability ultimately cascade on the fish distribution, either through changes in habitat conditions (oxygen, temperature) and/or food availability (\cite{aNinoSouthernOscillation2020}). For instance, the spatial distribution of purse seine catches in the western Pacific is strongly influenced by ENSO events, with catches typically expanding to the east during El Niño events and shifting back to the west during La Niña years (\cite{lehodeyNinoSouthernOscillation1997}), in conjunction with the migration of the convergence zone at the eastern edge of the warm pool, where important aggregating mechanisms occur for phytoplantkon (\cite{picautOceanicZoneConvergence2001}). The strong temperature gradient of the thermocline exerts a strong control on the vertical tuna habitat (e.g. \cite{schaeferMovementsBehaviorHabitat2002}) and its shoaling in the western Pacific during El Niño may squeeze their thermal and feeding vertical habitat, favouring their catchability (\cite{bertrandHydrologicalTrophicCharacteristics2002}). 

\textbf{Delayed ecological response to ENSO.} ENSO may not only have a synchronous effect on ecosystems through horizontal and vertical displacement of fish population but also a delayed impact on fish abundance by affecting the survival of larvae, which variability propagates through the population structure and can be detected with some delay in the adult population. Despite the absence of direct larvae abundance surveys, higher skipjack recruitments has been reported two to four years following a bloom in the central Pacific, suggesting that La Niña-related high primary productivity is favorable to nursery and feeding conditions (\cite{yenSpatialTemporalVariations2016}). El Niño has also been reported to negatively impact gonad maturation and mean length of skipjack tuna with time lags of 12 and 7 months, respectively (\cite{kimEffectsClimateinducedVariation2015}). For long‐living species, such as bigeye and albacore, the decreasing growth rate with age leads to a smoothed and damped recruitment variability associated with ENSO, while it combines with the internal dynamic processes of the species propagating toward the older cohorts (e.g. \cite{seninaImpactsRecentHigh2017}). Catches of albacore have however been reported to decrease between 4 to 8 years after an El Niño onset south of 10°S, a time interval that would be expected before the recruitment for the fish spawned during ENSO episodes (\cite{luRelationshipNinoSouthern1998}). A similar relationship has also been found in the northern Pacific (\cite{zhangStudyRelationshipsLargescale2014}), a stock productivity decrease lagging by four years El Niño events in this region.  Given that all tuna species spawn in tropical warm waters, it can be assumed that their spawning habitats and the subsequent fish recruitment are all impacted by ENSO variability (\cite{zhangStudyRelationshipsLargescale2014}), resulting in a delayed fluctuation in abundance of poleward moving adult tuna. This may also be the case for the Japanese eels spawning in the subtropical northwestern Pacific, where surface currents changes associated with ENSO could impact larval transport and the recruitment of juveniles in the northwestern Pacific (\cite{hsiungEffectENSOEvents2018}).

\textbf{Lower-frequency variations.} Marine populations dynamics do not only lag but may also integrate ENSO interannual variations, resulting in a marine population response at lower decadal timescales, characterized by potentially strong and prolonged apparent state transitions (\cite{lorenzoDoubleintegrationHypothesisExplain2013a}). Decadal to multi-decadal variations tuna population have indeed been partly attributed to ENSO-related low-frequency climate variability (e.g. \cite{lehodeyModellingClimaterelatedVariability2003,  singhImpactClimaticFactors2015, singhENVIRONMENTALCONDITIONSARE2017, wuDeterminingEffectMultiscale2020a}), such as the Pacific Decadal Oscillation (PDO). For instance, the PDO is leading by four to five years the recruitment and spawning stock biomass of albacore tuna in both the south (Lehodey et al. 2003; Singh et al. 2015) and the north Pacific (\cite{singhENVIRONMENTALCONDITIONSARE2017}) and by one to five years the yellowfin tuna recruitment in the western Pacific Ocean (\cite{wuDeterminingEffectMultiscale2020a}). The PDO is also leading by three years the recruitment of big-eye tuna mature cohorts in the western central Pacific Ocean (\cite{lanEffectsClimateChange2021}).  This has been related to synchronous shifts in the pelagic ecosystems at low trophic levels (eggs, Pacific saury, neon flying squid) and immature cohort of big-eye tuna with the PDO.

\textbf{Ecosystem modelling.} Most of the observational studies discussing the influence of tropical climate variability on marine ecosystems in the equatorial Pacific mainly rely on tuna catch data, which variability is not only controlled by climate variations but also by fishing pressure and internal ecosystem dynamics. These observations are thus heterogeneous, limited to the surface, focused on a few adult species and their embedded climate signals are aliased by other external factors. Several ecosystem models gathered in the Fisheries and Marine Ecosystem Model Intercomparison Project (Fish-MIP) have been developed to better characterize and understand ecosystems response to climate variations. These models have mainly been used to project biomass changes in response to climate change, generally pointing to a global decline of marine biomass more pronounced for higher trophic levels in response to temperature increase and primary production decrease (\cite{lotzeGlobalEnsembleProjections2019a}). While these models are now routinely used to project future biomass changes, very few studies did attempt to understand how these ecosystem models respond to past climate variability. This is however a must because (1) an accurate representation of past fish biomass variations would allow a more confident assessment of future projections and (2) a better understanding of the processes response for past variability would provide some guidance to understand future changes. To date, a single ecosystem model  (SEAPODYM, \cite{lehodeySpatialEcosystemPopulations2008}) has been used to evaluate the ecosystem response to ENSO in the tropical Pacific, focusing on tuna spatial population. This model reproduces the large-scale east-west displacement of skipjack tuna population in the equatorial Pacific in response to ENSO and attributes these migrations to ENSO-related changes in temperature, prey and oxygen concentrations. It also suggests that El Niño not only drives an eastward tuna displacement but also favour a strong larvae recruitment (\cite{seninaParameterEstimationBasinscale2008}). 

The modeling studies described in the above, however, have some drawbacks. First, all the models used simulate the marine ecosystem in 2D, although the impacts of El Nino events on the physical and biogeochemical ocean are 3D (shoaling/weakening of the thermocline and the relation with oxygen for instance, (\cite{leungENSODrivesNearsurface2019}). Add more limitations? The Apecosm model addresses these different drawbacks. Complete advantages of Apecosm (mechanistic, 3D, advection/diffusion)

The aim of the present simulation is to analyse the impacts of ENSO-related climate variations on the fish biomass variability simulated by Apecosm. The paper is organized as follows. First, the El Nino indexes, the statistical methods and the model simulations are described in section 2. The synchronous response of epipelagic fish biomass to ENSO variability is investigated in section 3, while the delayed and low frequency response is addressed in section 4. Discussions and perspectives are finally provided in section 5.

%Improving our understanding of how climate variability impacts marine ecosystems and  how this will change in the future is a critical issue for the countries bordering the tropical Pacific and harvesting its marine resources. Marine ecosystems in the Western and Central Pacific Ocean (WCPO) are indeed supporting major oceanic fisheries that are responsible for 60% of the world tuna catches, nearly 3 million metric tons worth almost $7 billion each year and they substantially contribute to the revenue of most Pacific Island Countries and Territories. They also underpin a variety of small-scale artisanal fisheries that are critical for food security and livelihood in most tropical Pacific islands. Small-scale fleets operate in the coastal waters of the tropical Pacific States while industrial purse-seine, pole-and-line and long-line fleets operate both in their exclusive economic zones (EEZs) and in international waters. Highly migratory top-predators such as the tropical skipjack and yellowfin tunas (Katsuwonus pelamis and Thunnus albacares) constitute the bulk of the catch in the WCPO region. They are mostly harvested by industrial purse seine fisheries in the equatorial zone (10\degree{}N-10\degree{}S), where they inhabit and reproduce in warm surface waters above 26\degree{}C and feed opportunistically on a wide range of small epipelagic planktonic and nektonic prey. Smaller quantities of the tropical skipjack, yellowfin and bigeye tunas (Thunnus obesus) and the temperate albacore tuna (Thunnus alalunga) are also taken by industrial longliners in the WCPO up to 20\degree{}N for bigeye and 20\degree{}S for albacore. \\ 
%
%The spatial distribution of purse seine catches in the WCPO is strongly influenced by ENSO events, with catches typically expanding to the east during El Niño years and shifting back to the west during La Niña years (Williams and Terawasi 2014). 
%The ocean physics, biogeochemistry and biology are tightly coupled in the tropical Pacific, which makes it an ideal region to study biophysical interactions. This oceanic region is characterized by prevalent trade wind conditions, which accumulate a pool of warm water in the western equatorial Pacific and induce an upwelling of cold nutrient-rich subsurface waters that cools the entire central and eastern equatorial Pacific. The Eastern Tropical Pacific (ETP) upwelling of, leading to an oligotrophic west Pacific warm pool and a mesotrophic, upwelling cold tongue (e.g., Le Borgne et al. 2002). \warn{INCLUDE A BRIEF DESCRIPTION OF THE MAIN EPIPELAGIC FISH DISTRIBUTION IN THE TROPICAL PACIFIC.}
%
%The tropical Pacific is also home to the most energetic earth’s year-to-year climate variations, the El Niño/ Southern Oscillation (ENSO) (Cai et al., 2015). This phenomenon affects climate worldwide through atmospheric teleconnections (Taschetto et al., 2020), causing droughts and floods (Goddard and Gershunov 2020), modulating globally-averaged annual surface air temperature, tropical cyclone activity (Lin et al., 2020) and impacting agriculture and ecosystems  worldwide (Bertrand et al., 2020; Holbrook et al., 2020).  An El Niño event, the warm phase of ENSO, develops as the result of a positive ocean–atmosphere feedback that was first suggested by Bjerknes (1969). In this positive feedback loop, an anomalously warm sea surface temperature anomaly (SSTA) in the central/eastern Pacific promotes enhanced deep atmospheric convection and westerly wind anomalies in the central Pacific (Gill, 1980). This wind anomaly drives anomalous eastward surface currents that pushes the warm pool eastward and a deepening of the thermocline in the equatorial central/eastern Pacific, which both reinforce the initial warming. An El Niño event generally starts in late spring, peak at the end of the calendar year and recede during the following winter. La Niña (i.e. the cold ENSO phase) can broadly be viewed as a mirror image of El Niño despite some asymmetrical features: El Niño events can indeed occasionally reach much larger amplitudes than La Niña events, like in 1982, 1997 and 2015.  \\
%
%ENSO not only strongly impacts the physical properties of the tropical Pacific Ocean but also its biogeochemical properties. The dominant mode of global primary production anomalies as estimated from satellites is indeed directly related to ENSO (e.g. Behrenfeld et al., 2006; Chavez et al., 2010; Méssié and Chavez, 2012), with the strongest imprint in the tropical regions. These biological changes during ENSO are largely driven by changes in nutrient supply through vertical nutricline movements and horizontal advective processes (e.g. Barber and Chavez, 1983; Chavez et al. 1999; Wilson and Adamec, 2001; Christian et al., 2002; Ryan et al. 2002), other processes such as changes in stratification or light availability being of secondary importance. El Niño events are generally characterized by a productivity decrease in the eastern and central Pacific and an increase in the western Pacific (Chavez et al. 1999; Murtugudde et al. 1999; Gierach et al., 2012; Radenac et al., 2012). This productivity increase in the eastern equatorial Pacific have been related to a reduction of upward vertical nutrient flux in response to the equatorial thermocline/nutricline depth deepening observed during El Niño events, while the opposite stands in the western equatorial Pacific (e.g., Chavez et al., 1999; Wilson and Adamec, 2001; Radenac et al., 2012). In addition, the zonal eastward advection of nutrient‐poor warm pool waters by anomalous eastward currents further contributes to the decrease in biological productivity in the central Pacific (e.g. Chavez et al., 1999; Picaut et al., 2001). During La Nina events, enhanced equatorial upwelling in the central and eastern Pacific lifts deep nutrients into the photic zone, resulting in sometimes spectacular large-scale phytoplankton blooms throughout the cold tongue (e.g. Chavez et al., 1999; Ryan et al., 2002; Gorgues et al., 2010). ENSO not only affects productivity in the equatorial Pacific region but also along the south American west coast. Equatorial downwelling Kelvin waves during El Niño events propagate along this coast as coastally-trapped downwelling Kelvin waves,  inducing a deepening of the thermocline, nutricline and oxycline along the south American coast (Leung et al., 2019)  (Espinoza-Morriberón et al., 2019). \\
%
%These impacts of ENSO variability on the physical and biogeochemical properties of the tropical Pacific ultimately impacts the fish distribution, either through changes in habitat conditions (oxygen, temperature) and/or food availability (Bertrand et al. 2020). For instance, ENSO is known to be a major driver of location and related catches of four prominent tuna species in the Pacific (e.g. Bertrand et al. 2020; Lehodey et al., 2006, Nichols et al., 2014): skipjack (Lehodey et al., 1997; Kim et al., 2020), yellowfin, bigeye (Lu et al. 2001), and albacore (Lu et al. 1998; Kimura et al. 1997; Briand et al., 2011). ENSO directly influences tuna horizontal movements (Lehodey et al., 2020). For example, skipjack tuna catch shift eastward from the western to the central equatorial Pacific during  El  Niño events in response to the eastward migration of the convergence zone at the eastern edge of the warm pool (Lehodey et al., 1997, 2011), where important aggregating mechanisms occur for phytoplantkon (Picaut et al., 2001). El Niño events however generally exert an overall negative influence on relative abundance of skipjack tuna when averaged over the western and central Pacific (Yen et al. 2017). In contrast, the skipjack habitat retracts westward during La Niña events in response to warm pool retraction in the far western Pacific. The extent of the warm pool might also be a good indicator for monitoring the effect of environmental variability on yellowfin tuna recruitment (Kirby et al. 2007). Bigeye tuna has also been reported to extend eastward during El Niño years (Yukinawa, et al., 1988). Prolonged and recurring El Niño increases the effort of finding suitable fishing grounds and leads to decreased tuna harvest being landed in the Philippines (Vera and Hipolito, 2006). This also happened to Taiwan mackerel purse seine fishery, which experienced a sharp decrease in harvest by ~50% during the 1997/1998 El Niño (Sun et al., 2006). The strong temperature gradient of the thermocline exerts a strong control on the vertical tuna habitat (Brill et al., 1999; Schaefer et al., 2002) and its shoaling in the western Pacific during El Niño may squeeze their thermal and feeding vertical habitat, favouring their catchability (Lu et al. 1998; Lehodey et al., 2004). The opposite occurs during La Niña. 
%ENSO may not only affect tuna migration and vertical distribution but also the survival of tuna larvae and, therefore, the abundance of tuna populations. Unfortunately, there are no direct abundance surveys, such as the eggs and larvae sampling commonly used for coastal small pelagic stocks, to monitor such large‐scale variability of tropical tuna species larval densities. However, this variability propagates through the population structure and can be detected with some delay in the exploited stock, either through the analysis of catch rates and size frequencies of catch or as inferred from model and stock assessment analyses. High recruitments of skipjack were for instance reported two to four years following a bloom in the central Pacific, suggesting that La Niña-related high primary productivity is favorable to nursery and feeding conditions (Yen and Lu, 2016). El Niño has also been reported to negatively impact gonad maturation and mean length of skipjack tuna with time lags of 12 and 7 months, respectively (Kim et al. 2015). For long‐living species, such as bigeye and albacore, the decreasing growth rate with age and its natural variability over time and space leads to a cohort (age) signal more and more difficult to detect in larger fish. Therefore, the recruitment variability associated with ENSO, i.e. low or high peaks of abundance in the first cohort, is smoothed and damped while it combines with the internal dynamic processes of the species propagating toward the older cohorts (Lehodey et al., 2010; Sibert et al., 2012; Senina et al., 2017). Despite these issues, catches of albacore have been reported to decrease between 4 to 8 years after an El Niño onset south of 10\degree{}S, a time interval that would be expected before the recruitment for the fish spawned during ENSO episodes (Lu et al., 1998). This negative relationship with El Niño has been hypothesized to be related to the extension of warm waters in the central Pacific during El Niño events which reduces the extent of spawning grounds of south Pacific albacore (Lehodey et al. 2003). A similar relationship has also been found in the northern Pacific (Zhang et al. 2014), a stock productivity decrease lagging by four years El Niño events in this region.  Strong El Niño events shift the distribution of small and medium pelagic fishes (anchovy, sardine, mackerel, and jack mackerel) closer to the coast and, in some cases, move into deeper waters (Alheit and Niquen, 2004). They also generally lead to a reduction of anchovy biomass but actual impacts on post-event recovery differ considerably between events (Alheit and Niquen, 2004). \\ 
%
%\emph{
%Discussion on decadal signals: 
%Lehodey et al. (2003): -> In terms of total fluctuations in stock size and recruitment rates, skipjack and yellowfin tuna are positively influenced by the El Niño events and by the warm phases of the Pacific Decadal Oscillation. 
%Suarez-Sanchez et al. (2004): -> Low-frequency yellow-fin tuna variability in the northeastern Pacific over the 1967-1993 period in apparent relation with IPO (but not discussed in the text which only mention the influence of strong EN)
%Pedraza and Diaz-Ochoa, 2006:
%Lima and Naya (2011): -> In addition to ENSO, one‐year lagged negative PDO effects on annual tuna fluctuations in the western tropical Pacific. The negative effects of PDO on skipjack tuna could be associated with the observation that during the warm phases of PDO (i.e. positive PDO values) there is a reduction in the equatorial upwelling and a rise of the sea surface temperatures at the equatorial Pacific Ocean (McPhaden and Zhang 2002). Therefore, this reduction in the ocean productivity could be influencing negatively the recruitment of skipjack tuna that are perceived the next year when the individuals recruit to the fishery. It seems that changes in sea surface temperature and ecosystem processes within the equatorial Pacific Ocean are influenced by the decadal oscillation within the North Pacific Ocean (Linsley et al. 2000, McPhaden and Zhang 2002). In sum, we determined one‐year lagged effects of SOI and PDO that can be related with ecological effects on recruitment, whereas the direct PDO effects could be caused by the effect of oceanographic conditions on tuna catchability. 
%Di Lorenzo and Ohman (2013): Theoretical modeling suggests that the cumulative integrations of white‐noise (high‐frequency) atmospheric forcing can generate red‐noise (low‐frequency) responses in oceanographic variables (as for the PDO) and thus generate marine population responses that are characterized by different regimes and strong transitions.
%Philips et al. (2014): Results indicate that SST had a positive and spatially variable effect on albacore CPUE, with increasingly positive effects to the North, while PDO had an overall negative effect. Although albacore CPUE increased with SST both before and after a threshold year of 1986, such effect geographically shifted north after 1986. This is the first study to demonstrate the non-stationary spatial dynamics of albacore tuna, linked with a major shift of the North Pacific. 
%Nicol et al. (2014): El Niño or La Niña events during multi-year periods, possibly in correlation with the PDO, could lead to regimes of high and low productivity in the tuna population (Kirby et al. 2004; Lehodey et al. 2003; Lehodey et al. 2006). However, particularly strong shifts in the environment were not always detected in tuna recruitment time series (Briand and Kirby 2006), which implies that the relationship between tuna recruitment and climatic oscillations is not linear and might depend on several interrelated factors including the adaptation of spawners to environmental variability.
%Singh et al. (2015): Significant links were established between PDO and albacore time series trajectory from 1957 to 2008 in the South Pacific Ocean.
%Singh et al. (2017): Lehodey et al. (2003) attempted to deduce the mechanisms by which alterations in environmental variables affect the stock of important tuna species in the Pacific. Results indicated that albacore tuna recruitment was significantly affected by the negative and positive phases of PDO. PDO had highly significant correlation with R in the same year and with SSB having a lag period of 5 years. This indicates that variability in the PDO pattern influences the early life stages of the North Pacific albacore tuna. With reference to Figure 3, the relationship of PDO with R and SSB is negative. This means that the negative PDO phase is favorable and positive PDO phase is unfavorable for the larvae and juvenile stages of albacore. Similar explanations have been derived for the South Pacific albacore stock by Lehodey et al. (2003) and Singh et al. (2015).
%Wu et al. 2020: The standardized CPUE in the western Pacific Ocean was significantly correlated to the Pacific Decadal Oscillation (PDO) with a 1–5 year lag. 
%Lehodey et al. (2020): given that all tuna species spawn in tropical warm waters under the influence of ENSO, it can be assumed that their spawning habitats and the subsequent fish recruitment are all impacted by ENSO variability, as has been demonstrated for skipjack tuna (cf section above). The result can be a delayed fluctuation in abundance of juvenile and adult tuna moving to the NWP. This also seems to be the case for the Japanese eels (Anguilla japonica) that spawn in the subtropical NWP, where the intensity of the flow and position of the bifurcation  of the North Equatorial Current change with ENSO and therefore impact larval transport and the recruitment of juveniles in the NWP (Hsiung et al., 2018).\\
%}
%
%Several ecosystem models have been developed to better characterize and understand ecosystems response to climate variability and change. Models gathered in the Fisheries and Marine Ecosystem Model Intercomparison Project (Fish-MIP) have been used to analyse changes in fish biomass induced by climate change projected by climate simulations gathered in the Climate Model Intercomparison Project (CMIP). This project pointed to a global decline of marine biomass, mainly driven by an increase of the mean temperature and by a decrease in the primary production  (Lotze et al., 2019). This biomass change is however spatially inhomogeneous, with a decrease at middle to low latitudes, and an increase at higher latitudes. The biomass response is also more pronounced for higher trophic levels.\\
%
%While these models are now routinely used to project future biomass changes, few studies did attempt to understand how these ecosystem models respond to past climate variability. This is however a must because (1) an accurate representation of past fish biomass variations would allow a more confident assessment of future projections and (2) a better understanding of the processes response for past variability would provide some guidance to understand future changes. A tuna spatial population model (SEAPODYM, Lehodey 2004) has for instance been shown to successfully reproduce the responses of the population dynamics to changes in their physical and biological habitat (Lehodey et al. 2003). Input dataset describing the oceanic environment includes temperature, currents and oxygen concentration averaged over three integrated layers: 0–200 m (epipelagic), 200–500 m (mesopelagic) and 500–1000 m (bathypelagic); and also primary production integrated over all the vertical layers.  This model indicates that the eastward extension of the species and fisheries distributions during El Nino phases are driven by changes in temperature, prey, and dissolved oxygen concentration (Lehodey et al. 2020), and conversely during La Nina conditions (westward contraction) . This model also suggests that ENSO not only drives large-scale displacement of tuna population but can also impact their recruitment and abundance statistics (Lehodey, 2001, Lehodey et al., 2003). For instance, SEAPODYM results also suggest an increasing skipjack and yellowfin tuna recruitment in the central and the western Pacific during El Niño events (Lehodey et al., 2003, 2006; Senina et al., 2008, Langley et al. 2009) that might be a result of four mechanisms: (1) the eastward extension of the warmpool, favouring spawning of these two species, (2) enhanced food for tuna larvae due to higher primary production in the west, (3) lower predation of tuna larvae and (4) larvae retention in these favourable areas as a result of ocean currents.  Similar positive effects of El Ninos on early life stages were detected with SEAPODYM in bigeye and yellowfin tuna species, mainly in the eastern Pacific Ocean. Favorable conditions for larvae survival increase during El Nino events in the eastern Pacific Ocean and decrease in the central region. These species with longer life spans are also more susceptible to present decadal regimes of high and low productivity due to the accumulation of successive low or high peaks of recruitment driven by the decadal modulation of ENSO. A dominance of either El Nino or La Nina events is observed over multiyear periods, possibly in correlation with the Pacific‐scale Interdecadal Pacific Oscillation (IPO).\\
%		
%\warn{Continue modelling studies}
%
%The modeling studies described in the above, however, have some drawbacks. First, all the models used simulate the marine ecosystem in 2D, although the impacts of El Nino events on the physical and biogeochemical ocean are 3D (shoaling/weakening of the thermocline and the relation with oxygen for instance, (Leung et al., 2019)). Add more limitations?
%
%The Apecosm model addresses these different drawbacks. Complete advantages of Apecosm (mechanistic, 3D, advection/diffusion)
%
%The aim of the present simulation is to analyse the impacts of ENSO on the fish biomass variability simulated by the Apecosm model when it is forced by a hindcast physical biogeochemical simulation that covers the 1958-2018 period. The paper is organized as follows. The physical-biogeochemical and ecosystem models used in this study are presented in section 2. The sea-surface temperature and chlorophyll simulated by the coupled physical biogeochemical model will be confronted to observation-based datasets, in order to ensure that it properly simulates the ocean response to ENSO variability (section 3). Then, the response of epipelagic fish biomass to ENSO variability will be investigated (section 4). Discussions and perspectives are provided in section 5.
