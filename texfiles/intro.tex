% !TeX root = ../article-enso.tex

\section{Introduction}

%\textbf{Societal relevance.} 
Understanding the impact of climate variability and change on marine ecosystems is key for the countries that border the tropical Pacific and exploit its marine resources. The marine ecosystems in the tropical Pacific Ocean indeed support a variety of small-scale artisanal fisheries that are essential for food security and livelihoods of most tropical Pacific islands and riparian countries \citep{batistaTropicalArtisanalCoastal2014}. They also support domestic and Distant Water Fishing Nations (DWFN) large-scale oceanic fleets that are responsible for 60\% of the world's tuna catches and  contribute substantially to the income of most Pacific Island Countries and Territories, through domestic production and the purchase of fishing rights (just in the Western and Central Pacific, the value of the total tuna catch has consistently fluctuated between 4.5 and 7.5 billion dollars since 2007, \citealt{williamsOverviewTunaFisheries2021}). Skipjack (\textit{Katsuwonus pelamis}), yellowfin (\textit{Thunnus albacares}) and young bigeye (\textit{Thunnus obesus}) tunas make up the bulk of purse seine catches that dominate tropical tuna fisheries \citep{allainOverviewTunaFisheries2018}. Their catches generally occur in the warm (above 26°C) surface waters of the western and the eastern Pacific where they live, reproduce and feed opportunistically on a wide range of small planktonic and nektonic epipelagic prey. Indeed, the prevailing trade wind conditions in the tropical Pacific leads to the accumulation of warm waters in the western Pacific that are favorable to tropical tuna. These winds also cause an upwelling of cold and rich waters along the equator throughout the central and eastern equatorial Pacific and induce the accumulation of epipelagic tuna prey that are part of trophic chains resulting from the equatorial upwelling. Smaller quantities of yellowfin and bigeye tuna as well as the temperate albacore tuna (\textit{Thunnus alalunga}) are also caught by industrial longliners in sub-equatorial and sub-tropical regions \citep{allainOverviewTunaFisheries2018}.

%\textbf{ENSO physical and biogeochemical response.} 
The climatological distribution of tropical tuna is strongly altered by the \nino{}/Southern Oscillation (ENSO), the Earth’s most energetic year-to-year climate event \citep{williamsOverviewTunaFisheries2014, caiChangingNinoSouthern2021}. ENSO indeed has a significant impact on the physical and biogeochemical properties of the tropical Pacific Ocean. 
An \nino{} event (i.e. the warm phase of ENSO) is characterized by a  deepening of the thermocline and nutricline in the central and eastern Pacific, which causes a warming of sea surface temperatures and a reduction of primary production in these regions, via a reduction of the upward vertical flux of nutrients and cold waters (e.g. \citealp{chavezBiologicalChemicalResponse1999, murtuguddeOceanColorVariability1999}). In contrast, the western Pacific Ocean is experiencing opposite changes with a shoaling of the thermocline and nutricline, resulting in a slight cooling. Zonal eastward advection of warm nutrient‐poor waters by anomalous eastward currents also contributes to the decrease of biological productivity in the central Pacific (e.g. \citealp{chavezBiologicalChemicalResponse1999, picautOceanicZoneConvergence2001}). \nina{} (i.e. the cold ENSO phase) are generally considered as a mirror image of \nino{}, despite some asymmetric features.  

%\textbf{Synchronous ecological response to ENSO.}
These changes in the physical and biogeochemical characteristics of the tropical Pacific Ocean during ENSO ultimately affect high trophic level organisms, including exploited fish populations, through changes in habitat conditions (oxygen, temperature, light penetration), currents and food abundance and availability \citep{bertrandNinoSouthernOscillation2020}.  Tuna fisheries data indicate that purse seine catches in the western Pacific generally move eastward during \nino{} events and retract  westward during \nina{} events, in conjunction with the zonal migration of the warm pool \citep{lehodeyNinoSouthernOscillation1997}. The strength of the vertical temperature gradients at the thermocline level also exerts a strong control on the vertical distribution of tunas \citep[e.g.][]{schaeferMovementsBehaviorHabitat2002}. It vertically compresses their thermal and feeding habitat in the western Pacific during \nino{}, which increases the formations of dense schools \citep{mauryCanSchoolingRegulate2017} thus promoting their catchability by purse seine fisheries \citep{bertrandHydrologicalTrophicCharacteristics2002}. ENSO not only impacts ecosystems through horizontal and vertical movements of fish populations, but can also affect the survival of larvae, whose variability propagates through the population structure and may be eventually be detected in the adult population some time later \citep{yenSpatialTemporalVariations2016, kimEffectsClimateinducedVariation2015}. 

%\textbf{Ecosystem modelling.} 
Most of the observational studies analyzing the influence of ENSO on Pacific marine ecosystems rely on tuna catch data, the variation of which is not only controlled by climate variability effects on population abundance and distribution but also by changes in fishing effort distribution, catchability and various dynamic processes internal to the ecosystem \citep{hobdayDetectingClimateImpacts2013}. In addition, these fisheries observations are heterogeneous, limited to narrow and varying gear-specific depth ranges (for instance the 0-150m surface layer for purse seine data), focused on a few species and small size ranges, so that potential climate signals in these data are likely to be biased and distorted by other factors \citep{hobdayDetectingClimateImpacts2013}. 

In complement to using fisheries observations, several ecosystem models have been developed as part of the Fisheries and Marine Ecosystem Model Intercomparison Project (Fish-MIP, \citealp{tittensorProtocolIntercomparisonMarine2018}) to characterize and understand marine ecosystem responses to climate fluctuations. These models have been used primarily to project biomass changes in response to global warming, generally pointing to a global decline of marine biomass, more pronounced for higher trophic levels and tropical waters \citep{lotzeGlobalEnsembleProjections2019, tittensorNextgenerationEnsembleProjections2021}. While these models are now commonly used to project future  changes in biomass, they are much less used to analyze their response to past climate variability. This is however necessary because (1) a reliable representation of past variations in fish biomass would improve confidence in their future projections and (2) a better understanding of the processes responsible for past variability would provide keys to improving the models and better understanding of future changes. To our knowledge, only the SEAPODYM \citep{lehodeySpatialEcosystemPopulations2008} ecosystem model has been specifically used to assess the ecosystem response to ENSO in the tropical Pacific, focusing on the spatial dynamics of the skipjack population \citep{lehodeyPelagicEcosystemTropical2001}. This model is able to reproduce the large-scale zonal migration of the skipjack tuna population in the equatorial Pacific in response to ENSO, which they attribute to ENSO-related changes in temperature, prey and oxygen concentrations that are driving active movements of skipjack tuna. Analysis of this model also suggests that \nino{} not only drives an eastward tuna displacement but also promotes  strong larval recruitment \citep{seninaParameterEstimationBasinscale2008}. 

However, most ecosystem models have certain limitations that may restrict their ability to capture the full complexity of ENSO's impact on ecosystems. In particular, they generally simulate the marine ecosystem in two dimensions, despite the  inherently three-dimensional nature of the impacts of \nino{} events on the physical and biogeochemical oceanic properties  (shoaling/weakening of the thermocline and the relation with oxygen for instance; \citealp{leungENSODrivesNearsurface2019}). They also generally do not consider the effect of passive transport by ocean currents or active movements along environmental gradients and when they do, this transport is applied  to only a limited number of size or age classes. Furthermore, they rarely simultaneously include  the bottom-up and top-down effects of predation as well as the various metabolic processes (growth, reproduction, development, maintenance, mortality) that contribute to the transfer and dissipation of energy along food chains and cause temporal changes characteristic of environmental variability. 

The objective of this paper is to revisit the question of ENSO impacts on tropical Pacific Ocean ecosystems using the mechanistic ecosystem model APECOSM (\citealp{mauryOverviewAPECOSMSpatialized2010}), which doesn't suffer from the main limitations highlighted above (2D models, no passive or volational movements for instance) and which considers explicitly the associated bio-ecological complexity. We focus our analysis on understanding the different bio-ecological processes by which ENSO influences the epipelagic community, which is  the most intensively exploited pelagic community, especially by industrial purse seine fisheries targeting skipjack and yellowfin tunas. Overall, we show that the role of passive transport through \nino{} related surface current anomalies is critical, not only for small organisms as usually assumed, but also for medium and large organisms. Furthermore, while passive transport effects dominate biomass changes for large organisms, we show that they can be amplified or offset for medium and small organisms by the interplay of bio-ecological processes such as temperature effects on growth, foraging success, and predatory mortality, in ways that differ in the western and central Pacific. Contrary to what is often assumed (e.g. \citealt{lehodeyPelagicEcosystemTropical2001, lehodeyENSOImpactMarine2020}), our model shows that active habitat-based movements are not required to explain the westward biomass shifts that are observed during ENSO.

The document is organized as follows. Section \ref{sec:model-des}) first describes the physical, biogeochemical and ecosystem models used in this study. Section \ref{sec:model-val} then assesses the ability of these models to reproduce the response to ENSO variability by comparing them to observations. Section \ref{sec:nino-epi} then investigates the dynamic and biological processes responsible for the modeled response of epipelagic fish biomass to \nino{} events as a function of the size class. Finally, section \ref{sec:conclusion} concludes this study by highlighting the main results, limitations and perspectives of this work. 
