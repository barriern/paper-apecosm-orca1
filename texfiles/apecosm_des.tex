% !TeX root = ../article-enso.tex

\subsection{Marine ecosystem model}
\label{sec:apecosm}

We use the Apex Predators Ecosystem Model (APECOSM, \citealp{mauryModelingEnvironmentalEffects2007, mauryOverviewAPECOSMSpatialized2010}) to simulate the energy transfer through marine ecosystems. 
APECOSM is a eulerian ecosystem model that represents the three-dimensional dynamics of size-structured pelagic populations and communities mechanistically. It integrates individual, population and community levels and includes the effects of life-history diversity with a trait-based approach \citep{mauryIndividualsPopulationsCommunities2013}. In APECOSM, energy uptake and utilization for individual growth, development, reproduction, somatic and maturity maintenance are modeled according to the Dynamic Energy Budget (DEB) theory \citep{koojmanDynamicEnergyBudget2010}. The model considers important ecological processes such as opportunistic size-structured trophic interactions and competition for food, predatory, disease, ageing and starvation mortality, key physiological aspects such as vision and respiration, as well as essential processes such as three-dimensional passive transport by marine currents and active habitat-based movements \citep{faugerasAdvectiondiffusionreactionSizestructuredFish2005}, schooling and swarming (see \citealp{mauryModelingEnvironmentalEffects2007, mauryIndividualsPopulationsCommunities2013, mauryCanSchoolingRegulate2017} for a detailed description of the model). 
%APECOSM is driven by the 3D fields of temperature, horizontal currents, large phytoplankton, small and large zooplankton, big particulate organic matter, light and oxygen provided by the NEMO-PISCES simulation (section \ref{sec:nemo}), which constrain the biological and ecological dynamics at various levels.
 All the metabolic rates are temperature-dependent and corrected by an Arrhenius factor \citep{mauryModelingEnvironmentalEffects2007, mauryIndividualsPopulationsCommunities2013}. While it can be prescribed in the model code, we do not use a limited temperature range to any of the simulated communities in the model configuration used here.

In APECOSM, the dynamics of communities is determined by integrating the core state equation below:

\begin{equation}
\partial_t \varepsilon = \underbrace{- \partial_w(\gamma \varepsilon) + \frac{\gamma}{w}\varepsilon}_{Growth} 
\underbrace{- M \varepsilon \vphantom{\frac{\gamma}{w}\varepsilon}}_{Mortalities}
\underbrace{-\overrightarrow{\nabla}.(\overrightarrow{V} \varepsilon) \vphantom{\frac{\gamma}{w}\varepsilon}}_{3D Adv} 
\underbrace{+ \overrightarrow{\nabla} . (D \overrightarrow{\nabla} \varepsilon) \vphantom{\frac{\gamma}{w}\varepsilon}}_{3D Diff.}
\label{eq:apecosm_trend}
\end{equation}

where $\varepsilon$  is the organisms' biomass density in the community, $w$ their individual weight, $\gamma$ is the growth rate, $M$ represents the different mortality rates (computed using equation 12 of \citealt{mauryIndividualsPopulationsCommunities2013}), $V$ and $D$ the sum of 3D passive and active velocities and diffusivity coefficients (computed following \citealt{faugerasAdvectiondiffusionreactionSizestructuredFish2005}). The growth contribution is made of an advection (i.e. the biomass transfer along the size-spectrum, left-hand side) and a source term (i.e. biomass creation, right-hand side). Reproduction is considered through a Dirichlet boundary condition that injects the reproductive outputs from all mature organisms in $w_0$.

In APECOSM, the energy ingested by organisms fuels individual metabolism according to the DEB theory. Ingestion is proportional to a functional Holing type II response function that depends on the size-dependent visibility of prey, their aggregation in schools and temperature. This functional response can be written in a simplified way as follows:

\begin{equation}
f_{w} = \frac{P_{w}}{C_{w} + P_{w}}
\label{eq:repfonct}
\end{equation}

with $P_{w}$ the prey biomass that is available to predator of size $w$ and $C_{w}$ the half-saturation constant.
%, which depends on weight, light and temperature.

Tn the APECOSM model, oxygen concentration only modifies the horizontal and vertical habitat of the different communities and size-classes and do not modify, in its current state, the  biological parameters or the physiological rates. Considering the region of interest of the given study, this limitation has barely no consequence. Which would not be the case if analysing outputs within an Oxygen Minimum Zone (OMZ). 

% \begin{displaymath}
% p_{c, u} = \sum_{k=1}^{N_{com}} \left[\int_{v=w_{egg}}^{w_{max}} P^{sch}_{k, v}\ s_{u, v}\ \xi_{k, v}\ dv\right] + \sum_{p=1}^{N_{ltl}} \left[ P^{sch}_{p}\ s_{u, p}\ \xi_{p}\right]
% \end{displaymath}

% \begin{displaymath}
% C_{c, w} = \frac{C_{REF}}{T_{ahr}\times h_c^{light}\times h_c^{tlim}\times W_{w}^{\chi}}
% \end{displaymath}

The APECOSM simulation used in this study is forced by three-dimensional temperature, horizontal current velocities, dissolved oxygen concentration, diatoms, mesozooplankton, microzooplankton and big particulate organic matter carbon concentrations \citep{aumontPISCESv2OceanBiogeochemical2015}, photosynthetically active radiation (PAR) and dynamical layer thickness outputs from the NEMO-PISCES simulation (section \ref{sec:nemo}). Nutrients concentrations simulated by NEMO/PISCES are not used as a forcing to Apecosm.

The APECOSM simulation runs with a daily time step for the biological processes, which is decomposed into a day/night cycle, the duration of which  depends on latitude and day of the year \citep{forsytheModelComparisonDaylength1995}. A sub time-stepping ($dt =0.8h$) is used for horizontal advection and diffusion to ensure numerical stability.

In APECOSM, the depth dimension is explicit, i.e. each biological variable (mortality, functional response) is computed in 3 dimensions (depth, latitude, longitude). The vertical distribution is thus determined from habitat functions that depend on the choice of the communities. In this study, three interactive communities are simulated:
\begin{itemize}
\item{The epipelagic community, which includes the organisms that are feeding during the day near the surface such as yellowfin or skipjack tunas for example. Its vertical distribution is influenced by light and visible food during daytime as well as temperature and oxygen during both day and night, while its functional response is influenced by light and temperature.}
\item{The migratory mesopelagic community, which feeds in the surface layer at night and migrates to deeper waters during the day. Its vertical distribution is influenced by light and visible food during the night.}
\item{The resident mesopelagic community, which remains at depth during both night and day. Its vertical distribution is influenced by light and visible food during the day.}
\end{itemize}

To ensure that the size-spectrum is fully unfolded and a pseudo-steady state is achieved, the model was integrated successively over three 1958-2018 cycles. It was first initialized with an arbitrary small biomass value in each size-class and community and integrated from 1958 to 2018 (61 years). Then, the end of this first integration phase was used to run another cycle, which in turn was used to initialize the simulation analyzed in this study.

For each community, equation \ref{eq:apecosm_trend} is integrated over 100 logarithmically distributed size classes, ranging from $0.123cm$ to $196cm$. Since saving the outputs in 3D for the 3 communities and 100 size-classes is very costly, mortality rate, growth rate and functional response for each community and size are vertically averaged as follows:

\begin{equation}
F(y,x,c,w) = \frac{\sum_{z=0}^{H} F(z, y, x, c, w) B(z, y, x, c, w)}{\sum_{z=0}^{H}B(z, y, x, c, w	)}
\end{equation}

with $x$ the longitude, $y$ the latitude, $z$ the depth, $c$ the community, $w$ the size-class, $F$ the variable to consider (functional response, mortality rate, growth rate) and $B$ the 3D biomass (in $J.m^{-3}$).

In what follows, the focus is solely put on the response of the epipelagic community; its near-surface location makes it more sensitive to ENSO variability \citep{lemezoNaturalVariabilityMarine2016}, it corresponds to organisms such as skipjack and yellowfin that are targeted by the industrial purse seine fleet, it accounts for the majority of tuna catches in the region, and have been reported to respond markedly to ENSO \citep{lehodeyNinoSouthernOscillation1997}.
