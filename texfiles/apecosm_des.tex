The marine ecosystem model used in this study is the Apex Predators Ecosystem Model (Apecosm, \citealt{mauryModelingEnvironmentalEffects2007, mauryOverviewAPECOSMSpatialized2010}), which simulates the energy transfer in marine ecosystems. The biological processes include size-based opportunistic trophic interactions, competition for food, allocation of energy between growth and reproduction, somatic and maturity maintenance, predatory and starvation mortality (see \citealt{mauryModelingEnvironmentalEffects2007} for a detailed description of the model). The physiological bases of the model are derived from the Dynamic Energy Budget theory (DEB, \citealt{kooijmanDynamicEnergyMass2000}). All the physiological rates are temperature-dependent. In addition to biological processes, energy density is also subjected to both passive (via ocean currents) and active (foraging behaviour) advection/diffusion, following \cite{faugerasAdvectiondiffusionreactionSizestructuredFish2005}.

Changes in fish biomass is given by:

\begin{equation}
\partial_t \varepsilon = \underbrace{- \partial_w(\gamma \varepsilon) + \frac{\gamma}{w}\varepsilon}_{Growth} 
\underbrace{- M \varepsilon \vphantom{\frac{\gamma}{w}\varepsilon}}_{Predation}
\underbrace{-\overrightarrow{\nabla}.(\overrightarrow{V} \varepsilon) \vphantom{\frac{\gamma}{w}\varepsilon}}_{Adv} 
\underbrace{+ \overrightarrow{\nabla} . (D \overrightarrow{\nabla} \varepsilon) \vphantom{\frac{\gamma}{w}\varepsilon}}_{Diff.}
\label{eq:apecosm_trend}
\end{equation}

where $\varepsilon$  is the fish biomass density, $w$ the weight, $\gamma$ is the growth rate, $M$ is the predation mortality rate (computed using equation 12 of \citealt{mauryIndividualsPopulationsCommunities2013}), $V$ and $D$ the sum of passive and active velocity and diffusivity coefficient (computed following \citealt{faugerasAdvectiondiffusionreactionSizestructuredFish2005}).

The biological simulation used in this study is forced by the temperature, current velocity, oxygen, low-trophic level concentration (diatoms, mesozooplankton and microzooplankton, big particulate organic matter), photosynthetically active radiation (PAR) and layer thickness outputs of the NEMO-PISCES simulation (section \ref{sec:nemo}). It uses a daily time-step for biological processes, which is decomposed into a day/night cycle, whose duration depends on the latitude and on the day of year \citep{forsytheModelComparisonDaylength1995}. A sub time-stepping ($dt =0.8h$) is used for advective and diffusive processes in order to insure numerical stability.

%In order to assess the mechanisms of fish biomass response to ENSO variability, biomass changes induced by growth, predation, advection and diffusion are stored for the entire simulation. Therefore, the biomass for a given size class, at a given location and for a specific time can be reconstructed by using the following equation:
%
%\begin{equation}
%\varepsilon(T) = \varepsilon(T=0) + \int_{t=0}^{T} \left[ 
%T_{pred}(t) +
%T_{growth}(t) + 
%T_{adv}(t) + 
%T_{diff}(t) 
%\right] dt 
%\label{eq:rec_oope}
%\end{equation}
%
%with $\varepsilon(x,y,s,T=0)$ the fish biomass at the beginning of the simulation and $T_{pred}$, $T_{growth}$, $T_{adv}$ and $T_{diff}$ the respective biomass increments due to predation, growth, advection and diffusion, given by equations \ref{eq:pred}, \ref{eq:growth} and \ref{eq:move}, respectively.

Three generic communities are simulated:
\begin{itemize}
\item{The epipelagic community, which remains close to the surface and whose vertical distribution is influenced by temperature and oxygen, and whose functional response is influenced by light.}
\item{The migrant community, which is influenced by light through vertical habitat and functional response and whose distribution varies between day and night (at depth during daytime and close to the surface at night).}
\item{The  mesopolagic community, which remains at depth during both night and day, and whose vertical distribution is only impacted by the light.}
\end{itemize}

For each community, 100 size classes are simulated, ranging from $0.123cm$ to $196cm$.

%Epilagic feed on other epilagic fish only during night daytime. Migrants feed on other migrants and epipelagics, only during night-time. While mesopelagics feed on migrant and mesopelagic during daytime, and only on other mesopelagic during night-time.

In order to ensure that the size-spectrum is fully unfolded, the model has been integrated three times. First, the model has been initialized from rest and integrated from 1958 to 2018. Then, the end of this first spin-up phase has been used to run another cycle, which was in turn used to run the simulation presented in the following.

%In the following, the focus will be put on the epipelagic community, since its near-surface location makes it more sensitive to ENSO variability \citep{lemezoNaturalVariabilityMarine2016}2016}.