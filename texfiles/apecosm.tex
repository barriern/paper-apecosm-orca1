\section{Interannual response to ENSO variability}

In this section, the interannual response of epipelagics to ENSO variability is investigated using covariance analysis, as done in section \ref{sec:sst}. 
The focus will be first laid on vertically integrated 
fish biomass, in order to have a spatial vue of the response. Then, the response of the fish biomass as a function of longitude and depth will be investigated.\\

For the sake of simplicity, the focus will be laid on three size classes: 2cm, representing small fishes, 20cm, representing intermediate sizes, and 90 cm, representing large individuals. These sizes are also
representative of the sizes of tuna target species within the region.\\

\subsection{Vertically integrated biomass}

The left panels of figure \ref{fig:mean-cov-ape} show the yearly mean vertically integrated fish biomass ($J.m^{-2}$) over the entire simulation for the three different size classes. For small sizes (figure \ref{fig:mean-cov-ape}a), the fish biomass is concentrated at around 10\degree S and 10 \degree N in the central Pacific and close to 0the equator in the western Pacific. High biomass concentration is also found east of 90\degree W, off the coasts of Chili. As size increases, the equatorial "blue spot" extends to the south and to the west. \\

\begin{figure}[h!]
    \centering
    \includegraphics[width=\textwidth]{covariance_mean_epi.png}
    \caption{Yearly average of biomass density (left panel, log-scale, $J.m^{-2}$) and 
    covariance between the yearly average biomass density and the winter-average ONI index (right panels, $J.m^{-2}$)}
    \label{fig:mean-cov-ape}
\end{figure}

Covariance maps between vertically integrated fish biomass and the winter ONI index are shown in the right panels of figure \ref{fig:mean-cov-ape}. 
Small epipelagics show negative anomalies in the Western Equatorial Pacific and positive anomalies in the Central Equatorial Pacific. This pattern can be interpreted as an eastern displacement of the mean biomass in the Western Pacific.\\

Similar dipolar patterns are also obtained for intermediate and large sizes, but the anomalies westward shifted as size increases. This can be interpreted, as for small sizes, by a westward shift of fish biomass during positive \nino\ phases.\\

\subsection{Equatorial biomass profile}

The yearly mean equatorial profile of fish biomass density, obtained by averaging the 3D biomass density between 2\degree S and 2\degree N, 
is shown figure \ref{fig:mean-forage}, for day (left panels) and night (right panels). Small size classes (0-3cm and 3cm-20cm) show similar 
patterns during both nights and days. In the east, the fish biomass is concentrated at the surface, betweem 0 and 25m. In the west, small epipelagics
are found down to 75m. 

For larger sizes (20cm-90cm and 90cm-200cm),

\begin{figure}[h!]
    \centering	
    \includegraphics[width=\textwidth]{forage_mean.png}
	\caption{Mean epipelagic biomass density integrated over 2\degree S and 2\degree N}
    \label{fig:mean-forage}
\end{figure}

\begin{figure}[h!]
    \centering
    \includegraphics[width=\textwidth]{forage_covariance.png}
	\caption{Covariance between yearly anomalies of epipelagic biomass density integrated over 2\degree S and 2\degree N and the winter ONI index.}
    \label{fig:covariance-forage}
\end{figure}

The corresponding covariance is shown in figure \ref{fig:covariance-forage}.

%\subsection{Interannual response to ENSO variability}
%
%\subsection{Transient biological response to \nino}
%
%\begin{figure}[h!]
%    \centering
%    \includegraphics[scale=0.5] {debugged_corr_mask_hovmoller_composites_OOPE}
%    \caption{Composites of \hov\ diagram composite of fish biomass density ($J.m^{-2}$) along the equatorial Pacific ocean.}
%    \label{fig:hov-oope}
%\end{figure}

\clearpage
