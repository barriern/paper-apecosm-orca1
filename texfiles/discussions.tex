\section{Discussions}

In the present paper, we analysed the response of fish biomass to changes in the physical and biogeochemical ocean in response to ENSO variability. \\

In the present study, the Apecosm model was forced by the outputs of the NEMO/Pisces physical and biogeochemical model, with no retroaction of the former on the latter. One question that arises is whether the response of fish biomass and ocean biogeochemistry to ENSO variability would be the same in a two-way coupled simulation. 	Such coupling, which has already been used in \cite{aumontEvaluatingPotentialImpacts2018} to analyse the impacts of the diurnal vertical migration on ocean biogeochemistry, could be used to determine whether the inclusion of fish biomass modifies the response of ocean biogeochemistry to ENSO. And whether the coupling attenuates or strenghthens the response of fish biomass. \warn{Improves the text}. \\

The study relies on a global simulation at a relatively coarse horizontal resolution (around 1\degree), which is insufficient to 
explicitly represents the mesoscale features of the California Current System \citep{capetMesoscaleSubmesoscaleTransition2008} and
of the Peru-Chile current system \citep{colasHeatBalanceEddies2012}. This may lead to a misrepresentation of temperature and biogeochemical variables in the upwelling systems, which will ultimately propagates to the higher trophic levels. A similar study
could be reproduced using a regional model of the upwelling systems, using the simulation presented in this paper to force open boundary conditions of a regional configuration. 

List of points that can be added to the discussion:

\begin{itemize}
    \item Coupling NEMO-Pisces and impacts?
    \item Regional simulations to better represents physical processes?
    \item Focus species (tuna), more suited for tropical pacific
    \item ENSO impacts in the other regions (cf. Racault)
    \item Separation of CP/EP Ninos
\end{itemize}

In this study, the transient response of marine ecosystems has been investigated by using covariance and composite analysis using a single \nino\ index.
Therefore, no distinction has been made between the Eastern Pacific \nino\ and the Central Pacific \nino\ events. However, it is now well established that these two types of events 
have different physical and biogeochemical signatures and causes. For instance, \cite{gierachBiologicalResponse19972012} have shown, using satellite observations and adjoint passive tracer simulations, 
that \epn\ causes a reduction chl-a in the Eastern Pacific, mainly due to weaker trade winds, reduced upwelling and vertical mixing, while \cpn\ reduces chl-a in the Central Pacific, mainly through a stronger
eastward advection of nutrient-depleted waters from the Western Pacific.

\warn{\cite{racaultImpactNinoVariability2017} for impact on plankton}

