\section{Discussions}

List of points that can be added to the discussion:

\begin{itemize}
    \item Coupling NEMO-Pisces and impacts?
    \item Regional simulations to better represents physical processes?
    \item Focus species (tuna), more suited for tropical pacific
    \item ENSO impacts in the other regions (cf. Racault)
    \item Separation of CP/EP Ninos
\end{itemize}

In this study, the transient response of marine ecosystems has been investigated by using covariance and composite analysis using a single \nino\ index.
Therefore, no distinction has been made between the Eastern Pacific \nino\ and the Central Pacific \nino\ events. However, it is now well established that these two types of events 
have different physical and biogeochemical signatures and causes. For instance, \cite{gierachBiologicalResponse19972012} have shown, using satellite observations and adjoint passive tracer simulations, 
that \epn\ causes a reduction chl-a in the Eastern Pacific, mainly due to weaker trade winds, reduced upwelling and vertical mixing, while \cpn\ reduces chl-a in the Central Pacific, mainly through a stronger
eastward advection of nutrient-depleted waters from the Western Pacific.

\warn{\cite{racaultImpactNinoVariability2017} for impact on plankton}

