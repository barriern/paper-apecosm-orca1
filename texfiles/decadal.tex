\section{Discussion}

Summary of our results: 
(1) our ecosystem model simulates an eastward (westward) shift of epipelagic biomass in the equatorial Pacific in response to El Niño (La Niña), in line with the shift of observed for tuna fishes catches during ENSO. 
(2) While the modelled response is similar for all size classes, processes responsible for these changes considerably vary as a function of size class: large -> transport of fish biomass from western to central Pacific by El Niño-related eastward surfac e current anomalies, intermediate -> central Pacific increase largely explained by advection but decrease in the west explained by reduced functional response due to lower SST and less food availability, small -> stronger importance of biological processes through growth rate changes in response to temperature changes. (3) Vertical???

Discussion:
(1) Comparison with other studies (for not saying Lehodey's studies): need to look at bibliography properly but my impression is that Lehodey attribute most of changes to active transport, where tuna move eastward towards more favorable grounds with warmer water and more food? Our simulation is able to accurately simulate biomass shift without a specific representation of tuna physiology (unrealistically sharp temperature boundaries) -> passive advection by ENSO related currents is sufficient to explain the eastward expansion of tuna towards central Pacific. These dynamical processes need to be accounted for in ecosystem models. Advantage of mechanistic models against shitty models. Fuck Lehodey!!!!!

(2) Limitation: Albeit favorable comparison, our model do not represent the peculiar physiology of tuna -> include focus species in APECOSM.  Problem with active transport? Etc...

(3) Future avenues: Our modelling framework allows for sensitivity experiments where interannual variations of main environmental drivers (T, currents, Food) can be shut down to separate their relative contribution thoughout the food chain. Look in details at the response of other communities to ENSO (migrants, meso). Investigate sensitivity of ecosystem response to ENSO diversity and other known climate modes in other basins: IO -> IOD. Climate change etc...

Conclusion: Importance to evaluate models involved in FishMIP to be more confident in the future response projected by these models in response to climate change and better understand their diversity.