% !TeX root = ../article-enso.tex

\section{Conclusion and discussion}
\label{sec:conclusion}

%Summary of our results:

\subsection{Conclusion}

% (1)
ENSO, the most energetic interannual climate mode on a  global scale, is known to strongly impact marine ecosystems through changes in habitat conditions (oxygen, temperature, light penetration), currents and food availability. In particular, tuna catch data point to a shift of epipelagic biomass from the equatorial western to the central Pacific in response to El Niño and an accumulation of the biomass in the extreme western equatorial Pacific in response to La Niña. These indirect and heterogeneous data, which do not solely reflect changes in fish populations, do not allow to address the mechanisms responsible for these changes. Here, we use a simulation from a mechanistic ecosystem model that captures the zonal shift of the epipelagic community in response to ENSO similar to the response of tuna catches and allows, through the analysis of the tendency terms of biomass changes, to unravel underlying mechanisms.

% (2)
Despite a relatively similar modeled response of epipelagic communities to El Niño among different size classes, characterized by a decrease in biomass in the western Pacific and an increase in the central Pacific, our analyses reveal that the processes responsible for these changes vary considerably by size. For large organisms,  eastward passive transport by El Niño-related eastward surface currents anomalies is largely responsible for the movement of organisms from the western to the central Pacific and dominates the effects of growth and predation, which are structurally weaker for large organisms. For intermediate-sized organisms, while the increase in biomass in the central Pacific is also largely explained by eastward advection by zonal currents anomalies, the decrease in biomass in the western Pacific is explained initially by increased predation by large organisms, and then by reduced growth due to a decrease in the functional response related to both colder waters and decreased food availability. For small organisms, changes in growth rate  induced by the influence of temperature on fish physiology are an important process, reinforcing the increase in biomass induced by passive horizontal transport in the eastern Pacific and the decrease in biomass induced by increased predation by intermediate organisms near the dateline.

% (3) 
%The model simulates important effects of ENSO on the vertical distribution of organisms, with a shallower and more pinched vertical distribution in the west and deeper and less pinched in the central and eastern tropical Pacific during El Nino events.

\subsection{Discussion}
%Discussion:

% (1) 
Previous studies (e.g. \citealp{lehodeyNinoSouthernOscillation1997, lehodeyPelagicEcosystemTropical2001}) have attributed the eastward biomass shift of skipjack tuna during El Niño to active swimming to track the eastward migration of favorable warm waters. Here we show that such a biomass shift can be realistically simulated without the need to specify temperature preferences that would lead to active movements of the tuna towards the most favorable waters. Passive horizontal transport by ENSO-related currents is indeed sufficient in our simulations to explain the eastward movement of tunas. Our study highlights that it is therefore essential that marine ecosystem models account for the dynamical role of ocean currents in shaping the spatial distribution of marine communities and their response to climate variability.

%(2) 
Our analysis demonstrates the added value of using a mechanistic ecosystem model to disentangle the role of the different processes controlling biomass changes and understand their interactions. The analysis of biomass tendency terms is particularly powerful to isolate the effects of dynamical processes (passive transport by currents) from those of biological processes (growth, reproduction and predation in particular), and understand how these different processes change with environmental conditions and with organism size. Thus, the mechanistic foundations of the APECOSM model, which is based on the DEB theory, are particularly well suited to an in-depth analysis of the processes involved.

%(3) 
While La Niña has long been largely considered a mirror image of El Niño, the study of asymmetries between these two phases of ENSO has recently become an important research topic (e.g., \citealt{anENSOIrregularityAsymmetry2020}). This interest is firstly driven by an amplitude asymmetry, where the most intense El Niño events reach much larger amplitudes than the most intense La Niña events, but also by a spatial structure asymmetry, where La Niña SST anomalies are shifted westward and have a wider meridional extent compared to that of El Niño (e.g. \citealt{takahashiENSORegimesReinterpreting2011}). In this study, we focused primarily on the ecosystem response to extreme El Niño events because of their dramatic ecological and socioeconomic consequences. However, our analyses indicate that the epipelagic response to La Niña events that typically follows extreme El Niño is far from a perfect mirror image of El Niño (Fig. 4d-f), with the increase in biomass associated with extreme El Niño located in the central/eastern equatorial Pacific, while the decrease in biomass associated with the following La Niña remains confined to the western Pacific. These asymmetries associated with this ecosystem response also appear to be considerably larger than those associated with the physical response (Figure 4a) or the chlorophyll response (Figure 4c). A more refined assessment of the asymmetries in the response of marine ecosystems to ENSO and their associated driving processes is outside the scope of this study but deserves to be explored in detail in the future.

% (4) 
Although the historical simulation from our ecosystem model compares favorably with observations, it nevertheless has a number of limitations that deserve discussion. First, the model configuration used here corresponds to the level of generality that has been used in FishMIP to date. It simulates a single generic community for epipelagic organisms and two mesopelagic communities. We thus did not implement any temperature limitation to be as generic as possible and do not take into account the specifics of tuna physiology. Since these specifics are likely to affect the results, configuring APECOSM to specifically represent tuna species would help to refine our findings for particular species. We were also surprised to find that the role of active movements is negligible compared to passive movements, even for large organisms. Since this result may suggest an underestimation of active transport in our simulation, it is important to estimate more precisely the value of the movement parameters used from tagging data for example, and to study the sensitivity of our results to the value of this parameter.

% (5) 
Among the future developments envisioned, our modelling framework allows for sensitivity experiments where the interannual variations in key environmental factors (temperature, currents, food) can be artificially frozen to separate their relative influence on the food web dynamics. We also plan to examine the ENSO-related response of mesopelagic communities that are also explicitely simulated by our model. Finally, we plan to extend our analysis to other regions subject to significant climate variations, such as the Indian Ocean, which is home to the Indian Ocean Dipole (IOD), and to the analysis of climate change effects at the global scale. While the latter has been the focus of several recent studies (e.g. \citealp{lotzeGlobalEnsembleProjections2019, tittensorNextgenerationEnsembleProjections2021}), and the factors responsible for the strongest climate impacts discussed (e.g. \citealp{heneghanDisentanglingDiverseResponses2021}), a finer mechanistic analysis of the bio-ecological processes that climate change would bring into play in global marine ecosystems has not yet been conducted.

% Conclusion:
Reliable estimations of the magnitude of the impact of climate change on marine ecosystems and associated ecosystem services requires reliable numerical projections. Significant progress has been made in terms of modeling marine ecosystems and using them to project the impact of possible future climate change and construct relevant ensemble analyses (e.g. \citealp{lotzeGlobalEnsembleProjections2019, tittensorNextgenerationEnsembleProjections2021}). These analyses have notably contributed to the work of the IPCC \citep{portnerIPCCSpecialReport2019, portnerClimateChange20222022} and IPBES \citep{brondizioGlobalAssessmentReport2019} and it is important that the scientific community maintains this effort. However, the ability of the models used in these projections to reproduce the effects of past climate variability on ecosystems has not been thoroughly assessed yet, in particular due to the lack of relevant synoptic observations of high trophic levels. Yet, such an assessment is necessary and should be conducted to increase our confidence in these projections.

Finally, the magnitude of the expected climate change is such that marine ecosystems will operate in states without known analogues in the past. Mechanistic studies based on the fundamental principles governing the effects of past climate variability on marine ecosystems are very important in this regard. They help us to better understand the mechanisms that will be at play in those future no-analogue situations, when projections based on statistical analysis may become invalid. This can only increase our confidence in the future response projected by integrated ecosystem models to climate change, and allow us to better understand their diversity.