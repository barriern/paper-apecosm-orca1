\section{Conclusion and discussion}
\label{sec:conclusion}

%Summary of our results:

\subsection{Conclusion}

% (1)
ENSO, the largest interannual climate mode, is known to strongly impact marine ecosystems through changes in habitat conditions (oxygen, temperature, light penetration), currents and food availability. In particular, tuna catches data monitor a shift of epipelagic biomass from the equatorial western to the central Pacific in response to El Niño and an accumulation of the biomass in the extreme western equatorial Pacific in response to La Niña but these indirect and heterogeneous data do not only reflect fishes population changes and do not allow to address the mechanisms responsible for these changes. Here, we use a simulation from a mechanistic ecosystem model which captures the zonal migration of the epipelagic community in response to ENSO similar to those described by tuna catches  and allow, through the analysis of on-line tendency terms of biomass changes, to unravel the mechanisms responsible for this typical behaviour.

% (2)
While the modelled response to El Niño is relatively similar for all size classes, with a decrease in biomass in the western Pacific and an increase in the central Pacific, our analyses reveal that the processes responsible for these changes vary considerably with size. Passive transport of large fishes by El Niño-related eastward surface currents anomalies from the western to central Pacific dominates the effects of growth and predation anomalies, which are structurally weaker for large organisms. For intermediate-sized fishes, the biomass eastward shift is also largely explained by eastward advection by zonal currents anomalies but the decrease in growth flux also significantly contributes to the decrease in the western Pacific. For small organisms, the effect of temperature-related biological processes (primarily growth-associated biomass production and predation mortality) dominates the effect of passive horizontal transport.

% (3) 
The model simulates important effects of ENSO on the vertical distribution of organisms, with a shallower and more pinched vertical distribution in the west and deeper and less pinched in the central and eastern tropical Pacific during El Nino events.

\subsection{Discussion}
%Discussion:

% (1) 
Previous studies (e.g. \citealp{lehodeyNinoSouthernOscillation1997, lehodeyPelagicEcosystemTropical2001}) attribute the eastward biomass shift of skipjack tuna during El Niño to active swimming following the eastward migration of favorable warm waters. Here we show that such a biomass shift can be realistically simulated without the need to specify temperature preferences that would lead to active movementsof the tuna towards these preferred waters. Passive horizontal transport by ENSO-related currents is indeed sufficient in our simulations to explain the eastward movement of tunas that has been demontrated by tagging studies and the simultaneous expansion of tuna catches towards central Pacific that is observed during ENSO events. Our study highlights that it is therefore essential that marine ecosystem models account for the dynamical role of ocean currents in shaping the spatial distribution of marine communities and their response to environmental variability.

%(2) 
Our analysis demonstrates the added value of using mechanistic ecosystem models to disentangle the role of the different processes controlling biomass changes and understand their interactions. Biomass trends analyses are particularly useful to isolate the effects of dynamical processes (passive transport by currents) from those of biological processes (growth, reproduction and predation in particular), and understand how these different processes change with environmental conditions and with  organism size. Thus, the mechanistic foundations of the APECOSM model, which is based on the DEB theory, are particularly well suited to an in-depth analysis of the processes involved.

% (3) 
Although the historical simulation from our ecosystem model compares favorably with observations, it nevertheless has a number of limitations that deserve to be discussed. First, the model configuration used here represents a single community of epipelagic organisms, with asymptotic maximum sizes ranging from 5 cm to 2 m. Thus, we did not implement any temperature limitation to be as generic as possible and do not take into account the particularities of tuna physiology. Since these specifics are likely to affect the results, configuring APECOSM to specifically represent tuna species could likely refine our findings. We were also surprised to find that the role of active movements is negligible compared to passive movements, even for large organisms. As this result may suggest an underestimation of active transport in our simulation, it is important to estimate more precisely the value of the movement parameters used from tagging data for example, and to study the sensitivity of our results to the value of this parameter.

% (4) 
Amongst possible future developments, our modelling framework allows for sensitivity experiments where the interannual variations of the main environmental factors (temperature, currents, food) can be artificially blocked to separate their relative impacts on the food chain dynamics. We could also examine at the response to ENSO of mesopelagic communities that are simulated by the model but that we did not consider in the present analysis. Finally, we plan to extend our analysis to other regions subject to significant climate variations, such as the Indian Ocean, which is home to the Indian Ocean Dipole (IOD), and to the analysis of climate change effects at global scale. While these have indeed been highlighted in several recent studies (e.g. \citealp{lotzeGlobalEnsembleProjections2019, tittensorNextgenerationEnsembleProjections2021}), and the factors responsible for the strongest climate impacts well identified (e.g. \citealp{heneghanDisentanglingDiverseResponses2021}), a fine mechanistic analysis of the bio-ecological processes that climate change will bring into play in global marine ecosystems has not yet been conducted.

% Conclusion:
Reliable estimation of the magnitude of the impact of climate change on marine ecosystems and associated ecosystem services requires the use of reliable numerical projections. Significant progress has been made in terms of modeling marine ecosystems and using them to project the impact of possible future climate change and construct relevant ensemble analyses (e.g. \citealp{lotzeGlobalEnsembleProjections2019, tittensorNextgenerationEnsembleProjections2021}). These analyses have notably contributed to the work of the IPCC \citep{portnerIPCCSpecialReport2019, portnerClimateChange20222022} and IPBES \citep{brondizioGlobalAssessmentReport2019} and it is important that the scientific community maintains this effort. However, the ability of the models used in these projections to reproduce the effects of past climate variability on ecosystems has not been thoroughly assessed, particularly due to the lack of relevant synoptic observations of high trophic levels. Yet, such an assessment is necessary and should be conducted to increase our confidence in these projections.

Finally, the magnitude of the expected climate change is such that marine ecosystems will operate in states without known analogues in the past. Mechanistic studies based on the fundamental principles governing the effects of past climate variability on marine ecosystems are very important in this regard. They help us to better understand the mechanisms that will be at play in those future no-analogue situations, when projections based on statistical analysis may become invalid. This can only increase our confidence in the future response projected by integrated ecosystem models to climate change, and allow us to better understand their diversity.