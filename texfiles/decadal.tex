\section{Discussion}

Summary of our results:

(1) our coupled ecosystem model simulates an eastward (westward) shift of epipelagic biomass in the equatorial Pacific in response to El Niño (La Niña), in line with the known shift of tuna catches by the industrial purse seine fleet in the Western Pacific Ocean during ENSO (Lehodey, 1997).

(2) While the modelled response is relatively similar for all size classes, with a biomass decrease in the western part of the ocean and an increase in the central Pacific, the processes that are responsible for these changes vary considerably with size. The passive transport of large fish from western to central Pacific waters by El Niño-related eastward surface current anomalies dominate over the effects of growth and predation anomalies that are structurally weaker for large sized organisms. For fish of intermediate size, the central Pacific increase is still largely explained by the eastward advection by marine currents as well as the decrease in the west that is further amplified by the decreased growth flux. For small organisms, the effect of temperature driven biological processes (primarily the biomass production associated to growth and the predation mortality) dominate over the effect of passive horizontal transport.

(3) The model simulates important effects of ENSO on the vertical distribution of organisms, with a shallower and more pinched vertical distribution in the west and deeper and less pinched in the central and eastern tropical Pacific during El Nino events.


Discussion:

(1) Previous studies (e.g. Lehodey, 1997, 2001) attribute skipjack tuna biomass changes during El Niño to an active eastward swimming towards favorable warm waters. Here we show that such a biomass shift can be accurately simulated without the need to specify temperature preferences that would drive active movements with potentially unrealistic temperature limits. Passive horizontal transport by ENSO related currents is indeed sufficient in our simulations to explain the eastward movement of tunas that has been highlighted by tagging studies (REF) and the simultaneous expansion of tuna catches towards central Pacific that is observed during ENSO events (REF). It is therefore critical that marine ecosystem models account for the dynamical role of ocean currents in shaping the spatial distribution of marine communities and their response to environmental variability.

(2) Our analysis shows the value of using mechanistic ecosystem models to disentangle the role of different processes and understand their interactions. In particular, trend terms are particularly useful to distinguish the effects of dynamical processes (passive transport by currents) from biological processes (growth, reproduction and predation in particular), their interactions, spatial heterogeneity, and how these different processes change with environmental conditions and with the size of organisms. Thus, the mechanistic underpinnings of the APECOSM model, which is based on the DEB theory, lend themselves particularly well to an in-depth analysis of the processes involved.

(3) Although the simulation produced by our model matches the observations quite well, it has a number of limitations that should be considered in the future and that could affect our results. In particular, the model configuration used here represents one community of epipelagic organisms, with asymptotic maximum sizes ranging from 5 cm to 2 m. We did not implement any temperature limitation to be as generic as possible and the particularities of tuna physiology are not represented. In the future it would be interesting to configure and parameterize APECOSM to represent specifically particular tuna species, which would probably allow to refine our analysis. Among other potential problems, we were surprised to find that the role of active movements is negligible compared to passive movements, even for large organisms. Active transport is possibly underestimated in our simulation, it would be interesting to estimate more precisely the value of the motion parameters used, using tagging data for example, and to study the sensitivity of our results to the value of the motion parameters used.

(4) Amongst possible future developments, our modelling framework allows for sensitivity experiments where the interannual variations of the main environmental drivers (T, currents, Food) can be artificially shut down to separate their relative impacts on the food chain dynamics. We could also look at the response to ENSO of the mesopelagic communities that are simulated by the model but that we didn't consider in the present analysis. Finally, we plan to extend our analysis to other regions subject to significant interannual variability, such as the Indian Ocean, which is subject to the effects of the Indian Ocean Dipole (IOD), and to the global analysis of climate change effects. If these have indeed been shown in several recent studies (e.g. Lotze et al., 2019; Tittensor et al., 2021), and the factors responsible for the strongest climate impacts well identified (e.g. Hennegan et al., 2021), a fine mechanistic analysis of the bio-ecological processes that climate change will bring into play in global marine ecosystems has not yet been conducted.

Conclusion:
The likely magnitude of the impact of climate change on marine ecosystems and associated ecosystem services requires that we have reliable projections. Significant progress has been made in terms of modeling marine ecosystems and using them to project the impact of possible future climate change and build relevant ensemble analyses (e.g. Lotze et al., 2019; Tittensor et al., 2021). These analyses have notably contributed to the work of the IPCC (REF SROCC 2019 and IPCC WG2 2022) and IPBES (REF GLOBAL ASSESSMENT 2019) and it is important that the scientific community maintain this effort. However, to the best of our knowledge, the capacity of the models used in this projections to reproduce the effects of past climate variability on ecosystems has not been thoroughly analyzed or evaluated, notably due to the lack of relevant synoptic observations of high trophic levels. Yet, such an assessment is necessary and must be conducted in order to increase our confidence in these projections.
Moreover, the magnitude of the expected climate change is such that marine ecosystems will operate in states without known analogues in the past. Mechanistic studies based on the fundamental principles governing the effects of past climate variability on marine ecosystems are very important in this regard. They help us to better understand the mechanisms that will be at play in those future situations without analogues, when projections based on statistical regressions may become invalid. This can only but increase our confidence in the future response projected by integrated ecosystem models in response to climate change, and better understand their diversity.