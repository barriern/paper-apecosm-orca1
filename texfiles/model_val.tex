% !TeX root = ../article-enso.tex

\section{Evaluation of the modeled response to ENSO}
\label{sec:model-val}

Before analysing the main processes responsible for the modeled response of epipelagic communities to ENSO (next sections), we first assess the ability of the physical, biogeochemical and biological models that we use to reproduce ENSO-related fluctuations. Previous studies have already demonstrated the ability of NEMO-PISCES to reproduce many aspects of the physical (e.g., \citealt{vialardModelStudyOceanic2001, lengaigneMechanismsControllingWarm2012, drushkaProcessesDrivingIntraseasonal2015, puyModulationEquatorialPacific2019}) and biogeochemical (e.g., \citealt{ masottiLargescaleShiftsPhytoplankton2011,gorguesRevisitingNina19982010, martinezReconstructingGlobalChlorophylla2020}) response to ENSO in the tropical Pacific. In the following subsection, we briefly demonstrate the ability of our simulation to capture ENSO-related signals that are important to marine ecosystems, namely surface temperature (which modulates the functional response to prey as well as all the metabolic rates controlling growth, reproduction, development, maintenance and swimming speed in APECOSM), sea level anomalies (a proxy for thermocline depth, which modulates vertical habitats of epipelagic species), surface currents (which passively transport simulated biomass) and chlorophyll concentration anomalies (a proxy for primary production that fuels the food chain).

\subsection{Physical response}

 Figure \ref{fig:nemo-had-sst}a-c first assesses the ability of our simulation to reproduce the ENSO signature in sea surface temperature (SST). Figure \ref{fig:nemo-had-sst}a presents the temporal evolution of ENSO as observed and simulated by the physical model using the Oceanic Niño Index (hereafter ONI\footnote{\url{https://www.cpc.ncep.noaa.gov/data/indices/oni.ascii.txt}}), computed from a 3-month running mean of SST anomalies averaged over the Niño 3.4 region (5N-5S, 170W-120W). Over the entire period considered (1958-2018), the ONI index exceeds 2\degree{}C only on three occasions, corresponding to the three most intense \nino{} events observed over the period considered (1982/83, 1997/98 and 2015/16). Other smaller \nino{} events are also observed in 1986/87, 1991/92, 2002/03 and 2009/2010, with ONI values ranging between 1\degree{}C and 2\degree{}C. Major \nina{} events are observed in 1970/71, 1973/74, 1988/89, 1999/2000, 2007/08 and 2010/11. This panel also reveals that the model is able to accurately simulate the timing and amplitude of ENSO events, as shown by the strong correlation (0.92) between observed and modeled ONI indices, significant at the $95\%$ level of confidence (based on a Student t-test with an effective number of degrees of freedom that is corrected based on the 1 month-lag autocorrelation of each time-series, as reported in \citealp{brethertonEffectiveNumberSpatial1999}). Despite this very good general agreement, the model tends to overestimate the amplitude of the strongest \nino{} events. 

\begin{figure}[h!tp]
	\centering
	\includegraphics[scale=0.6]{figs/fig1.png}
	\caption{Time evolution of the ONI index for observations and model over the 1958-2018 period (a). ENSO-related SST patterns for observations \citep{raynerGlobalAnalysesSea2003} (b) and model (c) derived from covariance maps of detrended monthly SST anomalies onto the ONI index over the 1958-2018 period. Time evolution of zonal surface current anomalies over the Niño34 region for observations over the 1993-2018 period \citep{rioGOCEOceanCirculation2014} and model over the 1958-2018 period (d). ENSO-related sea-level and ocean current patterns for observations (e) and model (f) derived from covariance maps of detrended monthly sea-level and current anomalies onto the ONI index over the 1993-2018 period. The dashed box represents the Niño34 region used for averaging.}
	\label{fig:nemo-had-sst}
\end{figure}

Figure \ref{fig:nemo-had-sst}b-c then illustrates typical spatial SST patterns associated with ENSO for the observations (HadISST1, \citealp{raynerGlobalAnalysesSea2003}) and the model, based on the covariance maps of detrended monthly SST anomalies onto the ONI index. The observed and modelled SST patterns are very similar and are characterized by warm SST anomalies (1\degree{}C) located in the central and eastern equatorial Pacific, flanked by the traditional horseshoe cooling pattern in the western Pacific that extends into the northern and southern subtropical Pacific.    


ENSO-induced SST variations are known to be strongly related to variations in ocean currents and sea level (a proxy for thermocline depth), with SST signals largely driven by vertical displacement of the equatorial thermocline in response to equatorial wind variations. Figure \ref{fig:nemo-had-sst}d illustrates the temporal evolution of zonal current anomalies averaged over the Niño34 region for observations\footnote{\url{https://doi.org/10.48670/moi-00050}} (\citealp{rioGOCEOceanCirculation2014}) available from 1993 to present, and the model. The model faithfully reproduces the observed anomalies (significant correlation of 0.89), with eastward currents anomalies during \nino{} events (reaching 0.7 m.s-1 at the peak of the 1982/83 and 1997/98 events) and westward currents anomalies during \nina{} events. As shown in Figure \ref{fig:nemo-had-sst}e, these easterly current anomalies during \nino{} are seen over the entire equatorial Pacific between 2N and 5S, a spatial structure well captured by the model (Figure \ref{fig:nemo-had-sst}f). With respect to sea-level, its ENSO related observational signature\footnote{\url{https://doi.org/10.48670/moi-00148}} (Figure \ref{fig:nemo-had-sst}e) is characterized by a shoaling of the thermocline in the western Pacific (negative sea level anomalies) and a deepening in the central and eastern Pacific (positive sea level anomalies), a signal that is physically consistent with the cooling observed in the west and the warming in the east (Figure \ref{fig:nemo-had-sst}b). As shown in Figure \ref{fig:nemo-had-sst}f, the model captures this zonal sea-level tilt very accurately.

\subsection{Biogeochemical response}

Figure \ref{fig:nemo-sat-chl}a-c assesses the ability of our simulation to capture ENSO-related variability of cholorophyll concentration. To this end, we compare simulated chlorophyll concentrations with multi-satellite monthly CHL-a estimates from the OceanColour-CCI V5  dataset\footnote{\url{http://dx.doi.org/10.5285/1dbe7a109c0244aaad713e078fd3059a}} \citep{sathyendranathOceanColourTimeSeries2019}, available over the 1997-09/2018-12 period. 
%This high resolution (4 km)  dataset is regridded on a regular $1\times 1$ grid by computing weighted chlorophyll averages over $24\times24$ boxes, with the weights being provided by the cosine of latitude. When more than $1/3$ of the data used in the averaging is missing, the regridded cell is masked.

\begin{figure}[h!tp]
	\centering
	\includegraphics[scale=0.4]{figs/fig2.png}
	\caption{Time evolution of monthly surface chlorophyll anomalies in the equatorial Pacific for observations over the  1998-2018 period(yellow curve) and  model over the  1960-2018 period (black curve) (a). Covariance between the chlorophyll anomalies and the ONI index over the 1998-2018 period for observations (b) and model (c). The dashed box represents the equatorial region used in the averaging.}
	\label{fig:nemo-sat-chl}
\end{figure}

Figure \ref{fig:nemo-sat-chl}a represents the temporal evolution of chlorophyll anomalies averaged over the equatorial Pacific for the model and observations. In agreement with past literature, \nino{} events are associated with a decrease in chlorophyll all along the equator (Figure \ref{fig:nemo-sat-chl}a) in response to the combined action of nutricline deepening in the eastern Pacific and eastward advection of nutrient‐poor waters by anomalous eastward currents in the western and central Pacific. The reverse occurs during \nina{} events. As a result, equatorial chlorophyll anomalies are strongly anti-correlated with variations in Niño34 SST (R=-0.74) and sea level (R=-0.78). The model faithfully reproduces these observed chlorophyll variations, with a correlation coefficient between the observed and simulated time series reaching $0.80$ (significant at the $95\%$ level of significance).

Figure \ref{fig:nemo-sat-chl}b-c show typical spatial patterns of ENSO-associated surface chlorophyll anomalies for the observations and the model over their common period. In agreement with Figure \ref{fig:nemo-sat-chl}a, \nino{} causes a decrease in chlorophyll concentration along the equator east of 150\degree{}E. Despite an overestimation of the modeled chlorophyll decrease in the eastern Pacific and an underestimation off Panama, the observations (upper panel) and the model (lower panel) show similar patterns. Note that recalculating the simulated spatial pattern over the entire modeled period (1958-2018) gives similar patterns (not shown). 

\subsection{Ecosystem response}

In this section, we compare the evolution of the simulated epipelagic biomass to available observations in the equatorial Pacific. As mentioned in the introduction, the largest set of interannual observations of high trophic level marine organisms in the equatorial Pacific is based on tuna catches. Here we use monthly catches of skipjack and yellowfin tuna by purse seiners provided at 1° and 5° spatial resolution by the Western and Central Pacific Fisheries Commission (WCPFC) and processed by the French National Research Institute for Sustainable Development (IRD), as described in  \cite{taconetGlobalMonthlyCatch2018}\footnote{\url{https://doi.org/10.5281/zenodo.1164128}}. We first extract skipjack and yellowfin purse-seine catches from the raw input file. We then discard observations with a temporal resolution greater than one month and data for which the geographical coordinates are not referenced in the database. The remaining observations are finally binned onto a regular $1\degree{} \times 1\degree{}$ grid. The final product consists of monthly maps of tuna catches covering the 1959-2018 period. However, due to the limited spatial coverage of the purse-seine fleets in the early part of the record, we only analyse this dataset from 1985 onwards. We compare catch observations to the biomass of the epipelagic community integrated from 30cm to 70cm, the typical size range of  skipjack and yellowfin tunas caught by purse seiners in this region. In making this comparison, it should be kept in mind that fishing data not only depend on the available fish biomass but are also influenced by both climate variability and the many socio-economic factors that control the dynamics and distribution of fishing effort \citep{hobdayDetectingClimateImpacts2013}. Furthermore, although tuna represent the majority of epipelagic biomass in this size range in this region, the configuration of the model that we use here does not explicitly represent specific tuna species but generic oceanic communities such as the epipelagic community that we study here.
% Therefore, these data  highly biased estimators of the actual biomass of fish swimming in the ocean that are 

\begin{figure}[h!tp]
	\centering
	\includegraphics[scale=0.4]{figs/plot_validation_apecosm.png}
	\caption{Time-longitude diagram of observed catches (colors, log-scale in Tons) and simulated epipelagic biomass (contours, log-scale in Tons) cumulated between 10\degree{}N and 10\degree{}S (a). Difference between \nino{} and \nina{} composites over the 2007-2018 period for observed catches (b) and simulated epipelagic biomass (c). Temporal evolution of the barycenters' longitudes of simulated epipelagic biomass (thick blue line) and observed catches (thin dashed-dotted green line) over the 1985-2018 period. The detrended catch barycenter is also shown (thick orange line) (d). Panels (a) and (d) are positionned so that their temporal axes are aligned. All panels have different x-axis.}
	\label{fig:apecosm_validation}
\end{figure}


Figure \ref{fig:apecosm_validation}a represents a longitude-time diagram of observed catches integrated between 10\degree{}N and 10\degree{}S over the 2008-2018 period. This panel highlights significant variations in the zonal extent of tuna catches in the equatorial Pacific. These catches are indeed confined to the west of the dateline during certain periods such as in 2008 and 2011, when \nina{} conditions prevail over the Pacific. In contrast, they extend eastward into the central Pacific for other periods such as 2009-2010 and 2014-2016 that are characterized by \nino{} conditions. Despite their different nature, the zonal extension of the modeled biomass compares surprisingly well with that of tuna catches over the recent period: \nina{} events of 2008 and 2011 are indeed characterized by a westward retraction of the epipelagic biomass, while \nino{} periods of 2009-2010 and 2014-2016 are characterized by a clear eastward extension of the epipelagic biomass. 

Figure \ref{fig:apecosm_validation}b-c show the differences in observed catches and simulated biomass between an \nino{} (2009-10/2010-03, 2014-10/2015-03, 2015-10/2016-03) and a \nina{} composite (2007-10/2008-03, 2008-10/2009-03, 2010-10/2011-03, 2011-10/2012-03). Consistent with Figure \ref{fig:apecosm_validation}a, it illustrates the typical east-west shift pattern that is associated with ENSO in the recent period. 
Observed catches are greater in the central and eastern Pacific and lower in the Western Pacific under \nino{} conditions compared to \nina{} conditions, corresponding to an eastward shift of the epipelagic biomass. Observed catches and simulated biomass composites show similar patterns, although slightly shifted westward in the model.

Figure \ref{fig:apecosm_validation}d then assesses the agreement between the observations and the model over a longer period. It shows the temporal evolution of the barycenters' longitudes of observed tuna catches and modeled biomass over the 1985-2018 period, smoothed by a Gaussian filter ($\sigma=3.5$ and truncation above $4\sigma$). Consistently with the good agreement between model and observations presented in Figure \ref{fig:apecosm_validation}a, the evolution of the model barycenter  is very consistent with that of observations over the last decade (2008-2018), with a correlation of 0.88 between the two time series. This is particularly the case for the 2014-2016 \nino{} sequence, where both model and observations indicate an eastward shift in the data barycenter from 160\degree{}E in early 2014 to 180\degree{}W in early 2016, before retracting westward after that date. Looking at the observations over the entire 1985-2018 period, the most striking feature is a gradual eastward shift in the location of the catches barycenter, from 155\degree{}E in the 80's to 170\degree{}E in the last decade. This trend is consistent with a global expansion of the industrial purse seine tuna fleet distribution over the last decades \citep{coulterUsingHarmonizedHistorical2020}. 

Industrial tuna fisheries have indeed expanded considerably since the 1950s (e.g. \citealt{ticklerFarHomeDistance2018, faoStateWorldFisheries2022}). This growth in production means and catches, accompanied by a considerable spatial extension, in particular due to the increase in the range of fishing vessels, is well documented on a global scale (e.g.: \citealt{fonteneauAtlasTropicalTuna1998}) and in the western Pacific \citep{lodgeDevelopmentPalauArrangement1998, williamsOverviewTunaFisheries2021} where it has notably translated into a progressive westward extension of the fishing areas of the purse seiners \citep{fonteneauAtlasTropicalTuna1998}. In order to take into account this geographical expansion when comparing the fisheries data with the model outputs over the long period, the observed time series has been detrended. The model and detrended observations show a reasonable match over the entire period, with a correlation of 0.45. The strong correlation coefficient of the ONI timeseries with the model biomass barycenter (0.74) as well as with the observed catches barycenter (0.52) further highlights the control exerted by ENSO in both observations and model, particularly during the 1986/87, 1997/98 or 2001/02 \nino{} events, when observed catches shift eastward. While observations and the model generally shows a westward shift during \nina{} conditions and an eastward shift during \nino{} conditions, these time series deviate from each other during specific periods, such as the strong \nina{} of 1999/2000, which is characterized by a stronger westward retraction in the model or during the warm years of 2003-2005 with the barycenter of observed catches shifting westward relative to that of the modeled biomass.

In summary, the analyses presented above illustrate the ability of the various model components to satisfactorily reproduce the physical, biogeochemical and ecosystem response to ENSO. In particular, the ecosystem model is able to simulate zonal shifts of the large organisms of the epipelagic community in response to ENSO in a manner similar to that observed for tuna catches. This agreement provides support for using our ecosystem simulation to study the processes responsible for the epipelagic community response to ENSO. In what follows, our results are detailed for three selected size classes: small epipelagic organisms (3 cm), intermediate sizes (20 cm), and large individuals (90 cm). The latter two are representative of the lower and upper limits of the size range of the tunas exploited by purse seiners in the region.

%\clearpage
