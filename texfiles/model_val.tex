\section{Evaluation of the modelled response to ENSO}

Before analysing the main processes responsible for the response of epipelagic communities to ENSO in the next section, we evaluate here the ability of the physical, biogeochemical and biological models to reproduce ENSO-related fluctuations. Past literature already demonstrated the ability of NEMO/PISCES model to reproduce many aspects of physical (e.g., \citealt{vialardModelStudyOceanic2001, lengaigneOceanResponseMarch2002}) and biogeochemical (e.g., \citealt{ masottiLargescaleShiftsPhytoplankton2011}) response to ENSO in the tropical Pacific . In the following subsection, we will briefly demonstrate the ability of our simulation to capture ENSO-related signals of importance for marine ecosystems, namely surface temperature (which modulate the functional response), sea-level anomalies (a proxy of the thermocline depth, which modulates vertical habitats of epipelagic species) and chlorophyll concentration anomalies (the building block of the trophic food chain). 

\subsection{Physical response}

We first describe in Fig\ref{fig:nemo-had-sst}a-c the ability of our simulation to reproduce the SST signature of ENSO. Fig\ref{fig:nemo-had-sst}a shows for both NEMO model and observations\footnote{\url{https://www.cpc.ncep.noaa.gov/data/indices/oni.ascii.txt}} the Oceanic Niño Index (hereafter ONI), a well-known index of ENSO temporal evolution computed from a 3-month running mean of SST anomalies averaged over the Niño 3.4 region (5N-5S, 170W-120W). Over the time period considered (1958-2018), the three most intense El Niño events are observed in 1982/83, 1997/98 and 2015/16, with ONI indices exceeding 2\degree{}C at the peak of these events. Other more moderate El Niño events occurred in 1986/87, 1991/92, 2002/03 and 2009/2010, with ONI indices between 1\degree{}C and 2\degree{}C. Major La Niña events occurred in 1970/71, 1973/74, 1988/89, 1999/2000 and 2007/08 and 2010/11. The model is able to accurately simulate the timing and amplitude of major El Niño and La Niña events, as demonstrated by the strong correlation between observed and model ONI indices, which is significant at the $95\%$ level of confidence (based on Student's t-test with an effective number of degrees of freedom that is corrected based on the 1-lag autocorrelation of each time-series, as discussed in \cite{brethertonEffectiveNumberSpatial1999}). Despite this general agreement, the model however overestimates the amplitude of the largest El Niño events. 

\begin{figure}[h!tp]
	\centering
	\includegraphics[scale=0.6]{figs/fig1.png}
	\caption{Time evolution of the ONI index in observations and model over the 1958-2018 period (a). ENSO-related SST patterns for observations (REF-HadSST) (b) and model (c) derived from covariance maps of detrended monthly SST anomalies onto the ONI index. Time evolution of sea-level anomalies over the Niño34 region for observations (REF-SLA) over 1993-2018 and model over 1958-2018 (d).ENSO-related sea-level patterns for observations (REF-SLA) (b) and model (c) derived from covariance maps of detrended monthly sea-level anomalies onto the ONI index. The dashed box represents the Nino34 region used in the averaging.}
	\label{fig:nemo-had-sst}
\end{figure}

Fig.\ref{fig:nemo-had-sst}bc further shows the typical SST patterns associated with ENSO for both observations (HadISST1, REF) and model by displaying the covariance map of detrended observed monthly SST anomalies onto the ONI index. Observed and modelled SST patterns closely resemble each other and are characterized by warm SST anomalies (1\degree{}C) centred in the central and eastern equatorial Pacific  flanked by the traditional horseshoe cooling pattern in the western Pacific extending towards the subtropical north and south Pacific.    


ENSO SST variations are known to be strongly related to sea level variations (a proxy for the thermocline depth), with SST signals largely driven by vertical displacement of the equatorial thermocline in response to equatorial wind variations (REF-ENSO). Figure \ref{fig:nemo-had-sst}d shows the temporal evolution of the sea level anomalies averaged over the Niño34 region for both the model and observations from the DUACS multi-mission altimeter data \footnote{\url{https://doi.org/10.48670/moi-00148}} available from 1993 to present. The temporal evolution of observed sea level anomalies matches very well the observed SST evolution over the same region (Fig \ref{fig:nemo-had-sst}a), both being correlated at ?.??: the deepening of the thermocline observed in the central and eastern Pacific during El Niño events in response to westerly wind anomalies indeed contribute to El Niño warming by limiting the amount of cold waters brought to the surface, the opposite occurring during La Niña events.  The model accurately reproduces the observed sea level evolution (Fig \ref{fig:nemo-had-sst}d), with a correlation coefficient reaching 0.94 (significant at the $95\%$ level of confidence). In addition,  spatial pattern of observed sea-level anomalies associated with ENSO (Fig \ref{fig:nemo-had-sst}e) is characterized by shallowing of the thermocline  in the western Pacific (negative sea level anomalies) and deepening of the thermocline  in the central and eastern Pacific (positive sea level anomalies), which is physically consistent with the cooling observed in the west and the warming in the east (Fig \ref{fig:nemo-had-sst}b). As shown on Figure \ref{fig:nemo-had-sst}f, the model captures this zonal sea-level seasaw very accurately.

\subsection{Biogeochemical response}

We then evaluate in Fig\ref{fig:nemo-had-sst}d-f the ability of our simulation to capture cholorophyll variability associated with ENSO variations. Simulated chlorophyll are compared with observation-based estimates derived from the multi-satellite monthly OceanColour-CCI V5 CHL-a dataset\footnote{\url{http://dx.doi.org/10.5285/1dbe7a109c0244aaad713e078fd3059a}} \citep{sathyendranathOceanColourTimeSeries2019} available over the 1997-09/2018-12 period. This high resolution (4 km)  dataset is regridded on a regular $1\times 1$ grid by computing weighted chlorophyll averages over $24\times24$ boxes, with the weights being provided by the cosine of latitude. When more than $1/3$ of the data used in the averaging is missing, the regridded cell is masked.


\begin{figure}[h!tp]
	\centering
	\includegraphics[scale=0.4]{figs/fig2.png}
	\caption{Simulated (black) and observed (yellow) equatorial chlorophyll anomalies (a). Filtered anomalies are shown for the model in panel (b). Covariance between the observed and simulated chlorophyll anomalies with the ONI (c, d) and filtered TPI (e) index. The dashed box represents the equatorial region used in the averaging.}
	\label{fig:nemo-sat-chl}
\end{figure}

Fig \ref{fig:nemo-sat-chl}a shows the temporal evolution of the chlorophyll anomalies averaged in the equatorial Pacific for the model and observations. As already been widely discussed in the literature, El Niño events are associated with a decrease in chlorophyll all along the equator through the combined action of the nutricline deepening in the eastern Pacific and eastward advection of nutrient‐poor waters by anomalous eastward currents in the western and central Pacific. The reverse occurs during major La Niña events. As a result, equatorial chlorophyll anomalies are strongly correlated with Niño34 SST (R=?.??) and sea-level (R=?.??) variations. The model accurately accounts for these observed chlorophyll variations, with a correlation coefficient reaching $0.80$ (significant at the $95\%$ level of significance).

Fig \ref{fig:nemo-sat-chl}bc further shows the typical spatial patterns of surface chlorophyll anomalies associated with ENSO for the observations and the model over their common period. In agreement with \ref{fig:nemo-sat-chl}a, El Nino drives a decrease in chlorophyll concentration along the equator east of 150\degree{}E. Despite an overestimation of the observed chlorophyll decrease in the eastern Pacific, the observations (upper panel) and the model (lower panel) show similar patterns. Note that recalculating the simulated spatial pattern over the entire modeled period (1958-2018) does not significantly alter the results (not shown). 

\subsection{Ecosystem response}

Before analysing the mechanisms driving the epipelagic community reponse to ENSO, we also compare the evolution of the simulated epipelagic biomass to available observations in the equatorial Pacific. As mentioned in the introduction, the largest observational dataset is based on tuna catches in this region. We therefore use the IRD Level 2 global monthly catch of tuna, tuna-like and shark species dataset\footnote{\url{https://doi.org/10.5281/zenodo.1164128}} \citep{taconetGlobalMonthlyCatch2018}. We first extract skipjack and yellowfin purse-seine catches from the raw input file. We then discard observations with a temporal resolution greater than one month and data for which the geographical coordinates of the catch location are not referenced in the database. The remaining observations are finally regridded onto on a regular $1 \times 1$ grid. The final product consists of monthly tuna catches maps  covering the period 1959-2018. However, due to the very poor data coverage in the early part of the record, we only analyse this dataset from 1985 onwards. These catch observations are compared to the biomass of the integrated epipelagic community between 30cm to 70cm, the typical size of  skipjack and yellowfin tunas caught in this region. In making this comparison, it should be kept in mind that tuna catches are not only controlled by climate variability but also by fishing pressure. Furthermore, although tuna represent the majority of the epipelagic biomass in this size range, it should be noted that our model does not explicitly simulate the physiology of specific tuna species but rather the generic behaviour of the epipelagic community.

\begin{figure}[h!tp]
	\centering
	\includegraphics[scale=0.35]{figs/plot_validation_apecosm.png}
	\caption{Time-longitude diagram of observed catches (colors, log-scale in Tons) and simulated epipelagic biomass (contours, log-scale in Tons) integrated between 10N and 10S (a). Catch (b) and simulated fish-biomass (c) difference between El Nino and La Nina composites.}Temporal evolution of the barycenters of simulated epipelagic biomass (black), observed catches (blue) and detrended observed catches (yellow) (d). 
	\label{fig:apecosm_validation}
\end{figure}

Figure \ref{fig:apecosm_validation}a shows a longitude-time diagram of observed catches integrated between 10N and 10S over the period 2008-2018. This analysis highlights significant variations in the zonal extension of the tuna catch area on the equatorial Pacific. These catches are indeed confined to the west of the dateline during certain periods like in 2008 and 2011, when La Niña conditions prevail over the Pacific. In contrast, the catch area extends eastward into the central Pacific for other periods such as 2009-2010 and 2014-2016, characterized by El Niño conditions. Despite the different nature of observed catches and modeled epipelagic biomass, the evolution of the zonal extension of modeled biomass compares surprisingly well with that of tuna catches over this recent period: the La Niña events of 2008 and 2011 are indeed characterized by a westward retraction of epipelagic biomass, while the El Niño periods of 2009-2010 and 2014-2016 are characterized by a clear eastward extension of epipelagic biomass. Figures \ref{fig:apecosm_validation}c and \ref{fig:apecosm_validation}d illustrate the typical pattern biomass/catches shift associated with ENSO during this recent period, displaying the difference between composites of catches and simulated biomass for El Niño conditions (2009-10/2010-03, 2014-10/2015-03, 2015-10/2016-03) and La Niña conditions (2007-10/2008-03, 2008-10/2009-03, 2010-10/2011-03, 2011-10/2012-03). Observed catches increase in the central and eastern Pacific and decrease in the Western Pacific between La Niña to El Niño conditions, consistent with an eastward shift of the tuna catches. the simulated composite also shows a similar pattern, although the pattern is slighly shifted westward from that observed.

To infer whether this agreement holds true over a longer period, Figure \ref{fig:apecosm_validation}b shows the barycenters of the longitudinal position of tuna catches and model biomass over the period 1985-2018. Consistent with the good agreement between model and observations presented in Figure \ref{fig:apecosm_validation}a, the evolution of model and observations barycenters matches very well over the most recent 2008-2018 period, with a correlation of 0.?? between the two timeseries: this is notably the case for the 2014-2016 El Niño sequence, where both model and observations indicate a barycenter eastward shift from 160°E in early 2014 to 180°W in early 2016, before retracting westward after that date. Looking at the observations over the entire observational record, the most striking feature is a gradual eastward shift in the location of the catches barycenter from 155°E in the 80s to 170°E in the last decade. This trend is probably due to the increased power of fishing fleets, which allows them to move further away from their home ports, mostly located in the western Pacific (\warn{ref}). In addition to this low-frequency upward trend, higher frequency variations are also evident, especially during the 1986/87, 1997/98 or 2001/02 El Niño events, where observed catches shift eastwards. in order to perform the fairest comparison with model results, we detrended the observed timeseries to smooth out the influence of the influence of the increase in fishing effort (yellow curve on Figure \ref{fig:apecosm_validation}b). The model and detrended observational timeseries show a reasonable match over the entire time period, with a correlation of 0.??. Their relationship with ENSO is demonstrated by the strong correlation coefficient existing between ONI timeseries and model biomass barycenter (0.??) as well as observed catches barycenter (0.??). While both data generally show a westward shift during La Niña conditions and an eastward shift during El Niño conditions, the two timeseries deviate during specific periods, such as during the strong  La Niña of 1999/2000, when the westward retractation is larger in the model or during the warm period of 2003-2005, when observed catches shift westward in contrast to model biomass. 

Overall, the evaluations presented above illustrate the ability of the various model components to accurately reproduce the physical, biogeochemical and ecosystem response to ENSO. More specifically, the ecosystem model simulate zonal migrations of large epipelagic communities in the equatorial Pacific in response to ENSO in a manner very similar to that observed for tuna catches. This favourable comparison indicates that our ecosystem simulation is a relevant tool to investigate the processes responsible the response of epipelagic communities  to ENSO. In what follows, the results are detailed for three selected size classes: 3cm, representing small epipelagic fishes, 20cm, representing intermediate sizes, and 90 cm, representing large individuals. The latter two are representative of the lower and upper size limits of target tuna  species harvested in the region (\warn{REF}).

\clearpage