\section{Discussions}

In the present paper, we analysed the response of fish biomass to changes in the physical and biogeochemical ocean in response to ENSO variability. The Apecosm model was forced by the outputs of the NEMO/Pisces physical and biogeochemical model, with no retroaction of the former on the latter. One question that arises is whether the response of fish biomass and ocean biogeochemistry to ENSO variability would be the same in a two-way coupled simulation. 	Such coupling, which has already been used in \cite{aumontEvaluatingPotentialImpacts2018} to analyse the impacts of the diurnal vertical migration on ocean biogeochemistry, could be used to determine whether the inclusion of fish biomass modifies the response of ocean biogeochemistry to ENSO. And whether the coupling attenuates or strengthens the response of fish biomass. \warn{Improve the text}
 
The study relies on a global simulation at a relatively coarse horizontal resolution (around 1°), which is insufficient to explicitly represents the mesoscale features of the California Current System \citep{capetMesoscaleSubmesoscaleTransition2008} and of the Peru-Chile current system \citep{colasHeatBalanceEddies2012}. This may lead to a misrepresentation of temperature and biogeochemical variables in the upwelling systems, which will ultimately propagates to the higher trophic levels. Guiet et al. (in prep) have used Apecosm on a regional configuration of the California Current Ecosystem. They show that small sizes emerge from the coast, where they are trapped in mesoscale features where plankton biomass is abundant, and are advected westward while they grow.\\

In this study, the transient response of marine ecosystems has been investigated from covariance analysis using a single El Nino index. This methodology assumes that ENSO variability is linear, i.e. that positive ENSO phases (El Nino) are opposite to negative ENSO phases (La Nina), which is not the case \citep{larkinENSOWarmNino2002}. Furthermore, no distinction has been made between the Eastern Pacific Nino (EPN) and the Central Pacific Nino (CPN) events, which have different physical and biogeochemical signatures and causes. For instance, \cite{gierachBiologicalResponse19972012} have shown, using satellite observations and adjoint passive tracer simulations, that EPN causes a reduction chlorophyll-a in the Eastern Pacific, mainly due to weaker trade winds, reduced upwelling and vertical mixing, while CPN reduces chlorophyll-a in the Central Pacific, mainly through a stronger eastward advection of nutrient-depleted waters from the Western Pacific. These different type of El Nino events may therefore have different impacts on fish biomass.\\

In the present study, changes in fish biomass in relation with ENSO variability are potentially driven by changes in temperature, oxygen or plankton concentration. In order to distinguish these different effects, sensitivity simulations can be run, for instance by using a temperature climatology instead of an interannual one. \\

• Coupling NEMO-Pisces and impacts?  % done
• Regional simulations to better represents physical processes? %done
• Focus species (tuna), more suited for tropical pacific
• ENSO impacts in the other regions (cf. Racault)  
• Separation of CP/EP Ninos  % done
• Different communitities
• Sensitivity experiments (climPlk, climTemp)
• Impacts of other modes of variabilities (PDO, NAO, AR).

\warn{\cite{racaultImpactNinoVariability2017} for impact on plankton}

